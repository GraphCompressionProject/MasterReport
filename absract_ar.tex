\documentclass[12pt,a4paper]{report}
\usepackage{arabtex}
\usepackage[utf8]{inputenc}
\usepackage[LFE,LAE]{fontenc}
\usepackage[arabic]{babel}

\usepackage[top=2.5cm, bottom=2.5cm, left=2.5cm, right=2.5cm]{geometry}


\usepackage{fancyhdr}
\fancyhf{} % clear all header and footers
\renewcommand{\headrulewidth}{0pt} % remove the header rule
\fancyfoot[LE,LO]{\thepage} % Left side on Even pages; Right side on Odd pages
\pagestyle{fancy}
\fancypagestyle{plain}{%
  \fancyhf{}%
  \renewcommand{\headrulewidth}{0pt}%
  \fancyhf[ref,lof]{\thepage}%
}

%\pagenumbering{Roman}
\begin{document}


\begin{center}
\thispagestyle{plain}

	\par
	\textbf{
		\vskip 0.5in
		\LARGE ملخص
			 \\[0.35in]}
\end{center}


	\par
  % \newfontfamily\arabicfont[Script=Arabic]{Scheherazade}
\begin{otherlanguage}{arabic}

إننا نعيش في وقتنا الحالي واقعا يعرف تزايد في الكم المعرفي و المعلوماتي مما اقتضى و أوجب استعمال المنحنيات التي اصبحت واسعة الانتشار و اكتسحت عدة ميادين مختلفة انطلاقا من مواقع التواصل الاجتماعي و الاتصالات وصولا لميادين الكمياء و البيولوجيا .  إن  هذه الكمية الهائلة من المعلومات تلزم الرجوع إلى  تقنية قديمة قدم مجال  معالجة البيانات و التي تواجه تحديات جديدة : الضغط ... ضغط المنحنيات هو مجال يخضع فيه الرسم البياني الأولي لتحولات من اجل الحصول على نسخة أصغر  تسمح في  غالبية الحالات بتحسين الوقت اللازم لمعالجة و تحليل البيانات.\\  

من خلال هذا العمل سنقدم دراسة مفصلة عن الطرق المختلفة للصغط الموجودة بهدف استكمال تصنيفات و تحسينها في اطار التحضير لرسالة الماستر. سنركز على طرق الضغط المعتمدة على استخراج الانماط ، و كذلك الطرق المعتمده على الأشجار ، و من ثم نقترح انشاء منشورات ببليوغرافية على هاتين الفئتين ، كما سننشىء دراسات مقارنة بين أساليب هاتين 
 الفئتين وفقا لمعايير مختلفة.\\\\



 \textbf{ 
 كلمات مفتاحية:} ضغط الرسم البياني ، البيانات الكبيرة ، استخراج الأنماط ، أشجار ، الرسم البياني على الويب.
 
\end{otherlanguage}
 
\newpage

       \end{document}