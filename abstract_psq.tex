
\thispagestyle{plain}
\begin{center}
	\par
	\textbf{
		\vskip 0.5in
		\LARGE 
			Résumé \\[0.15in]
			\addcontentsline{toc}{chapter}{\numberline{}Résumé}
	}
\end{center}
	\par
	
	Nous vivons dans un monde où la quantité d'informations ne cesse d'augmenter et dont la bonne gestion implique l'utilisation des graphes qui se sont répandus dans différents domaines allant des réseaux sociaux et de communication jusqu'aux domaines de la chimie et de la biologie. Cette abondance de données générées fait appel à une technique aussi vieille que la discipline de traitement de données mais qui connait de nouveaux défis aujourd'hui : la compression. La compression de graphes est un domaine dans lequel le graphe initial subit des transformations pour en obtenir une version plus réduite et compacte permettant, dans la majorité des cas, d'effectuer les traitements dans un temps nettement meilleur.\\
	
Deux classes de méthodes de compression feront l'objet de notre étude: les méthodes de compression par extraction de motifs et les méthodes basées sur les k2-trees. De ce fait, nous proposons deux moteurs de compression chacun englobant une ou plusieurs méthodes de chaque classe. Le premier moteur permet de compresser un graphe à travers ses structures les plus denses. Tant dis que le deuxième moteur exploite les propriétés de la matrice d'adjacence pour obtenir une représentation compacte. Nous proposons aussi une méthode de compression destinée aux graphes dynamiques et qui se situe à l'intersection des deux classes étudiées. En effet, elle permet de compresser le graphe à travers ces structures les plus denses tout en gardant trace de l'erreur, généralement représentée sous forme de matrices creuses, dans une structure k2-trees.\\

Nous conclurons par une étude comparative des performances des différents algorithmes de compression existants et la méthode que nous proposons où nous nous sommes basées sur des métriques d'évaluation tels que : le taux de compression et le gain obtenu. Afin de pouvoir réaliser cette étude en toute objectivité, les méthodes de compression seront testées sur des benchmarks connus de graphes issus de divers domaines.\\\\\textbf{Mots Clés :} \textit{Compression de graphes, Big Data, Extraction de motifs, K2-trees, Graphe du Web.}
 
\newpage
\thispagestyle{plain}
\begin{center}
	\par
	\textbf{
		\vskip 0.5in
		\LARGE 
			Abstract \\[0.15in]
	}
\end{center}
	\par
    
    We live in a world where the amount of data is constantly increasing and whose good management involves the use of graphs that have spread in different fields from social and communication networks to the fields of chemistry and biology. This abundance of generated data calls for a technique that is as old as the discipline of data processing but which is facing new challenges today: compression. Graph compression is a field in which the initial graph undergoes transformations in order to obtain a smaller and more compact version allowing, in the majority of cases, to perform the processings in a much better time.\\

 Two classes of compression methods will be the focus of our study: pattern extraction compression methods and k2-tree based methods. As a result, we propose two compression engines each encompassing one or more methods of each class. The first engine makes  compress a graph through its densest substructures. While, the second engine exploits the properties of the adjacency matrix to obtain a compact representation. We also propose a compression method for dynamic graphs at the intersection of the two classes studied. Indeed, it allows to compress the graph through his most important substructures, while keeping track of the error, usually represented as sparse matrices, in a k2-trees structure.\\
 
 We will conclude with a comparative study of the performances of the different existing compression algorithms and the method we propose where we use based on evaluation the folowing metrics: the compression ratio, number of bits per vertex and processing time. In order to be able to carry out this study in all objectivity, the compression methods will be tested on known benchmarks of graphs coming from various domains. \\\\\textbf{Key words :} \textit{Graph compression, Big Data, Pattern extraction, K2-trees, Web graph.}

\newpage
 
 %%% Manque resume en arabe 

