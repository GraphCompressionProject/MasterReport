
\thispagestyle{plain}
\begin{center}
	\par
	\textbf{
		\vskip 0.5in
		\LARGE 
			Résumé \\[0.15in]
			\addcontentsline{toc}{chapter}{\numberline{}Résumé}
	}
\end{center}
	\par
    
    Nous vivons dans un monde où la quantité d'informations ne cesse d'augmenter et dont la bonne gestion implique l'utilisation des graphes qui se sont répandus dans différents domaines allant des réseaux sociaux et de communication jusqu'aux domaines de la chimie et de la biologie. 	Cette abondance de données générées fait appel à une technique aussi vieille que la discipline de traitement de données mais qui connait de nouveaux défis aujourd'hui : la compression. La compression de graphes est un domaine dans lequel le graphe initial subit des transformations pour en obtenir une version plus réduite et compacte permettant, dans la majorité des cas, d'effectuer les traitements dans un temps nettement meilleur.\\

 Ce mémoire de fin d'études traitent deux classes de méthodes de compression de graphes: les méthodes basées sur les arbres k2-trees et les méthodes basées sur l'extraction de motifs. Nous proposons, en premier lieu, d'enrichir un projet entamé dans un PFE précédent. Nous établirons pour cela deux moteurs de compression : $k^2$-GraCE pour $k^2$-trees Graph Compression Engine et P-GraCE pour Pattern Graph Compression Engine. En second temps, nous proposons de généraliser un schéma existant de compression par extraction de motifs pour le cas des graphes dynamiques tout en l'hybridant avec une méthode $k^2$-Trees afin de tirer profit de ses performances. Cette proposition est motivé par le caractère évolutif de la majorité des situation modélisées par les graphes.\\
 
 Nous nous concentrerons  à la fin de ce travail sur l'évaluation des méthodes des deux moteurs en utilisant des benchmarks de graphes réels de domaines hétérogènes. Pour mener à bien cette dernière étapes, nous utiliserons deux métriques : le taux de compression et le nombre de bits par liens (bpe pour bits per edge).\\\\\textbf{Mots Clés :} \textit{Compression de graphes, Big Data, Extraction de motifs, K2-trees, Graphe du Web.}
 
\newpage
\thispagestyle{plain}
\begin{center}
	\par
	\textbf{
		\vskip 0.5in
		\LARGE 
			Abstract \\[0.15in]
	}
\end{center}
	\par
    
    We live in a world where the amount of data is constantly increasing and whose good management involves the use of graphs that have spread in different fields from social and communication networks to the fields of chemistry and biology. This abundance of generated data calls for a technique that is as old as the discipline of data processing but which is facing new challenges today: compression. Graph compression is a field in which the initial graph undergoes transformations in order to obtain a smaller and more compact version allowing, in the majority of cases, to perform the processings in a much better time.\\

 This dissertation deals with two classes of graph compression methods: the k2-trees methods and the methods based on pattern extraction. In the first place, we propose to extend and enrich a project started in a previous work. In order to accomplish this, we propose two compression engines: $k^2$-GraCE for $k^2$-trees Graph Compression Engine and P-GraCE for Pattern Graph Compression Engine. Secondly, we propose to generalize an existing pattern extraction compression scheme for the case of dynamic graphs while hybridizing it with a $k^2$-Trees method in order to take advantage of its performance. This proposition is motivated by the evolutionary nature of the majority of situations modeled by graphs.\\
 
 We will focus at the end of this work on evaluating the methods of the two engines using benchmarks of real graphs from heterogeneous domains. To objectively carry out this last step , we will use two metrics: the compression ratio and the number of bits per link (bpe). \\\\\textbf{Key words :} \textit{Graph compression, Big Data, Pattern extraction, K2-trees, Web graph.}

\newpage
 
 %%% Manque resume en arabe 

