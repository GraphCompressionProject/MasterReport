\documentclass[a4paper,oneside,12pt]{report}

%------------ package pour langue fr ------
\usepackage[utf8]{inputenc}
\usepackage[french]{babel}
\usepackage[T1]{fontenc}
\usepackage{multicol}
\usepackage{enumitem}
\usepackage{multirow}
%------------- for embedding images----------
\usepackage{graphicx} 
\usepackage{float}
\usepackage[export]{adjustbox}
\usepackage{amsfonts,epsfig,epstopdf,titling,url,array}
\usepackage{lscape}

\usepackage{amsmath}
\usepackage{amssymb}
\usepackage{amsthm}
%--------- pour le style de la page ----------
\usepackage[top=2.5cm, bottom=2.5cm, left=2.5cm, right=2.5cm]{geometry}

\usepackage[]{geometry}


\usepackage[french,ruled,vlined]{algorithm2e}
\usepackage[noend]{algorithmic}

%%% francisation des algorithmes 
\renewcommand{\algorithmicrequire} {\textbf{\textsc{Entrées:}}} \renewcommand{\algorithmicensure} {\textbf{\textsc{Sorties:}}} \renewcommand{\algorithmicwhile} {\textbf{tantque}} 
\renewcommand{\algorithmicdo} {\textbf{faire}} 
\renewcommand{\algorithmicendwhile}{\textbf{fin tantque}} \renewcommand{\algorithmicend} {\textbf{fin}} 
\renewcommand{\algorithmicif} {\textbf{si}} 
\renewcommand{\algorithmicendif} {\textbf{finsi}} 
\renewcommand{\algorithmicelse} {\textbf{sinon}} 
\renewcommand{\algorithmicthen} {\textbf{alors}} 
\renewcommand{\algorithmicfor} {\textbf{pour}} 
\renewcommand{\algorithmicforall} {\textbf{pour tout}}
\renewcommand{\algorithmicdo} {\textbf{faire}} 
\renewcommand{\algorithmicendfor} {\textbf{fin pour}} \renewcommand{\algorithmicloop} {\textbf{boucler}} \renewcommand{\algorithmicendloop} {\textbf{fin boucle}} \renewcommand{\algorithmicrepeat} {\textbf{répéter}} \renewcommand{\algorithmicuntil} {\textbf{jusqu'à}}
\renewcommand{\algorithmicreturn} {\textbf{Retourner}}
\renewcommand{\algorithmiccomment}[1]{\hfill$\blacktriangleright$ #1}


\usepackage{setspace}
\setstretch{1,15}
\usepackage{txfonts} %//pour utiliser times new roman dans le document
\usepackage{fancyhdr}
\fancyhf{} % clear all header and footers
\renewcommand{\headrulewidth}{0pt} % remove the header rule
\fancyfoot[RE,RO]{\thepage} % Left side on Even pages; Right side on Odd pages
\pagestyle{fancy}
\fancypagestyle{plain}{%
  \fancyhf{}%
  \renewcommand{\headrulewidth}{0pt}%
  \fancyhf[lef,rof]{\thepage}%
}
\renewcommand{\headrulewidth}{0.4pt}
\lhead{\textbf{Chapitre \thechapter}}
\fancyhead[R]{\rightmark}
%--------------------------- Sommaire ----------------------------%
\usepackage{hyperref}
\usepackage{amssymb}
\usepackage{natbib}
\theoremstyle{definition}
		\newtheorem{defn}{Definition}[section]
		\newtheorem{conj}{Conjecture}[section]
		\newtheorem{exmp}{Example}[section]
\usepackage[table,dvipsnames]{xcolor}	

\usepackage{pifont}% http://ctan.org/pkg/pifont
\newcommand{\xmark}{\color{red}\ding{55}}%
\newcommand{\cmark}{\color{PineGreen}\ding{51}}	
\usepackage[T1]{fontenc}
\newcolumntype{R}[1]{>{\raggedleft\arraybackslash }b{#1}}
\newcolumntype{L}[1]{>{\raggedright\arraybackslash }b{#1}}
\newcolumntype{C}[1]{>{\centering\arraybackslash }b{#1}}
%\usepackage{arabtex}
\usepackage[most]{tcolorbox}


\tcbset{
    frame code={}
    center title,
    left=0pt,
    right=0pt,
    top=0pt,
    bottom=0pt,
    colback=gray!70,
    colframe=white,
    width=\dimexpr\textwidth\relax,
    enlarge left by=0mm,
    boxsep=5pt,
    arc=0pt,outer arc=0pt,
    }
% Load the package with the acronym option
\usepackage[acronym]{glossaries}
 \usepackage{appendix}
%\usepackage{setspace}

 
 
 
 
% Generate the glossary
\makeglossaries

\begin{document}

%Page de garde (page de titre)							Obligatoire
\pagenumbering{Roman}
\begin{titlepage}

\newgeometry{top=0mm,right=20mm,left=20mm,bottom=0mm}



	%--------------------  Entete de l'Ecole ----------------------%

	\begin{figure}[t]
		\includegraphics[scale=0.75]{./ressources/image/ESI.png}\\[0.6in]
	\end{figure}
	
	
	
	%--------------------------------------------------------------%
	\begin{center}
	
	%------------------------  Le sujet ---------------------------%
		\LARGE \textbf{ Mémoire}\\
		\Large{
			Pour Obtention du diplôme d'ingénieur En Informatique\\
			\textbf{Option : Système Informatique (SIQ)}
			%\textsc thèse D'Ingéniorat En Informatique sous le thème :
		}\\[0.2in]
		\huge {
		\rule{\linewidth}{.5pt}
			\textbf{
				Étude et implémentation des méthodes de compression de graphe par extraction de motifs et k2-trees
			} 
			\rule{\linewidth}{.5pt}
		}\\[0.5in]
		\Large
	%--------------------------------------------------------------%	
	
	%-------------------------  Mon nom ---------------------------%
	\textbf{Réalisé par:}\\
	\begin{multicols}{2}
			\Large 	Mlle. Hafsa Bousbiat\\
			\large eh\_bousbiat@esi.dz\\
			ESI\\
		\columnbreak
 			\Large Mlle. Sana Ihadadene\\
			\large es\_ihadadene@esi.dz\\
			ESI \\
		
	\end{multicols} 
	
	\vskip 0.1in
	%--------------------------------------------------------------%	

	%-----------------------  Encadreures -------------------------%
	 \textbf{Encadré par:}\\
	 
	 \begin{multicols}{2}
			\Large 	Dr. Karima Amrouche\\
			\large k\_amrouche@esi.dz\\
			ESI\\
		\columnbreak
 			\Large Dr. Hamida Seba\\
			\large hamida.seba@univ-lyon1.fr\\
			Université de Lyon \\
	\end{multicols}
	
	%--------------------------------------------------------------%	
	
	\small
	\vskip 0.3in
	Octobre 2018 \\
	Année Universitaire: 2018-2019\\
	
	\end{center}		
\restoregeometry
\end{titlepage}

%Remerciements											Obligatoire
\thispagestyle{plain}
\begin{center}
	\par
	\textit{
		\vskip 1in
		\Huge 
			Remerciement \\[0.5in]
			\addcontentsline{toc}{chapter}{\numberline{}Remerciement}
	}
\end{center}
	\par
  Nous tenons d'abord à exprimer notre gratitude à l'égard de nos deux encadrantes, Madame Seba Hamida, Madame Amrouche Karima pour leur précieuse aide, leurs judicieux conseils et pour le temps qu'elles nous ont consacré tout au long de ce travail.\\


Nous remercions également toute l'équipe pédagogique de l'École Nationale
Supérieure d'Informatique, et à leur tète nos enseignants, pour la richesse et la qualité de la formation qu'ils nous ont offerte tout au long de notre cursus.\\



Nous tenons aussi à remercier les membres du jury qui nous ont fait l'honneur d'accepter de juger cet humble travail.\\


Enfin, nous adressons nos plus sincères remerciements à tous nos proches et amis, qui nous ont toujours encouragées au cours de la réalisation de ce mémoire.\\

\textit{Merci à tous et à toutes} ...

\newpage
 


%Résumé												Obligatoire

\thispagestyle{plain}
\begin{center}
	\par
	\textbf{
		\vskip 0.5in
		\LARGE 
			Résumé \\[0.15in]
			\addcontentsline{toc}{chapter}{\numberline{}Résumé}
	}
\end{center}
	\par
	
	Nous vivons dans un monde où la quantité d'informations ne cesse d'augmenter et dont la bonne gestion implique l'utilisation des graphes qui se sont répandus dans différents domaines allant des réseaux sociaux et de communication jusqu'aux domaines de la chimie et de la biologie. Cette abondance de données générées fait appel à une technique aussi vieille que la discipline de traitement de données mais qui connait de nouveaux défis aujourd'hui : la compression. La compression de graphes est un domaine dans lequel le graphe initial subit des transformations pour en obtenir une version plus réduite et compacte permettant, dans la majorité des cas, d'effectuer les traitements dans un temps nettement meilleur.\\
	
Deux classes de méthodes de compression feront l'objet de notre étude: les méthodes de compression par extraction de motifs et les méthodes basées sur les k2-trees. De ce fait, nous proposons deux moteurs de compression chacun englobant une ou plusieurs méthodes de chaque classe. Le premier moteur permet de compresser un graphe à travers ses structures les plus denses. Tant dis que le deuxième moteur exploite les propriétés de la matrice d'adjacence pour obtenir une représentation compacte. Nous proposons aussi une méthode de compression destinée aux graphes dynamiques et qui se situe à l'intersection des deux classes étudiées. En effet, elle permet de compresser le graphe à travers ces structures les plus denses tout en gardant trace de l'erreur, généralement représentée sous forme de matrices creuses, dans une structure k2-trees.\\

Nous conclurons par une étude comparative des performances des différents algorithmes de compression existants et la méthode que nous proposons où nous nous sommes basées sur des métriques d'évaluation tels que : le taux de compression et le gain obtenu. Afin de pouvoir réaliser cette étude en toute objectivité, les méthodes de compression seront testées sur des benchmarks connus de graphes issus de divers domaines.\\\\\textbf{Mots Clés :} \textit{Compression de graphes, Big Data, Extraction de motifs, K2-trees, Graphe du Web.}
 
\newpage
\thispagestyle{plain}
\begin{center}
	\par
	\textbf{
		\vskip 0.5in
		\LARGE 
			Abstract \\[0.15in]
	}
\end{center}
	\par
    
    We live in a world where the amount of data is constantly increasing and whose good management involves the use of graphs that have spread in different fields from social and communication networks to the fields of chemistry and biology. This abundance of generated data calls for a technique that is as old as the discipline of data processing but which is facing new challenges today: compression. Graph compression is a field in which the initial graph undergoes transformations in order to obtain a smaller and more compact version allowing, in the majority of cases, to perform the processings in a much better time.\\

 Two classes of compression methods will be the focus of our study: pattern extraction compression methods and k2-tree based methods. As a result, we propose two compression engines each encompassing one or more methods of each class. The first engine makes  compress a graph through its densest substructures. While, the second engine exploits the properties of the adjacency matrix to obtain a compact representation. We also propose a compression method for dynamic graphs at the intersection of the two classes studied. Indeed, it allows to compress the graph through his most important substructures, while keeping track of the error, usually represented as sparse matrices, in a k2-trees structure.\\
 
 We will conclude with a comparative study of the performances of the different existing compression algorithms and the method we propose where we use based on evaluation the folowing metrics: the compression ratio, number of bits per vertex and processing time. In order to be able to carry out this study in all objectivity, the compression methods will be tested on known benchmarks of graphs coming from various domains. \\\\\textbf{Key words :} \textit{Graph compression, Big Data, Pattern extraction, K2-trees, Web graph.}

\newpage
 
 %%% Manque resume en arabe 



%Sommaire (Table des matières)							Obligatoire
\pagestyle{plain}
\tableofcontents
\newpage



%Liste des figures										Selon besoin
\listoffigures
\addcontentsline{toc}{chapter}{\numberline{}Liste des figures}
\cleardoublepage

%Liste des tableaux										Selon besoin

\listoftables
\addcontentsline{toc}{chapter}{\numberline{}Liste des tableaux}
%\cleardoublepage



\printglossaries



\cleardoublepage

%Introduction (début de la pagination)					bligatoire

\pagenumbering{arabic}

	\thispagestyle{plain}
		\Huge{ 
			\textbf {Introduction générale}} \\[0.5in]
			\addcontentsline{toc}{chapter}{\numberline{}Introduction générale}
			\normalsize
			Avec l'énorme quantité de données produites par les activités humaines de nos jours, le problème de données massives (Big data) est devenu un enjeu essentiel. Un des outils les plus efficaces pour structurer et manipuler ces données est l'utilisation des graphes. Les graphes sont des outils de modélisation utilisés dans beaucoup de domaines pour la représentation des données : réseaux sociaux et de communication (entités reliées entre elles par des liens physiques ou communautaires), chimie (relations entre les atomes), biologie (interactions entre protéines par exemple) et bien d'autres domaines.\\
			
	Face à cette infobésité, les algorithmes classiques de traitement et de gestion des donnes se montre incapable d'offrir des réponse dans un temps raisonnable. Plusieurs solution ont été pensées pour contrer ce volume de données. Une des solutions les plus anciennes mais qui connait de nouveaux défis de nos jours est \textit{la compression de données}.\\

	Le domaine de compression de données est une branche de la théorie de l'information qui s'intéresse à minimiser la taille des données à stocker, traiter et transmettre améliorant ainsi de façon directe les temps de traitement. Parallèlement à cela, nous trouvons la compression des graphes qui est un domaine dans lequel le graphe initial subit des transformations pour en obtenir une version plus réduite. Différentes techniques, basées sur différentes approches, permettent cette compression, avec ou sans perte d'information, et génèrent de nouveaux graphes sur lesquels il est beaucoup plus intéressant d'effectuer les différents traitements.\\ 

Cependant, deux types de méthodes de compression de graphes se sont distinguées parmi tout les autres types de méthodes : les méthodes de compression en utilisant les arbres k2-trees et les méthodes de compression par extraction de motifs. En effet, elles permettent de trouver dans la majorité des cas un bon compromis entre l'espace mémoire et les temps de traitement. Ces deux classe de méthodes feront l'objet de notre étude.\\
		
			Notre première contribution  portera sur la conception, l'implémentation et l'évaluation de 	deux moteurs de compression, 
		\newacronym{$k^2$-GraCE}{$k^2$-GraCE}{$k^2$-trees Graph Compression engine}	
		\newacronym{P-GraCE}{P-GraCE}{Pattern Graph Compression engine}			
			\gls{$k^2$-GraCE} 
			 et \gls{P-GraCE}, chacun englobant les méthodes relatif à une classe. Nous visons à travers cela à comparer entre leurs méthodes. Notre deuxième contribution consiste en la proposition d'une nouvelle méthode de compression pour les graphes dynamiques, s'intitulant 
			 \newacronym{ddsm}{DDSM}{Dynamic Dense Subgraph Mining}
			 \gls{ddsm}. \\
			
			
			 Nous avons hiérarchisé notre mémoire en trois grands chapitres. Le premier est une introduction au domaine de la théorie. Dans le second chapitre, nous introduisons les définitions de base du domaine de compression de données appliqué aux graphes ainsi que les différentes méthodes de compression existantes sous forme d'une classification que nous proposons. Par la suite, dans le chapitre trois, nous détaillons la conception de nos deux moteurs de compression de graphes et nous décrirons les bases théorique de notre méthode, puis nous donnons dans le chapitre quatre les détails de l'implémentation de nos moteurs. Dans le chapitre cinq nous présentons nos tests de performance et résultats .
	
	\cleardoublepage
\pagestyle{fancy}
\fancypagestyle{plain}{%
  \fancyhf{}%
  \renewcommand{\headrulewidth}{0pt}%
  \fancyhf[lef,rof]{\thepage}%
}
\renewcommand{\headrulewidth}{0.4pt}
\lhead{\textbf{Chapitre \thechapter}}
\fancyhead[R]{\rightmark}

\part{État de l'art}
%---->chapitre 01 :« cadrage du projet »
	\chapter{ Théorie des graphes}
	%%Ajouter une petite introduction pour ce chapitre
	  Pour faciliter la compréhension d'un problème, nous avons tendance à le dessiner ce qui nous amène parfois même à le résoudre. La théorie des graphes est fondée à l'origine sur ce principe. De nombreuses propriétés et méthodes ont été pensées ou trouvées à partir d'une représentation schématique pour être ensuite formalisées et prouvées.


La théorie des graphes est historiquement un domaine mathématique qui s'est développé  au sein d'autres disciplines comme la chimie, la biologie, la sociologie et l'industrie. Elle constitue aujourd'hui un corpus de connaissances très important et un instrument efficace pour résoudre une multitude de problèmes.


De manière générale, le graphe sert à représenter les structures, les connexions entre différents composants, les acheminements possibles pour un ensemble complexe composé d'un grand nombre de situations en exprimant les dépendances et les relations entre ses éléments,(e.g. réseau routier ou ferroviaire, réseau de communication, diagramme d'ordonnancement, ..). 


Dans ce chapitre, nous présenterons les notions et les concepts clés relatifs aux graphes qui serviront de base pour la suite de notre travail, à savoir : la définition d'un graphe, ses types et sa représentation structurelle. Nous clôturons le chapitre avec quelques domaines d'application des graphes.

	
	\section{Graphe non orienté}
			\subsection{Définitions et généralités}
		
Un graph non orienté G est la donnée d’un couple (V , E) où V = \{$ \textit{v}_{1} , \textit{v}_{2} ,..., \textit{v}_{n} $\} est un ensemble fini dont les éléments sont appelés sommets ou nœuds ( Vertices en anglais ) et E=\{ $\textit{e}_{1} ,  \textit{e}_{2} ,…, \textit{e}_{m} $  \} est un ensemble fini d'arêtes ( Edges en anglais ). Toute arête \textit{e} de E correspond à un couple non ordonné de sommets \{ $\textit{v}_{i} , \textit{v}_{j}$ \} $\in$ E $\subset$  $V \times V$ représentants ses extrémités \citep{muller} \citep{fages2014exploitation}.
\\Soient \textit{e} = ($\textit{v}_{i} , \textit{v}_{j}$) et \textit{e'}=($\textit{v}_{k} , \textit{v}_{l}$) deux arêtes de E, On dit que :
\begin{itemize}
\item $\textit{v}_{i}$ et $\textit{v}_{j}$ sont les extrémités de  \textit{e} et \textit{e} est incident en $\textit{v}_{i}$ et en  $\textit{v}_{j}$ \citep{hennecart2012elements}.
\item $\textit{v}_{i}$ et $\textit{v}_{j}$ sont voisins ou adjacents, car il y’a au moins une arête entre eux dans E \citep{IUTLyonInformatique}.
\item L'ensemble des sommets adjacents au sommet \textit{e} est appelé le voisinage de \textit{e} \citep{muller}. 
\item \textit{e} et \textit{e'} sont voisins si ils ont une extrémité commune , i.e. : $\textit{v}_{i}$ = $\textit{v}_{k}$ par exemple \citep{lopez2003cours}.
\item L’arête \textit{e} est une boucle si ses extrémités coïncident, i.e. : $\textit{v}_{i}$ = $\textit{v}_{j}$ \citep{IUTLyonInformatique}. 
\item L’arête \textit{e} est multiple si elle a plus d’une seule occurrence dans l'ensemble E.
\end{itemize}	
		 
		\subsection{Représentation graphique}
		Un graph non orienté G peut être représenter par un dessin sur un plan comme suit \citep{muller}:
		\begin{itemize}

\item Les nœuds de G : $\textit{v}_{i}$ $\in$ V sont représenter par des points distincts.
\item 	Les arêtes de G : \textit{e} = ($\textit{v}_{i}$,$\textit{v}_{i}$) $\in$ E sont représenter par des lignes pas forcement rectilignes qui relient les extrémités de chaque arête \textit{e}.
\end{itemize}

%% L'exemple de la représentation graphique 
\textbf{Exemple :}
 Soit g=(V1 , E1) un graphe non orienté tel que : V1=\{ 1,2,3,5 \} et E=\{ (1,2), (1,4), (2,2), (2,3) ,(2,5) ,(3,4) \}.
La représentation graphique de g est alors donnée par le schéma de la figure \ref{graphNonOriente}.
\\
\begin{figure}[H]
\begin{center}
\includegraphics[height=120 pt, width=130 pt]{./ressources/image/graphNonOriente.png} 
\end{center}
\caption{Exemple de représentation graphique d'un graphe non orienté}
\label{graphNonOriente}
\end{figure}

		\subsection{Propriété d'un graphe} %% a revoir 
		
			\textbf{Ordre d'un graphe:} On appel ordre d’un 					graphe le nombre de ses sommets.i.e. Card(V) \citep{DUT}.
			
			\textbf{Taille d'un graphe:} On appel taille d’un 				graphe le nombre de ses arêtes.i.e. Card(E) \citep{DUT}.
			
			\subsubsection{Degré d'un graphe:} %% esk le degre de graphe ou de neoud
			\textbf{Degré d'un sommet : } Le degré d’un sommet 					noté \textit{d}($\textit{v}_{i}$) est le nombre 					d'arêtes incidents a ce sommet, sachant qu’une boucle 			compte pour 2 \citep{muller} . Dans l'exemple de la figure \ref{graphNonOriente}, le 					degré du sommet (1) est : \textit{d}(1)=2.
			
			\textbf{Degré d'un graphe : }Le degré d’un graphe est 			le degré maximum de ses sommets. e.i. c’est 						max(\textit{d}($\textit{v}_{i}$)) \citep{muller}. Dans l’exemple de 				la figure \ref{graphNonOriente}, le degré du graphe est \textit{d}(2)=5.
			
			\subsubsection{Rayon et diamétre d'un graphe: }
			\textbf{Distance : }La distance entre deux sommets 					\textit{v} et \textit{u} est le plus petit nombre 					d’arêtes qu’on doit parcourir pour aller de 						\textit{v} a \textit{u} ou de \textit{u} a 							\textit{v} \citep{muller}. 
			
			\textbf{Diamètre d’un graphe :} C’est la plus grande 				distance entre deux sommets de ce graphe 							\citep{muller}. 
			
			\textbf{Rayon d’un graph : }C’est la plus pette 					distance entre deux sommets de ce graphe \citep{parlebas1972centralite}. 
			%%%% A revoire d'autre memoire sur Graph theory
		
	
			
	\section{Graphe orienté}	
		
		\subsection{Définitions et généralités}
		Un graphe orienté G est la donnée d'un couple (V , E) où
		V est un ensemble fini dont les éléments sont appelés les sommets de G et 
		E  $\subset$ V x V est un ensemble de couples ordonnés de sommets dits arcs ou arêtes \citep{muller}. G est appelé dans ce cas digraphe (directed graphe).\\
		 Pour tout arc e = ( $v_{i}$ , $v_{j}$) $\in$ E :
		 \begin{itemize}  
			\item $v_{i}$ est dit extrémité initiale ou origine de e et $v_{j}$ est l'extrémité finale de e \citep{muller}.
			
			\item $v_{i}$ est le prédécesseur de $v_{j}$ et $v_{j}$ est le successeur de $v_{i}$ \citep{IUTLyonInformatique}.
			
			\item les sommets $v_{i}$ , $v_{j}$ sont des sommets adjacents \citep{Pres}.
			
			\item e est dit sortant en $v_{i}$ et incident en $v_{j}$ \citep{Pres}.
			
			\item e est appelé boucle si $v_{i}$ = $v_{j}$, i.e l'extrémité initiale et finale représente le même sommet \citep{IUTLyonInformatique}.
			
		\end{itemize}
		 
		
		\subsection{Représentation graphique}
		
		
		Un graphe G = (V , E) peut être projeter sur le plan en représentant:
		\begin{itemize} 
		\item dans un premier temps les nœuds $v_{i}$ $\in$ V par des points disjoints du plan.
		\item et dans un second temps les arêtes e = ( $v_{i}$ , $v_{j}$) $\in$ E par des lignes orientées reliant par des flèches les deux extrémités de e. 
		\end{itemize}
		
		\textbf{Exemple:}
		
		Soit g = ($V_{1}$ , $E_{1}$) un digraphe tel que : $V_{1}$ = \{ 1,2,3,4 \} et  $E_{1}$ = \{(1,2),(1,3),(3,2),(3,4),(4,3)\}.
		
		
		Le représentation graphique de g est alors donnée par le schéma de la figure ci-dessous.
		
		
			\begin{figure}[h]
			\includegraphics[scale=0.15,center]{./ressources/image/RepDiGraphe.png}
			\caption[Exemple de représentation graphique d'un digraphe.]{Exemple de représentation graphique d'un digraphe.}
			\end{figure}
			
		
		\subsection{Quelques Propriétés:} %%% Arevoire 
		
		\begin{itemize}
			\item \textbf{Ordre d'un digraphe:}
			est le nombre de sommets n = Card(V) \citep{DUT}.
			
			\item \textbf{taille d'un digraphe:} est le nombre d’arcs m = Card(A) \citep{DUT}.
			
			\item \textbf{Degré dans un digraphe:}
			Le degré d'un sommet $v_{i}$ $\in$ V dans un digraphe G = (V , E) est donnée par la formule :
			\begin{center}
				d($v_{i}$) = $d^+(v_{i}$) + $d^-(v_{i}$)
			\end{center}
				où $d^+(v_{i})$ est le nombre d'arcs sortants au sommet $v_{i}$ et est appelé degré extérieure et $d^-(v_{i}$) représente le nombre d'arcs incidents et est appelé degré intérieur \citep{muller}.
			 
			 \item \textbf{Voisinage dans un digraphe:}
			 Le voisinage d'un sommet $v_{i}$ $\in$ V, noté V($v_{i}$), dans un digraphe G = (V , E) est:\begin{center}
			 V($v_{i}$) = succ($v_{i}$) $\bigcup$ pred($v_{i}$) 
			 \end{center}
			avec succ($v_{i}$) qui est l'ensemble des successeurs de $v_{i}$ et pred($v_{i}$) qui l'ensemble de ses prédécesseurs \citep{bac}, i.e le voisinage de $v_{i}$ est l'ensemble des sommets qui lui sont adjacents.
				 
		\end{itemize}
		
			
			
			
			
			
			 
			 
		
	\section{Notion de Connexité}
	
	 
 
Les graphes sont généralement exploitables à travers leur interrogation qui permet de fournir des réponses aux problèmes modélisés. L'une des informations les plus importantes dans un graphe est la notion de relations (directes ou indirectes) entre deux nœuds ou plus formellement \textit{la connexité} dans un graphe. Dans cette partie, nous allons définir les concepts relatifs à cette notion dans le cas d'un graphe non orienté (resp. orienté).
 \begin{itemize} [label = $\bullet$]
			 \item \textbf{Chaine (resp. Chemin):}
			est une liste de sommets S= $(v_{0},v_{1},v_{2},...,v_{k})$ tel qu'il existe une arête (resp. un arc) entre chaque couple de sommets successifs \citep{muller}.
			 
			 
			  \item \textbf{Cycle (resp. Circuit):} 
			 est une chaine (resp. chemin) dont le premier et le dernier sommet sont identiques \citep{DUT}.
			 
			 \item \textbf{Graphe connexe:}
			Un graphe non orienté (resp. orienté) est dit connexe (resp. fortement connexe) si pour toute paire de sommets ($v_{i}$, $v_{j}$), il existe une chaine (resp. chemin) S les reliant \citep{muller}.
		
		\end{itemize}
	
	
	
	\section{Graphe partiel et sous-graphe:}
    		
	La quantité de donnée disponible aujourd'hui et sa croissance de manière exponentiel ont favorisé la décomposition des graphes en des entités plus petites afin de garantir une facilité de compréhension et d'analyse dans le but d'extraire l'information la plus pertinente. Dans cette partie nous allons définir de manière plus formelle ce que ces entités sont ainsi que leurs types.
	
		
		
		
		\subsection{Définitions:}
		Soient G = (V , E), G' = (V' , E') et G'' = (V'' , E'') trois graphes.
		\begin{itemize}[label=$\circ$]
		
			\item Le graphe G' est appelé graphe partiel de G si : V' = V et E' $\subset$ E \citep{DUT}. En d'autres termes, un graphe partiel est obtenu en supprimant une ou plusieurs arêtes de G.
				

			\item Le graphe G'' est dit sous-graphe de G si: V'' $\subset$ V et 
			 E''$\subset$ E $\cap$ (V'' x V'') \citep{bac}, i.e un graphe partiel est obtenu en enlevant un ou plusieurs nœuds du graphe initial ainsi que les arêtes dont ils représentent l'une des deux extrémités.
			 
		\end{itemize}
		
		\subsection{Quelques Types de sous graphes:}
		
		\begin{itemize} [label = $\bullet$]
		
		
			\item \textbf{Une Clique :} est un sous graphe complet de G \citep{bac}.
			
			\item \textbf{Bipartie :} G' est un sous graphe bipartie si : V' = $V_{1} \cup V_{2}$ , tel que $V_{1}$ $\cap$ $V_{2}$ = $\phi$ , et E' = $V_{1}$ x $V_{2}$ \citep{bac}.
			
			\item \textbf{Étoile :}
			 est un cas particulier de sous graphe bipartie ou X est un ensemble contenant le sommet central uniquement et Y contient le reste des nœuds \citep{koutra2015summarizing} .
			 
			 \item \textbf{Chemin (resp. Chaine):}
			est une liste de sommets S= $(v_{0},v_{1},v_{2},...,v_{k})$ telle qu’il existe un arc (resp. une arête) entre chaque couple de sommets successifs.
			 
			 
			  \item \textbf{Cycle (resp. Circuit):} 
			 est un chemin (resp. chaine) dont le premier et le dernier sommet sont identiques \citep{DUT}.
			 
		\end{itemize}		
	
	\section{Quelques types de graphe}
		 %% classifier selon le type du graphe en entree

	Avec les avancées technologique au fil du temps, plusieurs types de graphes ont connus le jours. En effet, La complexité et la variété des problèmes scientifiques existants modélisés par ces derniers ont poussé les chercheurs à adapter leurs structure selon le problème auquel  ils font face. Durant cette section nous allons définir les principaux types existants.
	
		\begin{itemize}[label=$\circ$]
		
			\item \textbf{Graphe Complet:} Un graphe G = (V , E) est un graphe complet si tous les sommets $v_{i}$ $\in$ V sont adjacents \citep{Pres}. Il est souvent noté $K_{n}$ où n = card(V) \citep{DUT}.
				
			
			\item \textbf{Graphe étiqueté et graphe pondéré:}
			 Un graphe étiqueté G = (V , E , W) est un graphe, qui peut être orienté ou non orienté, dont chacune des arêtes $e_{i}$ $\in$ E est doté d'une étiquette $w_{i}$. Si de plus, $w_{i}$ est un nombre alors G est dit graphe pondéré (valué) \citep{DUT}.
		
			\item \textbf{Graphe simple et graphe multiple:}
			Un graphe G = (V , E) est dit simple si il ne contient pas de boucles et tout pair de sommet est reliée par au plus une arête. Dans le cas contraire, G est dit multiple \citep{IUTLyonInformatique}.
			
	
		\end{itemize}
			
	
	
    		
    	\section{Représentation Structurelle d'un graphe}	
		\input{./Chapitres/TheorieDesGraphes/structRep.tex}
	
	\section{Les domaines d'application}
		
	La diversité des domaines faisant appel à la modélisation par des graphes ne cesse d'augmenter, allant des réseaux sociaux aux réseaux électriques et réseaux biologiques et arrivant jusqu'aux World Wide Web. Dans cette partie nous allons décrire trois domaines d'application les plus répandus des graphes.
	
		\subsection{Graphes des réseaux sociaux:}
		Les réseaux sociaux représentent un lieu d'échange et de rencontre entre individus (entités) et dont l'utilisation est devenue de nos jours une nécessité.  
		Pour représenter les interactions entre ces individus, nous avons généralement besoin de faire recours aux graphes où les sommets sont des individus ou des entités et les interactions entre eux sont représenté par des liens. 
		Vue la diversité des interactions sociales, la modélisation de ces réseaux  nécessite différents types de graphes: graphes non orientés pour pour les réseaux sociaux avec des relations non
orientées, graphes orientés pour représenter des relations non symétriques
comme c'est la cas dans les réseaux de confiance, graphes pondérés pour les réseaux sociaux qui contiennent différents niveaux d'intensités dans les relations, ... etc \citep{lemmouchi2012etude}.
		
		\subsection{Graphes en Bioinformatique:}
		
		La bio-informatique est un domaine qui se trouve à l'intersection des deux grands domaines celui de l'informatique et celui de la biologie. Elle a pour but d'exploiter la puissance de calcule des équipements informatiques pour effectuer des traitements sur des données moléculaires massives \citep{pellegrini2004protein}.
		
		Elle est largement utilisée pour l’analyse des séquences d’ADN et des protéines à travers leurs modélisation sous forme de graphe. A titre d'exemple, les graphes non orientés multiples sont un outil modélisation des réseaux d’interaction protéine-protéine \citep{pellegrini2004protein}, 
		le but dans ce cas est donc l'étude du fonctionnement des protéines par rapport à d'autre.
		
		\subsection{Le Graphe du web:}
		 Le graphe du Web est un graphe orienté dont les sommets sont les pages du web et les arêtes modélise l'existence d'un lien hypertexte dans une page vers une autre \citep{brisaboa2009k}. Il représente l'un des graphes les plus volumineux: en juillet 2000 déja, on estimait qu’il contenait environ 2,1 milliards de sommets et 15 milliards d’arêtes avec 7,3 millions de pages ajoutées chaque jour \citep{guillaume2002web}. De ce fait, ce graphe a toujours attiré l'attention des chercheurs. En effet, l'étude de ses caractéristiques a donné naissance à plusieurs algorithmes intéressants, notamment l'algorithme PageRank de classement des pages web qui se trouve derrière le moteur de recherche le plus connu de nos jours : Google.
		 
		 
		 \subsection{Raster :}
		 
		
		 
		 
		 \subsection{Bases de donnés RDF}	
		
	
	
			
	\section{Conclusion}
Dans ce chapitre nous avons présenté les notions et les concepts généraux qui touchent à la théorie des graphes : définitions des graphes, leurs principales propriétés, leurs représentations ainsi que leurs domaines d'application.\\
Le point important qu'on a pu tirer de cette partie est que les graphes sont devenus un moyen crucial et indispensable dans la modélisation des problèmes dans plusieurs domaines. Cependant ils deviennent de plus en plus complexes et volumineux avec la grande quantité de données disponible de nos jours. De ce fait, leur stockage, visualisation et traitement sont devenus difficile. La compression de graphe est naît comme solution à ce problème. Dans le chapitre suivant nous allons présenter la compression de graphes, son rôle et ses différentes méthodes.  
	
	
	

%---->chapitre 02 :« »
	\chapter{Compression de graphes}
	La puissance des processeurs, de nos jours, augmente plus vite que les capacités de stockage ce qui engendre un déséquilibre entre le volume des données qu’il est possible de
traiter et de stocker. Dès lors, la réduction de la taille des données, plus formellement la compression de données, a été un domaine de recherche très active. 
		
		 Dans ce chapitre, nous allons
ouvrir un champ de réflexion en introduisant tout d’abord le domaine de compression des données et son application dans la théorie des graphes. Puis dans un second
temps, nous allons présenter une étude bibliographique sur les méthodes de compression existantes aujourd’hui et qui s'inscrivent dans l'une des deux classes de méthodes de compression: les méthodes de compression par les k2-trees et les méthodes de compression par extraction de motifs, pour finir avec une étude comparative entre elles.
	
		
		\section{Compression de données :}
		
			 La compression de données est principalement une branche de la théorie de l'information  qui traite des techniques et méthodes liées à la minimisation de la quantité de données à transmettre et à stocker.
Sa caractéristique de base est de convertir une chaîne de caractères vers un autre jeu de caractères occupant un espace mémoire le plus réduit possible tout en conservant le sens et la pertinence de l'information \citep{lelewer1987data}.

	Les techniques de compression de données sont principalement motivées par la nécessité d'améliorer l'efficacité du traitement de l'information. En effet, la compression des données en tant que moyen peut rendre l'utilisation des ressources existantes beaucoup plus efficace. 
	
	De ce fait, une large gamme d'applications usant du domaine de compression tel que le domaine des télécommunications et le domaine du multimédia est apparue offrant une panoplie d'algorithmes de compression \citep{sethi2014data}. Sans les techniques de compression, Internet, la télévision numérique, les communications mobiles et les communications vidéo, qui ne cessaient de croître, n'auraient été que des développements théoriques.
	
	%Afin de pouvoir comparer entre ces différentes méthodes de compression, plusieurs critères ont été proposés dans la littérature. Parmi les mesures utilisées on trouve:
	%\begin{itemize}
	%	\item \textbf{Le taux de compression:} représente le rapport entre la taille des données après et avant la compression.
		
		%%% donner la formule
		
		%\item \textbf{Le facteur de compression:} représente l'inverse du taux de compression.
		
		%\item \textbf{Le temps de compression:}
	%\end{itemize}
		
		\section{Compression appliquée aux graphes:}
			La compression des graphes a été proposée comme solution pour le traitement et le stockage des graphes volumineux. Elle permet de transformer un grand graphe en un autre plus petit tout en préservant ses propriétés générales et ses composants les plus importants.
Les principaux avantages de la compression sont  \citep{liu2018graph} :
\begin{itemize}

\item La réduction de la taille des données et de l'espace de stockage : De nos jours, les graphes représentant les bases de données, les réseaux sociaux et tous types de données numériques sont caractérisés par une croissance exponentielle de leurs volume, ce qui rend leur stockage difficile et couteux en terme d'espace mémoire. Les techniques de compression produisent des graphes plus petits qui nécessitent moins d'espace. Cela permet aussi de réduire le nombre d'opérations d'E/S et les communications entre nœuds dans un environnement distribué et de charger le graphe en mémoire centrale.   

\item L'exécution rapide des algorithmes de traitement et des requêtes sur les graphes : l'exécution des différents algorithmes de traitement sur des graphes volumineux peut s'avérer couteuse en terme de temps et peut ne pas donner les résultats attendus. La compression permet d'obtenir de petits graphes qui peuvent être traités, analysés et interrogés plus efficacement et dans un temps raisonnable. 
  
\item La Facilité d'analyse et de visualisation des graphes : Les techniques de compression permettent de représenter les données et les structures des graphes massives d'une manière plus significative permettant ainsi leur analyse et leur visualisation contrairement au graphes d'origines qui ne peuvent même pas être chargés en mémoire.  

\item L'élimination du bruit : les grands graphes du web sont considérablement bruités. Ce bruit peut perturber l'analyse en faussant les résultats et en augmentant la charge de travail liée au traitement des données. La compression permet donc de filtrer le bruit et de ne mettre en évidence que les données importantes.

\end{itemize}



			
	
			\subsection{Les types de compression:}
			La compression de graphes est définie comme l'ensemble des méthodes et techniques permettant de réduire l'espace mémoire occupé par ces derniers tout en gardant la même signification que le graphe d'origine. Dès lors, deux approches se présentent: la compression avec ou sans perte, que nous allons détailler dans ce qui suit.
			
			\begin{enumerate}[label=\alph*)]
			\item \subsubsection{Compression Sans Perte:}
			Certains domaines d'application de la compression nécessitent un niveau élevé d'exactitude et une restitution exacte, donc une compression sans perte. Dans cette catégorie, le graphe G subit des transformations pour avoir une représentation compacte $G'$ qui lors de la décompression donne exactement G. La figure ci-dessous illustre cette définition. 
			
			\begin{figure}[h]
			\includegraphics[scale=0.55,center]{./ressources/image/SansPerte.png}
			\caption[Compression sans perte.]{Compression sans perte.}
			\end{figure}
			
			%% citee qlq exemple de compression sans pertes
			
			\item \subsubsection{Compression Avec Perte:}
			Contrairement à la compression sans perte, la compression avec perte permet la suppression permanente de certaines informations jugées inutiles (redondantes) pour améliorer la qualité de la compression.  En d'autres termes, le graphe G subit des transformations pour avoir une représentation compacte $G'$ qui lors de la décompression donne un graphe $G''$ probablement différent de G mais l'approximant le plus possible. La figure ci-dessous illustre cette définition.   
			
			\begin{figure}[h]
			\includegraphics[scale=0.55,center]{./ressources/image/AvecPerte.png}
			\caption[Compression avec perte.]{Compression avec perte.}
			\end{figure}
			
			%%% citee qlq exemple de compression avec perte
				\end{enumerate}
			
			
			
			
	
			\subsection{Les métriques d'évaluation des algorithmes de compression:}
					
				Devant la panoplie d'algorithmes et de techniques de compression de graphe disponibles dans la littérature, des critères de comparaison  et d'évaluation entre ces méthodes doivent être bien définis. Dans cette partie, nous présenterons les principales mesures de performances.
				
				\begin{enumerate}[label=\alph*)]
					\item \subsubsection{Le temps de compression:}
				C'est une métrique qui donne le temps d'exécution de l'algorithme de compression. Elle est généralement mesurée en secondes (ou ms).
				
				%\newacronym{cr}{CR}{Ratio de compression}
				\item 	\subsubsection{Le ratio de compression:}
				Le ratio de compression (CR) est la mesure la plus courante pour calculer l'efficacité d'un algorithme de compression. Il est défini comme le rapport entre le nombre total de bits requis pour stocker les données non compressées et le nombre total de bits nécessaires pour stocker les données compressées.
				%la formule
				\begin{center}
				$
				CR = \frac{Nbre.\ de\ bits\ du\ graphe\ originale}{Nbre.\ de\ bits\ du\ graphe\ finale }
				$
				\end{center}
				
				
				Le CR est parfois appelé bit par bit (bpb) et il est défini alors comme étant le nombre moyen de bits requis pour stocker les données compressées \citep{uthayakumar2018survey}. Dans le cas des algorithmes de compression de graphe on a:
				\begin{itemize}
					\item \textbf{Le nombre de bits par nœud:}
					représente l'espace mémoire nécessaire pour stocker un nœud (bpn pour << \textit{bits per node} >> en Anglais).
					%% la formule
					
					\item \textbf{Le nombre de bits par lien:}
					représente l'espace mémoire nécessaire pour stocker un arc dans le cas d'un graphe orienté ou une arête dans le cas d'un graphe non orienté (bpe pour << \textit{bits per edge} >> en Anglais).
					%%la formule 
				\end{itemize}
				
				\item 	\subsubsection{Le taux de compression:}
				Exprimée en pourcentage, cette métrique permet de mesurer la performance de la méthode de compression. Elle peut être exprimée de deux manières différentes:
				
				\begin{itemize}
					\item \textbf{Le taux de compression:} Le rapport entre le volume du graphe après compression et le volume initial du graphe.
					\begin{center}
				$
				t = \frac{La\ taille\ du\ graphe\ finale}{La\ taille\ du\ graphe\ originale }
				$
				\end{center}
					\item \textbf{Le gain d'espace: }Le gain d'espace représente la réduction de la taille du graphe compressé par rapport à la taille du graphe original.
					
					\begin{center}
				$
				G = 1 - \frac{La\ taille\ du\ graphe\ finale}{La\ taille\ du\ graphe\ originale }
				$
				\end{center}
					
					
				\end{itemize}
				%\subsubsection{L'erreur quadratique:}
				\end{enumerate}
				
				
				
				
				
				
				
				
				
			
			
			\subsection{Classification des méthodes de compression:}
				Le domaine de compression des graphes est un domaine qui a connu une grande évolution vu son importance. Une multitude de méthodes ont été proposées au cours des dernières années. Elles diffèrent les unes des autres selon plusieurs points: le type de graphe en entrée, le type de structure en sortie, le type de compression et la technique utilisée pour la compression. En se basant sur ces différences, plusieurs classifications ont été suggérées. Nous allons dans ce qui suit présenter les plus importantes parmi ces classifications.\\

Dans \citep{maneth2015survey}, les auteurs proposent une classification basée tout d'abord sur le type de compression. Ils regroupent les méthodes en deux catégories principales : les méthodes de compression sans perte et les méthodes de compression avec perte. La première catégorie est subdivisée à son tour selon le type de représentation du graphe en sortie : représentation succinte, représentation structurelle ou une représentation sous forme de fichier RDF\newacronym{rdf}{RDF}{Resource Description Framework}\footnote{\gls{rdf} : est un modèle de graphe destiné à décrire de façon formelle les ressources Web et leurs métadonnées, de façon à permettre le traitement automatique de telles descriptions.}. 
Les méthodes donnant une représentation succincte représentent le graphe sous forme d'une chaine de bits succincte irréversible. La sortie de ces méthodes est ainsi une structure compacte du graphe original. Parmi les méthodes de cette classe, nous trouvons : Web Framework de Boldi et Vigna \citep{boldi2004webgraph}. La deuxième classe est la représentation structurelle. Contrairement à l'approche précédente, les méthodes de cette classe modifient la structure du graphe initial, sachant que les modifications apportées sont réversibles. La sortie sera donc une structure réduite et non pas compacte de la version initiale. Parmi ces méthodes, nous citons : RePair de Claude et Navarro \citep{claude2010fast}. La dernière classe est la compression  des fichiers \gls{rdf} et qui comporte des méthodes assez récentes. Nous trouvons parmi ces techniques : Dcomp de \citep{martinez2012compression} . Les méthodes de compression avec perte quant à elles apportent des modifications irréversibles sur le graphe en supprimant les informations redondantes et le bruit. Comme exemple, nous citons : ASSG de Zhang et al. \citep{zhang2014assg}.\\

Une autre classification a été exposée par Lui et al. dans \citep{liu2018graph} qui classe les méthodes sur trois niveaux. Au premier et deuxième niveaux, les techniques de compression sont regroupées en fonction du type de graphe en entrée selon deux critères : graphe statique ou dynamique et graphe simple ou étiqueté. Pour le troisième niveau, les auteurs catégorisent les méthodes selon la technique de traitement utilisée. Quatre catégories sont définies: les méthodes de regroupement ou d'agrégation, ces méthodes permettent d'agréger de manière récursive un ensemble de nœuds, liens ou carrément un cluster en un super nœud (appelé parfois nœud virtuel), comme exemple de ces techniques, nous trouvons Grass \citep{lefevre2010grass}. Le deuxième type de méthodes englobe les méthodes de compression de bits. Ces méthodes minimisent le nombre de bits nécessaires au stockage du graphe en se basant sur 
\newacronym{mdl}{MDL}{Minimum Description Length} 
le principe de description minimal (\gls{mdl} en Anglais). Elles peuvent être avec ou sans perte. Parmi elles, nous citons LSH-based \citep{khan2014set}. La troisième classe comporte les méthodes de simplification qui suppriment les arêtes les moins importantes selon un certain critère. Parmi ces méthodes, nous trouvons celle proposée dans pqr Shen et al. \citep{shen2006visual}. La dernière catégorie est la classe des méthodes basées sur l'influence, les méthodes de cette catégorie décrivent le graphe par les flux d'influence les plus importants ce qui permet de l'analyser plus facilement. Ces méthodes permettent de formuler le problème de compression comme un processus d'optimisation dans lequel la quantité de données liée à l'influence est maintenue en sortie. Parmi ces techniques, nous mentionnons \citep{shi2015vegas}.\\


La dernière classification que nous allons présenter est la classification proposée dans un master de l'année dernière par le binôme : Mlle. Belhocine et Mr. Guermah \citep{master2017}. Cette classification se base sur le principe utilisé dans le processus de compression. Elle regroupe six classes de méthodes : 1) compression basée sur l'ordre des nœuds en exploitant le principe de similarité et de localité du graphe, les méthodes de cette catégorie cherchent à trouver un ordre des nœuds, qui doit répondre à deux propriétés essentielles : la similarité \footnote{ Deux nœuds proches ont tendance à avoir des voisins similaires.} et la localité \footnote{ Les liens sortants d'un nœud ont tendance à se diriger vers un ensemble
de nœuds qui sont proches.}, cet ordre est ensuite utilisé dans la construction d'une structure de données qui compresse le graphe en entrée, comme exemple de ces méthodes, nous trouvons Layered Label Propagation \citep{boldi2011layered} qui utilise l'ordre LLP \footnote{LLP est un algorithme itératif qui produit une séquence d’ordres de nœuds en se basant sur les étiquettes affectées aux clusters après avoir partitionner le graphe initial.} et Recursive Graph Bisection de \citep{dhulipala2016compressing} qui utilise un ordre BP \footnote{BP est un ordre basé sur le problème de bissection de graphes, qui cherche à trouver une meilleure partition des nœuds du graphe tout en minimisant une fonction objectif.}. 2) compression basée sur l'ordre des nœuds en exploitant la linéarisation du graphe, les méthodes de cette classe se basent sur une nouvelle structure de données intitulée structure de données eulérienne, elles sont conçues principalement pour les grands graphes, parmi ces méthodes nous trouvons Neighbor Query Friendly Compression 
\citep{maserrat2012community}. 3) compression basée sur l'étiquetage des nœuds par des intervalles, ces méthodes visent à construire une structure d'index à partir du graphe initial qui permet de répondre aux requêtes de voisinage, parmi ces méthodes nous citons DAGGER \citep{yildirim2012grail}. 4) compression basée sur la structure d'arbre ${K}^{2}$, les méthodes de cette catégorie s'appuient sur la représentation ${K}^{2}$-trees pour compresser le graphe, ces méthodes font l'objet de notre travail et seront détaillées par la suit. 5) compression basée sur l'agrégation des nœuds, ces méthodes sont parmi les méthodes les plus populaires dans le domaine de compression, elles cherchent à compresser le graphe initial en agrégeant un certain nombre de nœuds en un seul nœud appelé super-nœud, cette agrégation se fait selon différentes manières, elle peut être orientée par une fonction objectif représentant l’espace de stockage optimal généralement établie par le principe de la longueur
de description minimale MDL ou par la similarité des caractéristiques des nœuds tels que les
labels et les attributs ou par l’extraction de motifs en utilisant des techniques de pattern mining, cette dernière représente l'une des matières de notre recherche. 6) compression basée sur l'agrégation des liens, contrairement aux techniques de la classe précédente, les méthodes de cette classe visent à compresser le graphe initial en fusionnant certains ensembles de ses liens en un seul liens appelé super-lien, elle est établie selon deux façons, elle est orientée soit par le biais des règles de grammaire, ou bien par l'extraction de motifs que nous allons étudier par la suit. la figure \ref{xx} représente le schéma de cette classification. \\
 \begin{figure}[H]
		\includegraphics[scale=0.6]{./ressources/image/xx.jpg}
		\caption{Classification des méthodes de compression \citep{master2017}}
		\label{xx}
	\end{figure}


%Après l'étude des méthodes basées sur l'agrégation par extraction de motifs, nous avons constaté que certaines classes ne sont pas bien définies. Les imperfections que nous avons remarqué peuvent être énumérées comme suit : 1) les méthodes d'extraction de motifs englobent certaines méthodes d'agrégation et non pas le contraire, 2) Les méthodes d'extraction de motifs ne sont pas toutes des méthodes agrégatives, nous trouvons parmi elles d'autres méthodes basées sur un vocabulaire. Nous avons donc apporté des modifications sur cette classification. La figure \ref{classif} représente la classification de l'année passé après raffinement où nous avons marqué nos apports en rouge.

Après l'étude des méthodes basées sur l'agrégation par extraction de motifs, nous avons constaté que certaines classes ne sont pas bien définies. Nous avons remarqué que la classification a proposé d'inclure les méthodes basées sur l'extraction de motifs dans deux classes de méthodes qui sont : les méthodes basée sur l'agrégation des liens et les méthodes  basée sur l'agrégation des nœuds, or ce sont les méthodes d'extraction de motifs qui englobent certaines méthodes d'agrégation et non pas le contraire. En effet, notre étude bibliographique a clairement montré que les méthodes de compression de graphe par extraction de motifs ne sont pas toutes des méthodes agrégatives. Nous trouvons, par exemple, certaines de ces méthodes qui donnent en sortie une liste des motifs les plus pertinents du graphe avec une matrice d'erreurs sans avoir recours à l'agrégation. Donc nous avons proposé de dissocier la classe des méthodes basées sur l'extraction de motifs de la classe des méthodes basées sur l'agrégation.
En outre, nous avons considéré, contrairement à la classification précédente, que les méthodes de compression basées sur les règles de grammaire font parties des méthodes de compression par extraction de motifs. Nous justifiant cela par le fait que ces méthodes utilisent l'un des deux principes suivants : 
\begin{enumerate}
\item La transformation de la liste d'adjacence en une chaine de caractères et le remplacement, de manière itérative, de la sous-chaine de longueur deux la plus fréquente par une règle de production et qui représente donc dans ce cas un motif.
\item La recherche d'un motif, de type \textit{digram}\footnote{Un digramme : est sous-graphe composé de deux arêtes ayant un sommet en commun.}, satisfaisant une certaine condition et le remplacement de toutes ses occurrences par une production dans la grammaire. 
\end{enumerate}
Le fait que les motifs dans le cas de cette classe n'ont pas une structure de sous-graphes connus (clique, étoile, ...) n'empêche pas que le processus appliqué s'intègrent toujours dans la discipline d'extraction de motifs. Nous avons essayé donc de rectifier ces imperfections tout en raffinant davantage les deux classes qui font l'objet de notre Master. 

La classe des méthodes de compression par les arbres k2-trees ne peut être encore raffinée car elles partagent toutes le même principe et diffèrent dans le type de graphe en entrée ou dans le codage en sortie.

Pour la classe des méthodes de compression par extraction de motifs, nous distinguons cinq classes: 1)les méthodes de compression basées vocabulaire faisant appel aux méthodes de clustering, 2) les méthodes de compression basées vocabulaire exploitant les propriétés de la matrice d'adjacence, 3) les méthodes de compression basées agrégation des nœuds des motifs, 4) les méthodes de compression basées agrégation des liens des motifs utilisant des règles de grammaire, 5) les méthodes de compression basées agrégation des liens des motifs faisant appel à des heuristiques de partitionnement de graphes. La première et la deuxième classe se caractérisent par l'ensemble des motifs qui est prédéfini au départ. Cependant, elles se distinguent l'une de l'autre dans le principe de fonctionnement : l'une utilise des méthodes de clustering et de détection de communautés tant dis que l'autre utilise directement la matrice d'adjacence en exploitant ses propriétés. Les trois (03) dernières classes forment l'ensemble des méthodes d'extraction de motifs basées sur l'agrégation de ces derniers. 




 La figure \ref{classif} représente la classification de l'an dernier après raffinement où nous avons mis en évidence nos apports et nos modifications en pointillé.


 \begin{figure}[H]
		\includegraphics[scale=0.6]{./ressources/image/Classification.png}
		\caption{Classification proposées des méthodes de compression}
		\label{classif}
	\end{figure}






				
			\section{Compression par les arbres $K^2$-Trees}
				$k^2$-tree est une structure de données dense conçue à l'origine pour la compression des graphes du web. L'algorithme de base a été proposé par Bisaboa et al. dans leur article \textit{$k^2$-trees for Compact Web Graph} \citep{brisaboa2009k}. Elle a été appliquée ensuite dans d'autres travaux de compression comme les réseaux sociaux \citep{shi2012optimizing}, les données rasters \citep{de2013compact} et les bases de données \gls{rdf} \citep{alvarez2017succinct}.\\

En général, l'algorithme peut être appliqué à n'importe quelle matrice binaire. Dans le cadre de notre étude nous nous intéressons seulement à la matrice d'adjacence d'un graphe.
La compression par les $k^2$-trees exploite les propriétés de la matrice d'adjacence et tire parti des zones vides pour réduire l'espace de stockage et permettre au graphe de tenir en mémoire centrale. Il offre aussi la possibilité de naviguer dans le graphe sans le décompresser, et de répondre aux requêtes de voisinage direct et inverse.

Étant donné une matrice d'adjacence A d'ordre $|n|$, $k^2$-tree la représente sous forme d'un arbre de recherche $k^2$-aires \footnote{Les arbres n-aires sont une généralisation des arbres binaires : chaque nœud a au plus n fils.} de hauteur h = [$log_{k}$ n]. Chaque nœud de l'arbre contient un seul bit avec deux valeurs possibles : 1 pour les nœuds internes et 0 pour les feuilles, sauf le dernier niveau où les feuilles représentent les cases de la matrice A et peuvent prendre une valeur 0 ou 1. Chaque nœud interne de l'arbre a exactement $k^2$ fils.  
Avant la construction de l'arbre, il faut s'assurer que n est une puissance de k. Dans le cas contraire, l'algorithme étend la matrice en rajoutant des zéros à droite et en bas de la matrice. L'ordre de la matrice devient donc $n'= k^{log_{k} n}$.\\

Pour construire l'arbre, $k^2$-tree commence par diviser la matrice en $k^{2}$ sous matrices d'ordre $|n/k|$. La racine correspond à la matrice complète. Chaque sous matrice représente un nœud dans le premier niveau de l'arbre, elle est ajoutée comme un fils à la racine suivant un ordre de gauche à droite et de haut en bas. Le nœud est à 1 si la sous matrice qu'il représente contient au moins un 1, et à 0 si elle ne contient que des 0. Le processus est répété de manière récursive sur les sous matrices représentées par des 1. $k^{2}$ sous matrices sont considérées à chaque subdivision. L'opération est répétée jusqu'à ce que la subdivision atteint les cases de la matrice qui représenterons les feuilles de l'arbre au dernier niveau. 

La figure \ref{k2-trees-exemples} illustre la représentation $k^2$-tree d'une matrice de taille $10 \times 10$, étendue à une  taille $16 \times 16$ pour un k=2 \citep{brisaboa2015efficient}.

\begin{figure}[H]
\begin{center}
\includegraphics[height=200 pt, width=450 pt]{./ressources/image/k2-trees.png} 
\end{center}
\caption{Exemple de représentation $k^2$-tree}
\label{k2-trees-exemples}
\end{figure}


Pour le stockage de l'arbre, l'algorithme utilise deux tableaux binaires : un tableau T (Tree) contenant tous les nœuds de l'arbre à l'exception du dernier niveau et un tableau L (Leaves) contenant les feuilles du dernier niveau. Les nœuds et les feuilles sont ordonnés selon un parcours en largeur de l'arbre.   
Ci-dessous les deux tableaux T et L de l'exemple précédent (figure \ref{k2-trees-exemples}) : \\
\begin{center}
	T = 1011 1101 0100 1000 1100 1000 0001 0101 1110\\
	L = 0100 0011 0010 0010 1010 1000 0110 0010 0100\\	
\end{center}
Dans le pire des cas, l'espace totale pour la description de la structure est $k^2*m(log_{k^2}\frac{n^2}{m}+ \textit{O}(1))$, où n est le nombre de nœuds du graphe et m le nombre de liens. Cependant, pour les graphes réels, l'espace nécessaire pour le stockage est bien meilleur. \\

Dans le même article \citep{brisaboa2009k}, et dans le but d'obtenir un compromis entre la taille de l'arbre et le temps de parcours, les auteurs proposent une hybridation qui consiste à changer la valeur du paramètre k en fonction du niveau de l'arbre en donnant à k une grande valeur au début pour réduire le nombre de niveaux et améliorer ainsi le temps de recherche, et une petite valeur à la fin pour avoir des petites sous matrices et réduire l'espace de stockage.
Pour le stockage de l'arbre, un tableau $T_{i}$ est utilisé pour chaque valeur $k_{i}$, le tableau L reste le même.\\

Plusieurs variantes de l'algorithme de base ont été proposées dans la littérature dont le but était soit d'obtenir un meilleur résultat de compression, soit d'appliquer la méthode sur d'autres types de graphes. Nous allons dans ce qui suit présenter les principaux travaux qui traitent de ce sujet.\\

Dans \citep{shi2012optimizing}, les auteurs proposent deux techniques d'optimisation de l'algorithme : la première consiste à trouver un certain ordre des nœuds qui permet de regrouper les 1 de la matrice d'adjacence dans une seule sous matrice au lieu qu'ils soient dispersés de manière aléatoire. La recherche d'un ordre optimal des nœuds n'est envisageable, avec k=2, car le problème peut être réduit à un autre problème (min bisection\footnote{Le problème de bissection minimale (MBP) est un problème NP-hard bien connu, qui est destiné à diviser les sommets d'un graphe en deux moitiés égales afin de minimiser le nombre de ces arêtes avec exactement une extrémité dans chaque moitié.}) qui est NP-difficile. Les auteurs utilisent alors un parcourt \newacronym{dfs}{DFS}{Deapth First Search}\gls{dfs}
\footnote{\gls{dfs} : Parcours en profondeurs du graphe}
 avec des heuristiques pour trouver une approximation de l'ordre optimal. Cette optimisation permet de réduire le nombre de nœuds internes et produit ainsi un arbre optimale. La deuxième optimisation est de trouver la valeur de k la plus adéquate pour chaque nœud interne, calculer cette valeur pour chaque nœud peut engendrer un temps de calcul très important. Pour éviter cela, les auteurs affectent la même valeur k pour les nœuds ayant le même parent. 	

Dans un travail ultérieur \citep{brisaboa2014compact}, les auteurs de l'article de base apportent deux améliorations principales de leur méthode dans le but d'optimiser l'espace et le temps de parcours de l'arbre produit. La première est de construire $k^2$ arbres distincts pour les $k_{0}^{2}$ sous matrices du premier niveau. Elle présente plusieurs avantages : (1) l'espace est réduit étant donné que la taille de chaque arbre est en fonction de $\frac{n^2}{k^2}$, (2) le temps de parcours s'améliore puisque T et L sont plus petits. La deuxième amélioration est la compression de L qui consiste à construire un vocabulaire \textit{V} de toutes les sous matrices du dernier niveau sous forme de séquences de bits, les classer par fréquence d'apparition et remplacer leurs occurrences dans L par des pointeurs. Cela permet d'éviter la redondance et de réduire la taille de la structure. Les pointeurs sont représentés par des codes de longueur variable ordonnés, le plus petit correspond à la sous matrice la plus fréquente. Néanmoins, cette représentation ne permet pas un accès direct dans L étant donné qu'une décompression séquentielle est nécessaire pour récupérer une position. Pour remédier à ce problème, les auteurs utilisent le principe de
\newacronym{dac}{DAC}{Directly Addressable Codes}
 \gls{dac}\footnote{\gls{dac} est une technique qui encode une séquence d'entiers ou mots en utilisant une structure à longueur variable, son avantage principale est l'accès direct au mot sans passer par le décodage.} \citep{brisaboa2013dacs} pour garantir un accès rapide au pointeur et conserver ainsi une navigation efficace.\\
Exemple : Pour la figure \ref{k2-trees-exemples}, le vocabulaire et L sont représentés comme suit :\\
\begin{center}
\textit{V}= [0010 0100 0011 1010 1000 0110]\\
L = $c_{1}c_{2}c_{0}c_{0}c_{3}c_{4}c_{5}c_{0}c_{1}$ \\
\end{center}

%\subsubsection{d$k^2$-trees}
Dans \citep{brisaboa2012compressed}, les même auteurs développent la représentation $k^2$-trees pour les graphes dynamiques. Ils proposent une nouvelle structure nommée d$k^2$-trees pour \textit{Dynamique $k^2$-trees} qui offre les même capacités de compression et fonctionnalités de navigation que le cas statique et qui permet également d'avoir des mises à jour sur le graphe. Pour atteindre ces objectifs, d$k^2$-tree remplace la structure statique de $k^2$-tree par une implémentation dynamique. Dans cette nouvelle implémentation, les deux tableaux T et L sont remplacés par deux arbres, nommés $T_{tree}$ et $L_{tree}$ respectivement. Les feuilles de $T_{tree}$ et $L_{tree}$ stockent des parties des bitmaps T et L. La taille de ces feuilles est une valeur paramétrable. Les nœuds internes des deux arbres permettent d'accéder aux feuilles et de les modifier.
Chaque nœud interne de $T_{tree}$ contient trois(03) éléments : deux compteurs b et o, qui contiennent respectivement le nombre de bits et le nombre de "1" stockés dans les feuilles descendantes de ce nœud, et un pointeur P vers le nœud fils. Les nœuds internes de $L_{trees}$ sont similaires sauf qu'ils ne contiennent que b et P. Avec cette structuration, $T_{tree}$ et $L_{tree}$ permettent l'ajout et la suppression des liens dans le graphe.\\
La figure \ref{dk2-trees} présente une représentation d$k^2$-tree \citep{brisaboa2012compressed}: 
\begin{figure}[H]
\begin{center}
\includegraphics[height=100 pt, width=380 pt]{./ressources/image/dk2-trees.png} 
\end{center}
\caption{Exemple d'une représentation d$k^2$-tree}
\label{dk2-trees}
\end{figure}

%\subsubsection{$k^n$-trees}
 Sandra et al. \citep{de2013compact} présentent $k^n$-tree, une généralisation des $k^2$-tree pour les problèmes multidimensionnels. Cette méthode possède plusieurs applications. Elle est utilisée pour représenter les bases de données multidimensionnelles, les données rasters et les graphes dynamiques. $k^n$-tree repose sur $k^2$-tree pour représenter une matrice à n-dimensions (dites \textit{tensor} en Anglais). La matrice est décomposé en $k^n$ sous-matrices de même taille, comme suit : sur chaque dimension, K-1 hyperplans devisent la matrice dans les positions i*$\frac{n}{K}$,pour $i \in [1, K-1]$. Une fois les dimensions partitionnées, $k^n$ sous-matrices sont induites, elles sont représentées par des nœuds dans l'arbre comme dans l'algorithme de base. Les structures utilisées pour le stockage sont aussi les mêmes (T et L).
En posant n=3, la méthode peut être appliquée sur les graphes dynamiques ou temporels. Ce type de graphe est représenté par une grille à 3 dimensions $X \times Y \times T$, où les deux premières dimensions représentent les nœuds de départ et de destination, et la troisième dimension représente le temps. 
La figure \ref{kn-trees} présente une représentation $k^3$-tree d'un graphe dynamique \citep{de2014new}.

\begin{figure}[H]
\begin{center}
\includegraphics[height=100 pt, width=380 pt]{./ressources/image/kn-trees.png} 
\end{center}
\caption{Exemple d'une représentation $k^3$-tree}
\label{kn-trees}
\end{figure}



%\subsubsection{$K^2$-tress1 }
La représentation de base des arbres k2-trees regroupe seulement les zones de zéros, puisqu'elle a été conçue au début pour les graphes du web qui possèdent une matrice d'adjacence extrêmement creuse. Dans \citep{de2014new}, Les auteurs proposent d'étendre cette représentation en regroupant les zones de "1" également. L'idée générale est d'arrêter la décomposition de la matrice d'adjacence quand une zone unie est trouvée, à savoir des zéros ou des uns. Pour distinguer entre les différents nœuds, une représentation quadtree est utilisée \citep{de1997computational}. Une couleur est attribuée à chaque nœud, blanc pour une zone de zéro, noir pour une zone de uns et gris pour les nœuds internes, i.e, les zones contenant des uns et des zéros. Pour le stockage des nœuds, les auteurs proposent quatre encodages présentés par la suite : 

\begin{enumerate}[label=$\bullet$]
\item \textbf{ $k^2$-$trees1^{2-bits-naive}$ :} Dans cet encodage, deux bits sont utilisés pour représenter chaque type de nœud. L'attribution des bits n'est pas arbitraire, le premier bit du poids fort indique si le nœud est un nœud interne (0) ou une feuille (1), le deuxième détermine si les feuilles sont blanches (0) ou noires (1). Nous aurons donc : "10" pour les nœuds gris, "01" pour les nœuds noirs et "00" pour les nœuds blancs. Notant que les feuilles du dernier niveau sont représentées par un seul bit.
Après le codage, le premier bit de chaque nœud sauf ceux du dernier niveau est stocké dans T, un autre tableau $T'$ est créé pour sauvegarder le deuxième bit, les nœuds du dernier niveau sont stockés dans L.
\item \textbf{ $k^2$-$tree1^{2-bits}$ :} Le même principe de l'encodage précédant sauf que les nœuds gris sont représentés par un seul bit, toujours à "1", le tableau $T'$ va contenir dans ce cas la couleur des feuilles ce qui va réduire la taille de la structure.
\item \textbf{ $k^2$-$tree1^{DF}$ :} Cet encodage est similaire à $k^2$-$trees1^{2-bits}$, mais il utilise un seul bit pour les nœuds blancs et deux bits pour les nœuds noirs et gris, compte tenue de la fréquence des nœuds blancs dans les graphes du monde réel par rapport aux autres. Nous aurons donc : "0" pour les nœuds blancs, "10" pour les nœuds gris et "11" pour les nœuds noirs.
\item \textbf{ $k^2$-$tree1^{5-bits}$ :} Le dernier encodage repose sur la représentation de base. Un nœud blanc est représenté par "0", un nœud noir ou gris par "1", exactement comme le $k^2$-tree d'origine. Pour identifier un nœud noir (zone de uns), il sera représenté par une combinaison impossible : $k^2$ fils de "0" sont ajoutés aux nœuds noirs pour les distinguer.
\end{enumerate}


La figure \ref{k2-trees1-exemple} illustre une représentation $k^2$-tress1 avec les quatre encodages \citep{de2014new} : 


\begin{figure}[H]
\begin{center}
\includegraphics[height=300 pt, width=450 pt]{./ressources/image/k2-trees1.png} 
\end{center}
\caption{Exemple d'une représentation $k^2$-tree1 avec les quatre encodages}
\label{k2-trees1-exemple}
\end{figure}


%\subsubsection{Delta-$K^2$-tress }

Dans \citep{zhang2014delta}, les auteurs proposent Delta-$k^2$-tree, une variante qui exploite la propriété de similarité entre les nœuds voisins du graphe pour réduire le nombre de uns dans la matrice d'adjacence. Notons par \textit{Matrix} la matrice d'adjacence, Delta-$k^2$-trees construit une nouvelle matrice appelée \textit{Delta-matrix}, une ligne \textit{i} de \textit{Delta-matrix} va contenir la différence entre les deux lignes \textit{Matrix}[i] et  \textit{Matrix}[i-1] si cela décroit le nombre de uns sinon elle sera égale à \textit{Matrix}[i], comme suit: \\

$\left\{
\begin{array}{llcl}
\textbf{Si}\ \ count1s(matrix[i]) < countDif(matrice[i],matrix[i-1])\\
\ \ \ \ Delta-matrix[i]  := matrix[i]  \\
\textbf{Sinon} & & &\\
\ \ \ \ Delta-matrix[i]  := matrix[i] \bigoplus matrix[i-1] & 
\end{array}
\right.$\\

Où \textit{count1s} compte le nombre de 1 dans une ligne, \textit{countDif} compte le nombre de bits différents entre deux lignes et $\bigoplus$ représente le "ou exclusif".

Pour déterminer si une ligne est identique à celle de la matrice d'adjacence ou non, un tableau D est utilisé : Si D[i]=1 la ligne est identique, sinon c'est une ligne de différence.\\
La matrice \textit{Delta-matrix} contient moins de uns que la matrice d'adjacence, d'où elle est plus creuse, ce qui permet de réduire la taille de la structure et avoir un meilleur taux de compression. Cependant le temps de parcours est plus grand car pour accéder a certaines lignes (lignes de différence), le graphe doit être décompressé et la matrice d'adjacence reconstituée.\\
La figure \ref{k2-trees-delta} présente un exemple de la représentation Delta-$k^2$-tree \citep{zhang2014delta}.


\begin{figure}[H]
\begin{center}
\includegraphics[height=150 pt, width=380 pt]{./ressources/image/k2-trees-delta.png} 
\end{center}
\caption{Exemple d'une représentation Delta-$k^2$-tree}
\label{k2-trees-delta}
\end{figure}

%\subsubsection{$K^2$-treaps }
$K^2$-treap est une autre variante de $k^2$-tree, elle a été proposée dans \citep{brisaboa2014k}. Cette variante combine les $k^2$-trees avec une autre structure de données appelée treap \footnote{Les Treaps sont des arbres de recherche binaire avec des nœuds ayant deux attributs : clés et priorité, la recherche dans ces arbres s'effectue selon ces attributs.} \citep{aragon1989randomized}. Les auteurs appliquent cette méthode sur des grilles multidimensionnelles comme 
\newacronym{olap}{OLAP}{Online Analytical Processing}
\gls{olap} pour pouvoir les stocker et répondre efficacement aux requêtes top-K \citep{badr2013traitement}. La méthode peut être également appliquée sur les graphes pondérés, où chaque case de la matrice d'adjacence du graphe comporte le poids de l'arête qu'elle représente au lieu d'un 1.
Comme dans l'algorithme de base, une décomposition récursive en $k^2$ sous-matrices est appliquée sur la matrice d'adjacence et un arbre $k^2$-air est construit comme suit : la racine de l'arbre va contenir les coordonnées ainsi que la valeur de la cellule ayant le plus grand poids de la matrice. La cellule ajoutée à l'arbre est ensuite supprimée de la matrice. Si plusieurs cellules ont la même valeur maximale, l'une d'elles est choisie au hasard. Ce processus est répété récursivement sur chaque sous matrice. La procédure continue sur chaque branche de l'arbre jusqu'à ce qu'on tombe sur les cellules de la matrice d'origine ou sur une sous matrice complètement vide (contient que des zéros).\\
La figure \ref{k2-treaps} suivante illustre la représentation $k^2$-treap d'un graphe pondéré \citep{badr2013traitement} :

\begin{figure}[H]
\begin{center}
\includegraphics[height=200 pt, width=380 pt]{./ressources/image/k2-treaps.png} 
\end{center}
\caption{Exemple d'une représentation $k^2$-treap d'un graphe pondéré}
\label{k2-treaps}
\end{figure}

Pour avoir une bonne compression, $k^2$-treap effectue des transformations sur les données stockées. La première transformation consiste à changer les coordonnées représentées dans l'arbre en des coordonnées relatives par rapport à la sous matrice actuelle. La deuxième est de remplacer chaque poids dans l'arbre par la différence entre sa valeur et celle de son parent.\\
Trois structures de données sont utilisées pour sauvegarder les coordonnées et les valeurs des cellules ainsi que la topologie de l'arbre. Chaque structure est détaillée dans ce qui suit :
\begin{itemize}
\item \textit{Listes de coordonnées locales :} La séquence de coordonnées de chaque niveau \textit{l} de l'arbre est stockée dans une liste \textit{coord}[\textit{l}].
\item \textit{Liste des valeurs : } Le parcours de l'arbre se fait en largeur, la séquence des valeurs récupérées est stockée dans une liste nommée \textit{values}. Un tableau nommé \textit{first} est utilisé pour sauvegarder la position début de chaque niveau dans \textit{values}.
\item \textit{L'arbre :} La structure de l'arbre $k^2$-treap est sauvegardée avec un arbre $k^2$-tree, les nœuds contenant des valeurs dans $k^2$-treap sont représentés par des uns et les nœuds vides par des zéros. Pour le stockage de l'arbre, un seul tableau T est utilisé. 
\end{itemize}

La figure \ref{k2-treaps-structure} représente les structures de données utilisées pour le stockage de l'arbre de la figure \ref{k2-treaps} précédente \citep{badr2013traitement} :

\begin{figure}[H]
\begin{center}
\includegraphics[height=200 pt, width=380 pt]{./ressources/image/k2-treaps-structure.png} 
\end{center}
\caption{Structures de données pour une représentation $k^2$-treap}
\label{k2-treaps-structure}
\end{figure}

%\subsubsection{I$k^2$-trees}
Garcia et al. \citep{garcia2014interleaved} proposent I$k^2$-tree pour Interleaved $k^2$-tree. Elle est appliquée sur les bases de données RDF ainsi que sur les graphes dynamiques. Dans les graphes dynamiques, les deux premières dimensions correspondent aux nœuds source et destination et la troisième dimension reflète le temps. Le graphe est donc défini par |T| matrices d'adjacence prisent à des instants $t_k$ différents. I$k^2$-tree représente les matrices simultanément. Chaque matrice est représentée par un arbre $k^2$-tree et ils sont par la suite regroupés dans un seul arbre. Chaque nœud de l'arbre obtenu représente une sous-matrice comme dans l'algorithme de base, sauf qu'au lieu d'utiliser un seul bit, I$k^2$-tree utilise 1 à |T| bits pour représenter le nœud. Le nœud racine contient |T| bits. Le nombre de bits de chaque fils dépend du nombre de uns de son parent. L'arbre finale est toujours stocké avec deux tableaux : T et L.
La figure \ref{Ik2-trees} est un exemple de I$k^2$-tree appliqué sur un graphe dynamique représenté dans trois instants \citep{garcia2014interleaved} :

\begin{figure}[H]
\begin{center}
\includegraphics[scale=0.55]{./ressources/image/Ik2-trees.png} 
\end{center}
\caption{Exemple d'une représentation I$k^2$-tree}
\label{Ik2-trees}
\end{figure}

%\subsubsection{diff I$k^2$-trees}
Une variante de I$k^2$-tree appelée Differential I$k^2$-tree a été étudiée dans \citep{alvarez2017succinct}. Elle vise à améliorer le taux de compression en représentant uniquement les changements survenus sur le graphe à un instant $t_i$ au lieu d'une instance complète : à l'instant $t_0$, une capture complète du graphe (matrice d'adjacence) est stockée. À l 'instant $t_k$, pour k>0, seules les arrêtes qui changent de valeurs entre $t_{k-1}$ et $t_k$ sont stockées. Les matrices sont représentées à la fin de la même manière que I$k^2$-tree. La limite de cette représentation est que la structure doit être décompressée lors d'une requête.
La figure \ref{Ik2-trees-diff} montre un exemple d'une représentation diffI$k^2$-trees \citep{alvarez2017succinct}.

\begin{figure}[H]
\begin{center}
\includegraphics[scale=0.8]{./ressources/image/Ik2-trees-diff.png} 
\end{center}
\caption{Exemple d'une représentation diff I$k^2$-tree}
\label{Ik2-trees-diff}
\end{figure}

%\subsubsection{Att $K^2$-trees }
Dans \citep{alvarez2018compact}, les auteurs étendent la représentation $k^2$-tree pour les bases de données orientées graphes. Ces graphes sont étiquetés, attribués, orientés et ont des arêtes multiples. Ils présentent le graphe sous forme d'une nouvelle structure intitulée Att$k^2$-tree pour Attributed $k^2$-trees.
La figure \ref{k2-trees-att-graphe} montre un exemple de graphe pris en compte par Att$K^2$-tree \citep{alvarez2018compact}.

\begin{figure}[H]
\begin{center}
\includegraphics[height=200 pt, width=280 pt]{./ressources/image/k2-trees-att-graphe.png} 
\end{center}
\caption{Exemple d'un graphe étiqueté, attribué, orienté et multiple}
\label{k2-trees-att-graphe}
\end{figure}

\textbf{Structures de données :} La représentation obtenue par la compression est composée d'un ensemble d'arbres $k^2$-tree et d'autres structures supplémentaires. Le graphe est représenté par trois composantes: un schéma de données, les données incluses dans les nœuds et les liens et finalement la relation entre les éléments du graphe. Chaque composant est présenté dans ce qui suit :
\begin{itemize}
\item \textit{Schéma : } Ce composant gère les étiquettes et les attributs de chaque type d'éléments, il joue le rôle d'un index dans la représentation. Il est composé de :
\begin{description}
\item[Un schéma de nœuds :] représenté par un tableau qui contient les étiquettes des nœuds ordonnées lexicographiquement. Un identifiant est attribué à chaque nœud du graphe selon l'ordre du tableau, les \textit{$m_1$} nœuds possédant la première étiquette du tableau vont avoir des identifiants de 1 à \textit{$m_1$}, les \textit{$m_2$} nœuds avec la deuxième étiquette du tableau vont avoir des identifiants de \textit{$m_1$}+1 à \textit{$m_1$}+\textit{$m_2$} et ainsi de suite. Chaque entrée du tableau va ainsi stocker le plus grand identifiant portant son étiquette, cela permet de trouver l'étiquette d'un nœud à travers son identifiant.
\item[Un schéma d'arêtes :] Comme dans le cas des nœuds, un tableau est utilisé pour stocker les étiquettes des arêtes avec le même principe.  
\end{description}
Le schéma est le point de départ de la représentation, il permet d'obtenir l'étiquette d'un nœud ou d'une arête, et d'accéder à ses attributs.
\item \textit{Données :} Ce composant contient toutes les valeurs que peut prendre un attribut dans le graphe. Un attribut peut être représenté de deux façons différentes selon sa fréquence d'apparition, on distingue donc deux types d'attributs :
\begin{description}
\item[Attributs rares :] Ce sont les attributs qui prennent généralement des valeurs différentes à chaque apparition, ils sont stockés dans des listes et indexés avec l'identifiant de l'élément.
\item[Attributs fréquents :] Ce type d'attributs est sauvegardé dans deux matrices, une pour les attributs des nœuds et l'autre pour les attributs des liens. Les matrices sont stockées sous forme d'arbres $k^2$-trees.
\end{description}
La figure \ref{k2-trees-att-schema} illustre les deux composants schéma et données de la représentation Att$k^2$-trees  de la figure \ref{k2-trees-att-graphe} \citep{alvarez2018compact}:
\begin{figure}[H]
\begin{center}
\includegraphics[scale=0.7]{./ressources/image/k2-trees-att-schema.png} 
\end{center}
\caption{Exemple d'une la représentation Att$k^2$-tree (1/2)}
\label{k2-trees-att-schema}
\end{figure}


\item \textit{Relations :} C'est le dernier composant de Att$k^2$-tree, il stocke les relations entre les nœuds et les arêtes du graphe en utilisant un arbre $k^2$-tree et d'autres structures pour sauvegarder les identifiants des arêtes ainsi que les arêtes multiples. Les structures supplémentaires sont les suivantes :
\begin{description}
\item[Multi :] Un tableau qui indique si l'arête est multiple ou non.
\item[Firt :] Un tableau qui donne l'identifiant de l'arête, ou de celui de la première dans le cas d'une arrête multiple.
\item[Next :] Un tableau qui contient les identifiants des arêtes multiples restantes.
\end{description}
La figure \ref{k2-trees-att-relation} donne la représentation Att$k^2$-tree des relations du graphe de la figure \ref{k2-trees-att-graphe} \citep{alvarez2018compact}:
\begin{figure}[H]
\begin{center}
\includegraphics[height=200 pt, width=280 pt]{./ressources/image/k2-trees-att-relation.png} 
\end{center}
\caption{Exemple d'une la représentation Att$k^2$-tree (2/2)}
\label{k2-trees-att-relation}
\end{figure} 
\end{itemize}

%\subsubsection{dynAtt$k^2$-trees }
Dans le même article \citep{alvarez2018compact}, les auteurs étendent Att$k^2$-tree pour les graphes dynamiques. Ils proposent une nouvelle variante appelée dynAtt$k^2$-tree qui supporte le changement dans les attributs et les liens du graphe. Comme Att$k^2$-tree, dynAtt$k^2$-tree représente le graphe avec trois composantes : Schémas, données et relations. Les composantes sont semblables à ceux de Att$k^2$-tree mais avec certaines amélioration vue la nature dynamique du graphe.\\
\textbf{Structure de données : } 
\begin{itemize}
\item \textit{Schéma :}  En ce qui concerne les nœuds, leurs étiquettes sont stockées dans une liste dynamique ordonnée lexicographiquement. En outre, une séquence dynamique est utilisée pour sauvegarder le type de chaque nœud. Elle est stockée ensuite sous forme d'un arbre d'ondelettes \footnote{Ou wavelet en anglais, est un arbre binaire équilibré qui contient des données compressées dans une représentation optimale.} \citep{grossi2003high}. Le même principe est appliqué sur les arrêtes. 
\item \textit{Données :} Les attributs rares sont stockés dans des listes dynamiques, quant aux attributs fréquents, ils sont sauvegardés avec des arbres d$k^2$-trees (un arbre pour chaque attribut).
\item \textit{Relations :} Le stockage des relations se fait à l'aide d'un d$k^2$-tree et des tableaux dynamiques pour stocker les identifiants des arêtes et les arêtes multiples.
\end{itemize}





























				
\subsubsection{Synthèse des méthodes de compression basée sur les $K^2$-trees }

Nous avons présenté dans les parties précédentes les différentes méthodes de compression existantes basées sur la représentation $k^2$-trees, et expliqué le principe de fonctionnement de chaqu'une d'elles. Nous allons par la suit comparer entre ces méthodes et résumer nos recherches.

Une présentation synthétique de notre étude est fournie dans le tableau \ref{K2-trees-table}. Chaque ligne du tableau représente une méthode, tandis que chaque colonne
représente un aspect susceptible d’être utilisé dans la méthode (type de graphe, type de compression, structure en sortie). Nous constatons que toutes les méthodes de cette classe sont des méthodes de compression sans perte qui supportent les graphes orientés et non orientés, cependant nous remarquons une variation dans l'aspect temporelle, certains méthodes sont destinées aux graphes statiques comme $k^2$-trees de base et $k^2$-trees1, tandis que d'autres sont appliquées aux graphes dynamiques comme $k^n$-trees et d$k^2$-trees. Certains méthodes se distinguent aussi par le type de graphe comme $k^2$-treaps qui s'applique aux graphes pondérés et Att$k^2$-trees qui accepte les graphes étiquetés, attribués et multiples. Nous observons aussi que toutes les méthodes donnent en sortie une représentation succincte. le tableau résume aussi les résultats de l'application des méthodes sur des graphes de test. Les figures \ref{eu2005-k2} et \ref{k2-taux} sont des représentations visuelles des ces résultats.\\

La figure \ref{eu2005-k2} représente les résultats obtenues par l'application des méthodes de compression basées sur les $k^2$-trees sur le graphe eu-2005 en tenant compte de l’ordre chronologique de leurs apparitions. Les résultats sont sous forme de nombre de bits par liens. Nous remarquons que la première méthode de base donne un résultat de compression de 5.2bpe, ce nombre a diminué avec l'apparition des améliorations et des nouvelles méthodes. Nous constatons alors une amélioration contenue de l'efficacité de ces méthodes de manière générale surtout avec les améliorations proposées par \citep{brisaboa2014compact} et Delta$k^2$-trees où le nombre de bits atteint approximativement 3.22 bits par liens.

 \begin{figure}[H]
		\includegraphics[height=10.5cm, width=16cm]{./ressources/image/eu2005-k2.png}
		\caption{Comparaison des performances des méthodes basées sur les $k^2$-trees appliquées sur le graphe eu2005 selon le nombre de bits par liens}
		\label{eu2005-k2}
	\end{figure}
	
A partir de la figure \ref{K2-trees-table} qui présente les résultats de l'application de différentes méthodes sur quatre graphes de test (CommNet, cond-mat, eu-2005, movilens10M) en fonction du taux de compression. Nous remarquons que entre $k^2$-trees et $k^2$-trees1, la méthode $k^2$-trees est nettement meilleure, cela s'explique par le fait que les graphes du web sont représentés par des matrices creuses et $k^2$-trees1 est mieux adaptée au matrices avec des zones de uns comme les matrices d'images par exemple.	Pour les graphes dynamiques, nous observons que $k^n$-trees donne de bons résultats par rapport à I$k^2$-trees mais avec une légère différence. Pour les graphes attribués, Att$k^2$-trees donne un très bon résultat avec un taux de compression de 93\%. 
	\begin{figure}[H]
		\includegraphics[height=9cm, width=18cm]{./ressources/image/k2-taux.png}
		\caption{Comparaison entre les méthodes basées sur les $k^2$-trees en fonction du taux de compression}
		\label{k2-taux}
	\end{figure}
	


    
 												
														\begin{landscape}
								\begin{table}
									\begin{tabular}{|C{3cm}|c|c|c|c|c|c|c|c|c|c|c|c|c|}
										\hline
										\multirow{2}{*}[-25pt]{     Article   }  & \multicolumn{5}{c|}{Graphe en entrée} & \multicolumn{2}{c|}{Compression} & \multicolumn{2}{c|}{Structure en sortie} & \multirow{2}{*}[-25pt]{Graphe de test} & \multirow{2}{*}[-25pt]{Résultat }  \\ \cline{2-10}
				     & \rotatebox[origin=c]{90}{ Orienté }  & \rotatebox[origin=c]{90}{ Non orienté } & \rotatebox[origin=c]{90}{ Statique } & \rotatebox[origin=c]{90}{ Dynamique } & \rotatebox[origin=c]{90}{ Autre Propriétés } &  \rotatebox[origin=c]{90}{ Avec perte } & \rotatebox[origin=c]{90}{ Sans perte } & \rotatebox[origin=c]{90}{ Succincte } & \rotatebox[origin=c]{90}{ Structurelle } & & \\ \hline				%%%%%%Fin du header
				
\hline  $k^2$-trees : Algorithme de base
   \citep{brisaboa2009k} 
   & \cmark & \cmark & \cmark & \xmark &  & \xmark &  \cmark & \cmark & \xmark	 & 		
	\begin{minipage}[t]{0.15\textwidth}
	eu-2005\\
	
	cond-mat
  \end{minipage}	
										 &
	\begin{minipage}[t]{0.4\textwidth}
	 5.21 bits/lien \\
	 taux de compression : 16.88\% \\
	 taux de compression : 15.58\% 
  \end{minipage}	\\

\hline $k^2$-trees : Hybridation \citep{brisaboa2009k} & \cmark & \cmark & \cmark & \xmark &  & \xmark &  \cmark & \cmark & \xmark  & 
										\begin{minipage}[t]{0.15\textwidth}
	eu-2005
  \end{minipage}	
										 &
	\begin{minipage}[t]{0.4\textwidth}
	 5.21 bits/lien 
  \end{minipage}	\\  
\hline $k^2$-trees : Optimisation \citep{shi2012optimizing} & \cmark & \cmark & \cmark & \xmark &  & \xmark &  \cmark & \cmark & \xmark  & 
\begin{minipage}[t]{0.15\textwidth}
	cond-mat
  \end{minipage}	
										 &
	\begin{minipage}[t]{0.4\textwidth}
	
	 taux de compression : 37.96\% 
  \end{minipage}	\\
  			\hline
  			
\hline $k^2$-trees : Amélioration \citep{brisaboa2014compact} & \cmark & \cmark & \cmark & \xmark &  & \xmark &  \cmark & \cmark & \xmark  & 
  				\begin{minipage}[t]{0.15\textwidth}
	eu-2005
  \end{minipage}	
										 &
	\begin{minipage}[t]{0.4\textwidth}
	
	 3.22 bits/lien
  \end{minipage}	\\
  				
  				\hline  		
  			
\hline d$k^2$-trees \citep{brisaboa2012compressed} & \cmark & \cmark & \xmark & \cmark &  & \xmark &  \cmark & \cmark & \xmark  &
  				\begin{minipage}[t]{0.15\textwidth}
	eu-2005
  \end{minipage}	
										 &
	\begin{minipage}[t]{0.4\textwidth}
	
	 6.2 bits/lien
  \end{minipage}	\\
  \hline 
  			
\hline $k^n$-trees \citep{de2013compact} & \cmark & \cmark & \xmark & \cmark & & \xmark & \cmark & \cmark & \xmark  &
  				\begin{minipage}[t]{0.15\textwidth}
	CommNet
  \end{minipage}	
										 &
	\begin{minipage}[t]{0.4\textwidth}
	
	 taux de compression : 65.16\% 
  \end{minipage}	\\
  \hline  	

  			

  					
 	
  			
									\end{tabular}
									\caption{Synthèse des méthodes de compression par $k^2$-trees.}									
									
								\end{table}
								
							\end{landscape}
							
							
							
%% deuxiéme tableau

					\begin{landscape}
								\begin{table}
									\begin{tabular}{|C{3cm}|c|c|c|c|c|c|c|c|c|c|c|c|c|}
										\hline
										\multirow{2}{*}[-25pt]{Article}  & \multicolumn{5}{c|}{Graphe en entrée} & \multicolumn{2}{c|}{Compression} & \multicolumn{2}{c|}{Structure en sortie} & \multirow{2}{*}[-25pt]{Graphe de test} & \multirow{2}{*}[-25pt]{Résultat }  \\ \cline{2-10}
				& \rotatebox[origin=c]{90}{ Orienté }  & \rotatebox[origin=c]{90}{ Non orienté } & \rotatebox[origin=c]{90}{ Statique } & \rotatebox[origin=c]{90}{ Dynamique } & \rotatebox[origin=c]{90}{ Autre Propriétés } &  \rotatebox[origin=c]{90}{ Avec perte } & \rotatebox[origin=c]{90}{ Sans perte } & \rotatebox[origin=c]{90}{ Succincte } & \rotatebox[origin=c]{90}{ Structurelle } & & \\ \hline				%%%%%%Fin du header
				

  			
\hline $k^2$-trees1\citep{de2014new} & \cmark & \cmark & \cmark & \xmark & & \xmark & \cmark & \cmark & \xmark  & 
  							\begin{minipage}[t]{0.1\textwidth}
	eu-2005
  \end{minipage}	
										 &
	\begin{minipage}[t]{0.35\textwidth}
	 taux de compression : 16.23\% 
  \end{minipage}	\\
  \hline  
  			
\hline Delta-$k^2$-trees  \citep{zhang2014delta} & \cmark & \cmark & \cmark & \xmark & & \xmark & \cmark & \cmark & \xmark  & 
  							\begin{minipage}[t]{0.1\textwidth}
	eu-2005
  \end{minipage}	
										 &
	\begin{minipage}[t]{0.35\textwidth}
	 3.24 bits/lien 
  \end{minipage}	\\
  \hline  
  
\hline $k^2$-treaps  \citep{brisaboa2014k} & \cmark & \cmark & \cmark & \xmark & 
\begin{minipage}[t]{0.15\textwidth}
  			Pondéré 
  \end{minipage}		
   & \xmark & \cmark & \cmark & \xmark  & 
  							\begin{minipage}[t]{0.1\textwidth}
	SalesDay
  \end{minipage}	
										 &
	\begin{minipage}[t]{0.35\textwidth}
	 2.48 bits/lien
  \end{minipage}	\\
  \hline  
 
\hline I$k^2$-trees  \citep{garcia2014interleaved} & \cmark & \cmark & \xmark & \cmark & & \xmark & \cmark & \cmark & \xmark  & 
  							\begin{minipage}[t]{0.1\textwidth}
	CommNet
  \end{minipage}	
										 &
	\begin{minipage}[t]{0.35\textwidth}
	
	 taux de compression : 60.43\% 
  \end{minipage}	\\
  \hline  	
  			
 
\hline Diff I$k^2$-trees  \citep{alvarez2017succinct} & \cmark & \cmark & \xmark & \cmark & & \xmark & \cmark & \cmark & \xmark  & 
  							\begin{minipage}[t]{0.1\textwidth}
	CommNet
  \end{minipage}	
										 &
	\begin{minipage}[t]{0.35\textwidth}
	
	 taux de compression : 60.43\% 
  \end{minipage}	\\
  \hline 
				
				 
\hline Att $k^2$-trees  \citep{alvarez2018compact} & \cmark & \cmark & \cmark & \xmark & 
\begin{minipage}[t]{0.1\textwidth}
  			Étiqueté\\
  			Attribué\\
  			Multiple
  \end{minipage}	
& \xmark & \cmark & \cmark & \xmark  & 
  							\begin{minipage}[t]{0.15\textwidth}
	Movielen-10M
  \end{minipage}	
										 &
	\begin{minipage}[t]{0.35\textwidth}
	
	 taux de compression : 89.97\% 
  \end{minipage}	\\
  \hline 
  			
\hline dynAtt $k^2$-trees  \citep{alvarez2018compact} & \cmark & \cmark & \xmark & \cmark & 
\begin{minipage}[t]{0.1\textwidth}
  			Étiqueté\\
  			Attribué\\
  			Multiple
  \end{minipage}	
& \xmark & \cmark & \cmark & \xmark  & 
			\begin{minipage}[t]{0.1\textwidth}
	Movielen-10M
  \end{minipage}	
										 &
	\begin{minipage}[t]{0.35\textwidth}
	
	 taux de compression : 93.75\% 
  \end{minipage}	\\  				
  				\hline 
				
									\end{tabular}
									\caption{Synthèse des méthodes de compression par $k^2$-trees.}									
								\label{K2-trees-table}
								\end{table}
								
							\end{landscape}			
							
							
	
			\section{Compression par extraction de motifs}
			 Les motifs fréquents sont des connaissances extraites sur des données. Leur but est de fournir à l'utilisateur des informations non triviales, implicites, présumées non connues. Ils lui offrent ainsi une meilleure appréhension des données. Dès lors, l'extraction de motifs fréquents est  devenue une tâche importante dans la fouille de données et un thème très étudié par la communauté. Elle a aussi été vastement %%%% verifier ce mot
			 utilisée dans le domaine de compression des graphes vu qu'elle permet de ne garder que l'information utile et d'éliminer les redondances de manière efficace. En effet, nous trouvons plusieurs méthodes basées sur ce principe où nous pourrons clairement distinguer deux grandes classes : 
			 (i) les méthodes de compression basées vocabulaire
			 (ii) les méthodes de compression basées Agrégation.
			 
				Dans cette section, nous allons expliquer le principe de base de chaque classe où nous allons subdiviser chacune en plusieurs sous-classes en se basant sur ce dernier. 
				%%% a revoir le dernier paragraphe
			 
				\subsection{Compression basée vocabulaire}
				\label{vog_desc}
				Les méthodes de compression par extraction de motifs basées vocabulaire sont des méthodes qui ont attirées l'attention des chercheurs ces dernières années car elles permettent une meilleure compréhension du graphe. Elles partent toujours d'un ensemble de structures prédéfinies qui ont été prouvées fréquentes dans les graphes réels. Deux sous classes de cette dernières peuvent être identifiées :
				 
					 \textbf{Les Méthodes basées sur des techniques de clustering}
							
							Les méthodes de cette classe s'appuient sur le fait qu'on ne peut pas comprendre facilement les graphes denses, alors que quelques structures simples sont beaucoup plus faciles à comprendre et souvent très utiles pour analyser le graphe. Elles se basent sur des algorithmes de détection de communautés. 
							La question suivante peut alors se poser: pourquoi ne pas appliquer l'un des nombreux algorithmes de détection de communautés ou de partitionnement de graphes pour compresser le graphe en termes de communautés? La réponse est que ces algorithmes ne servent pas tout à fait le même objectif que la compression. Généralement, ils détectent de nombreuses communautés sans ordre explicite, de sorte qu'une procédure de sélection des sous-graphes les plus «importants» est toujours nécessaire. En plus de cela, ces méthodes renvoient simplement les communautés découvertes, sans les caractériser (par exemple, clique, étoile) et ne permettent donc pas à l'utilisateur de mieux comprendre les propriétés du graphe. 
							
							%Le partitionnement des graphes et la détection de communautés présentent un grand intérêt pour de nombreux domaines, notamment les sciences sociales, biologiques et Web. 
%			Ces dernières permettent d'obtenir des résumés de graphes diversifiés, chaque méthode étant orientée vers certains types de structures, telles que des cliques et des noyaux bipartites ou des étoiles. 
			Parmi les méthodes de détection les plus importantes dans la littérature nous trouvons :
			\begin{enumerate}
				\item \textbf{METIS\citep{karypis2000multilevel}}:est un schéma de partitionnement de graphe multiniveau basé sur la bissection récursive multiniveau (MLRB). Tant que la taille du graphe n'est pas sensiblement réduite, il grossit d'abord le graphe d'entrée en regroupant les nœuds dans des super-nœuds de manière itérative, de sorte que la coupe des bords soit préservée. Ensuite, le graphe grossi est partitionné à l'aide de MLRB et le partitionnement est projeté sur le graphe d'entrée G par le biais du retour en arrière. La méthode produit k partitions à peu près égales.
				\item \textbf{SPECTRAL\citep{hespanha2004efficient}}: partitionne un graphe en effectuant une classification en k-means sur les k premiers vecteurs propres du graphe d'entrée. L'idée derrière cette classification est que les nœuds avec une connectivité similaire ont des scores propres similaires dans les k premiers vecteurs.
				\item \textbf{LOUVAIN\citep{blondel2008fast}}:est une méthode de partitionnement basée sur la modularité pour détecter la structure de communauté hiérarchique. LOUVAIN est une méthode itérative: (i) Chaque nœud est placé dans sa propre communauté. Ensuite, les voisins j de chaque nœud i sont pris en compte et i est déplacé vers la communauté j si le déplacement produit le gain de modularité maximum. Le processus est appliqué à plusieurs reprises jusqu'à ce qu'aucun gain supplémentaire ne soit possible. (ii) Un nouveau graphe est créé dont les super nœuds représentent les communautés et les super arêtes sont pondérés par la somme des poids des liens entre les deux communautés. L'algorithme converge généralement en quelques itérations.
				\item \textbf{SLASHBURN \citep{kang2011beyond}}:est un algorithme de ré-ordonnancement de nœud initialement développé pour la compression de graphes. Il effectue deux étapes de manière itérative: (i) il supprime les nœuds de haute centralité du graphe (ii) Il réorganise les nœuds de manière à ce que les identificateurs les plus petits soient attribués aux nœuds de degré élevé et les nœuds des composants déconnectés obtiennent les identificateurs les plus grands. Le processus est répété sur le composant connecté.
				\item \textbf{BIGCLAM\citep{yang2013overlapping}}:est une méthode de détection de communauté à chevauchement évolutive. Elle est construite sur le constat que les chevauchements entre les communautés sont étroitement liés. En modélisant explicitement la force d’affiliation de chaque couple nœud-communauté, un facteur latent non négatif est attribué à cette dernière, qui représente le degré d'appartenance à la communauté. Ensuite, la probabilité d'un bord est modélisée en fonction des affiliations de communautés partagées.
				\item \textbf{HYCOM\citep{araujo2014beyond}}:est un algorithme qui détecte les communautés à structure hyperbolique. Il se rapproche de la solution optimale en détectant de manière itérative les communautés importantes. L'idée clé est de trouver dans chaque étape une communauté unique qui minimise une fonction d'objectif basée sur le principe MDL en fonction des communautés précédemment détectées. 
				
			\end{enumerate}
			
			%%%%%Un tableau Comparative
			 Nous présentons dans le tableau \ref{tab:clusteringMethode} une synthèse entre ces méthodes dans le but de mieux comprendre leurs impact sur les méthodes de compression. Nous constatons que le choix de l'algorithme de clustering peut orienter le processus de recherche vers un ou plusieurs types des structures ce qui influence le résultat de la compression.
			 \begin{table}[!h]
							\footnotesize
							\begin{tabular}{|R{2cm}||C{1cm}|C{1.5cm}|C{1.5cm}|C{1.5cm}|C{1.5cm}|C{1.5cm}|C{2.5cm}|}
								\hline & Chevau-che-ment & Clique & Étoile & sous-graphe Bipartie & Chaîne & Structure Hyperbolique & Complexité \\
								\hline \textbf{Metis} & \textcolor{red}{\xmark} & \textcolor{PineGreen}{Beaucoup} & \textcolor{BurntOrange}{certaines} & \textcolor{BurntOrange}{certains} & \textcolor{red}{peu}  & \textcolor{red}{peu}  & O(m·k)\\
								\hline	\textbf{Spectral} & \textcolor{red}{\xmark} & \textcolor{PineGreen}{Beaucoup} & \textcolor{BurntOrange}{certaines} & \textcolor{PineGreen}{Beaucoup} & \textcolor{red}{peu}  & \textcolor{red}{peu}  & O($n^3$)\\
								\hline \textbf{Louvain} & \textcolor{red}{\xmark} & \textcolor{PineGreen}{Beaucoup} & \textcolor{BurntOrange}{certaines} & \textcolor{red}{peu}  & \textcolor{red}{peu}  & \textcolor{red}{peu} & O($n log n$)\\
								\hline \textbf{SlashBurn} & \textcolor{PineGreen}{\checkmark} & \textcolor{PineGreen}{Beaucoup} & \textcolor{PineGreen}{Beaucoup} & \textcolor{BurntOrange}{certains} & \textcolor{red}{peu} & \textcolor{red}{peu} & O(t($m+nlogn$)) \\
								\hline \textbf{Bigclam} & \textcolor{PineGreen}{\checkmark}  & \textcolor{PineGreen}{Beaucoup} & \textcolor{BurntOrange}{certaines} & \textcolor{red}{peu}  & \textcolor{red}{peu}  & \textcolor{red}{peu}  & O(d · n · t)\\					
								\hline \textbf{Hycom} & \textcolor{PineGreen}{\checkmark}  & \textcolor{BurntOrange}{certaines} & \textcolor{PineGreen}{Beaucoup} & \textcolor{BurntOrange}{certains} & \textcolor{red}{peu}  & \textcolor{PineGreen}{Beaucoup}  & O($k(m + h log h^2 +hm_{h} )$)\\
								
								
								
								\hline \textbf{KCBC} & \textcolor{PineGreen}{\checkmark} & \textcolor{PineGreen}{Beaucoup} & \textcolor{BurntOrange}{certaines} & \textcolor{red}{peu}  & \textcolor{red}{peu}  & \textcolor{red}{peu}  & O(t(m + n))\\
								
								\hline 
							\end{tabular} \\
								\caption{Tableau comparative entre les méthodes de clustering avec n = nombre de nœuds, m = nombre d'arêtes, k = nombre de clusters, t = nombre d'itérations, d = degré moyen de nœuds, h($m_{h}$) = nombre de nœuds (arêtes) dans la structure hyperbolique.}
    								\label{tab:clusteringMethode} 
							\end{table}
							\normalsize	
			 \newpage
							%%VOG*

Une première technique de compression usant de ces méthodes s'intitule 
\newacronym{vog}{VOG}{Vocabulary Graph}
\gls{vog}
 \citep{koutra2015summarizing}. C'est une méthode de base sur laquelle s'appuient toutes les autres méthodes de cette classe. Elle permet de compresser un graphe statique non orienté G à l'aide d'un vocabulaire de sous-structures qui apparaissent fréquemment dans les graphes réels et ayant  une signification sémantique, tout en minimisant le cout du codage en utilisant le principe de longueur de description minimale (\gls{mdl})\footnote{MDL est un concept de la théorie de l'information permettant de trouver le modèle ayant une longueur minimale: \textit{min}(\textit{D},\textit{M}) = L(\textit{M}) + L(\textit{D} | \textit{M}) où L(\textit{M}) est la longueur du modèle et L(\textit{D} | \textit{M}) est la longueur en bits de la description des donnés en utilisant le modèle \textit{M}.}. 
%Soit un graphe non orienté G(V,E), sans boucle avec n nœuds et m arêtes et soit A sa matrice d'adjacence. le résultat de \gls{vog} est une liste ordonnée de structures notés \textit{M}, on note par 
Le vocabulaire $\Omega$ utilisé est composé de six structures qui sont : clique (\textit{fc}) et quasi-clique (\textit{nc}), noyau bipartie (\textit{cb}) et quasi-noyau bipartie (\textit{nb}), étoile (\textit{st}) et chaine (\textit{ch}). On peut avoir un chevauchement au niveau des nœuds, les liens quand à eux sont pris selon un ordre FIFO et ne peuvent pas se chevaucher, i.e, la première structure $ s \in \textit{M} $ qui décrit l'arête dans A détermine sa valeur.

On note par $\mathcal{C}_{x}$ l'ensemble de tous les sous-graphes possibles de type $x \in \Omega$, et $\mathcal{C}$ l'union de tous ces ensembles, $\mathcal{C}$ = ${\cup}_{x}\mathcal{C}_{x}$. La famille de modèles noté $\mathcal{M}$ représente toutes les permutations possibles des éléments de $\mathcal{C}$. Par MDL, on cherche $\textit{M} \in \mathcal{M}$ qui minimise le mieux le cout de stockage du modèle et de la matrice d'adjacence.
En d'autre terme, \gls{vog} formule le problème de compression comme un problème d'optimisation dont la fonction objective est :
 
 
 \textit{min}(\textit{D},\textit{M}) = L(\textit{M}) + L(E) avec E= A $\bigoplus$ $\mathbf{M} $ représentant l'erreur. 

Pour l'encodage du modèle, on a pour chaque $\textit{M} \in  \mathcal{M}$ : 

\begin{center}
L(\textit{M}) = $L_{\mathbb{N}}$(|M|+1) + log $\left( \begin{array}{c}
|\textit{M}| + |\Omega| -1 \\
|\Omega| -1 \\
\end{array} \right)$ + $\sum\limits_{s  \in \textit{M}} ( - log Pr(x(s)  |  \textit{M} ) + L(s) )$\\
\end{center}

Le premier terme représente le nombre de structures dans le modèle  %$L_{\mathbb{N}}$
, le second terme encode le nombre de structures par type $x \in \Omega$ tant dis que le troisième terme permet pour chaque structure $s \in \textit{M}$, d'encoder son type x(s) avec un code de préfixe optimal et d'encoder sa structure. Le codage des structures se fait selon leurs types :
%----------------------------------------------------------------
%----------------------------------------------------------------
%----------------------------------------------------------------

\textbf{Clique :} Pour l'encodage d'une clique, on calcule le nombre de nœuds de celle-ci, et on encode leurs ids :
L(\textit{fc}) = $L_{\mathbb{N}}$(|\textit{fc}|) + log ${n}\choose{|f_{c}|}$
%----------------------------------------------------------------
%----------------------------------------------------------------
%----------------------------------------------------------------

\textbf{Quasi-Clique :} Les quasi-cliques sont encodées comme des cliques complètes, tout en identifiant les arêtes ajoutées dont le nombre est ||\textit{nc}|| et manquantes dont le nombre est noté ||\textit{nc}||' en utilisant des codes de préfixe optimaux : 
L(\textit{nc}) = $L_{\mathbb{N}}$(|\textit{nc}|) + log ${n}\choose{|n_{c}|}$ + log(|\textit{nc})|) + ||\textit{nc}||\textit{$l_{1}$} +  ||\textit{nc}||'\textit{$l_{0}$}
où \textit{$l_{1}$} = - log(||\textit{nc}||/(||\textit{nc}||+||\textit{nc}||')) et analogique à \textit{$l_{0}$} sont les longueurs des codes de préfixe optimaux des arêtes  ajoutées et manquantes.
%----------------------------------------------------------------
%----------------------------------------------------------------
%----------------------------------------------------------------

\textbf{Noyau biparti :} notant par A et B les deux ensembles du noyau bipartie. On encode leurs tailles, ainsi que les identifiants de leurs sommets : 
L(\textit{fb}) = $L_{\mathbb{N}}$(|\textit{A}|) + $L_{\mathbb{N}}$(|\textit{B}|) + log ${n}\choose{|A|}$ + log ${n-|A|}\choose{|B|}$.
%----------------------------------------------------------------
%----------------------------------------------------------------
%----------------------------------------------------------------

\textbf{Quasi-Noyau biparti :} Comme les quasi-cliques, les quasi-noyaux bipartie sont codés comme suit :
L(\textit{nb}) = $L_{\mathbb{N}}$(|\textit{A}|) + $L_{\mathbb{N}}$(|\textit{B}|) + log ${n}\choose{|A|}$  + log ${n-|A|}\choose{|B|}$+log(|area(\textit{nb})|) + ||\textit{nb}||\textit{$l_{1}$} + ||\textit{nb}||'\textit{$l_{0}$}.
%----------------------------------------------------------------
%----------------------------------------------------------------
%----------------------------------------------------------------

\textbf{Étoile :} L'étoile est un cas particulier d'un noyau bipartie. D'abord on calcule le nombre de nœuds des extrémités de l'étoile, ensuite on identifie le nœud central parmi les n sommets et les nœuds des extrémités parmi les n-1 restants.
L(\textit{st}) = $L_{\mathbb{N}}$(|\textit{st}|-1) +  log n + log ${n-1}\choose{|st|-1}$.
%----------------------------------------------------------------
%----------------------------------------------------------------
%----------------------------------------------------------------

\textbf{Chaine :} On calcule d'abord le nombre d'éléments de la chaine, ensuite on encode les identifiants des nœuds selon leurs ordre dans la chaine :
L(\textit{ch}) = $L_{\mathbb{N}}$(|\textit{ch}| - 1) + $\sum\limits_{i=0}^{|\textit{ch}|} ( n - i )$

%----------------------------------------------------------------
%---------------------La matrice de l'erreur----------------------
%----------------------------------------------------------------
\textbf{Matrice d'erreur :} la matrice d'erreur E est encodée en deux parties $E^{+}$ et $E^{-}$. $E^{+}$  correspond à la partie de A que M modélise en rajoutant des liens non existants contrairement à $E^{-}$ qui représente les parties de A que M ne modélise pas. Notons que les quasi-cliques et les quasi-noyaux bipartie ne sont pas inclut dans la matrice d'erreur puisqu'ils sont encodés avec exactitude. Le codage de $E^{+}$ et $E^{-}$ est similaire à celui des quasi-cliques, on a :
\begin{center}
L(${E}^{+}$) = log(|${E}^{+}$|) + ||${E}^{+}$||\textit{$l_{1}$} + ||${E}^{+}$||'\textit{$l_{0}$}\\
L(${E}^{-}$) = log(|${E}^{-}$|) + ||${E}^{-}$||\textit{$l_{1}$} + ||${E}^{-}$||'\textit{$l_{0}$}\\
\end{center} 
Pour la recherche du meilleur modèle $ \textit{M} \in \mathcal{M} $ , \gls{vog} procède sur trois étapes :
%---------------------------------------------------
%---------------------etape 01----------------------
%---------------------------------------------------
\begin{enumerate}
 \item \textbf{Génération des sous-structures }: Dans cette phase, Les méthodes de détection de communautés et de clustering sont utilisées pour décomposer le graphe en sous-graphes pas forcément disjoints. La méthode de décomposition utilisée dans \gls{vog} est SlashBurn.  
%---------------------------------------------------
%---------------------etape 02----------------------
%---------------------------------------------------
 \item \textbf{Étiquetage des sous-graphes }: L'algorithme cherche pour chaque sous-graphe généré dans l'étape précédente la structure $x \in \Omega$ qui le décrit le mieux, en tolérant un certain seuil d'erreur.
  \begin{enumerate}[label=\alph*]
     \item \textbf{Étiquetage des structures parfaites} : Tout d'abord, le sous-graphe est testé pour une similarité sans erreur par rapport aux structures complètes du vocabulaire :
		%----------------------------------------------------
		%---------------------etape 2.1----------------------
		%----------------------------------------------------
\begin{itemize}[label=$\circ$]
	\item si tous les sommets d'un sous graphe d'ordre n ont un degré égale à n-1, il s'agit alors d'une clique.
	\item si tous les sommets ont un degré de 2 sauf deux sommets ayant le degré 1, le  sous-graphe est une chaine.
	\item si les amplitudes de ses valeurs propres maximales et minimales sont égales, le sous-graphe est un noyau bipartie où les sommet de chaque ensemble du noyau sont identifiés à travers un parcours 
	\newacronym{bfs}{BFS}{Breadth First Search}
	\gls{bfs}\footnote{\gls{bfs} : est un algorithme de parcours en largeur.}	
	 avec coloration des sommets.
	\item  Quant à l'étoile, elle est considérée comme un cas particulier d'un noyau bipartie, il suffit donc que l'un des ensembles soit composé d'un seule sommet.
\end{itemize}     
		%----------------------------------------------------
		%---------------------etape 2.2----------------------
		%----------------------------------------------------
     \item \textbf{Étiquetage des structures approximatives }: Si le sous graphe ne correspond pas à une structure complète, on cherche la structure qui l'approxime le mieux en terme du principe MDL.
     
  \end{enumerate} 
 % \textbf{Remarque} : Pour les structure parfaites, en plus du cout d'encodage de la structure, on rajoute le cout de l'erreur local c'est a dire, L($x^{*}$) + L($\textit{E}^{+}_{x^{*}}$) + L($\textit{E}^{-}_{x^{*}}$) où L($x^{*}$) est la description de la structure, L($\textit{E}^{+}_{x^{*}}$) et L($\textit{E}^{-}_{x^{*}}$) sont les arêtes incorrectement modélisés et les arêtes non  modélisés respectivement dans area($x^{*}$).
  
  
Après avoir représenté le sous graphe sous forme d'une structure, on l'ajoute à l'ensemble des structures candidates $\mathcal{C}$, en l'associant à son cout.

\item \textbf{Assemblage du modèle }: Dans cette dernière étape, une sélection d'un ensemble de structures parmi ceux de $\mathcal{C}$ est réalisée. Des heuristiques de sélections sont utilisées car le nombre de permutations est très grand ce qui implique des calculs exhaustifs. Les heuristiques permettent d'avoir des résultats approximatifs et rapides, parmi les heuristiques utilisées dans \gls{vog} on trouve :
\begin{itemize}
\item PLAIN : Cette heuristique retourne toutes les structures candidates, e.i, \textit{M}= $\mathcal{C}$.
\item TOP-K :  Cette heuristique sélectionne les k meilleurs candidats en fonction de leur gain en bits.
\item GREEDY'N FORGET(GNF) : Parcourt structure par structure dans l'ensemble $\mathcal{C}$ ordonné selon la qualité (gain en bits), ajoute la structure au modèle tant qu'elle n'augmente pas le cout total de la représentation, sinon l'ignore.
%matnsaych kayn réf ta3had la méthode w matnsaych aussi réf ta3 mdl
\end{itemize}  
\end{enumerate}

%Ci-dessous le pseudo algorithme de \gls{vog} :

%\begin{algorithm}
%\caption{Pseudo Algorithme VOG}\label{euclid}
%\begin{algorithmic}[1]
%	\STATE \textbf{Entré :} Graphe G.
%	\STATE \textbf{Étape 1 :}  Génération des sous-graphe en utilisant une méthode de décomposition.  
%	\STATE \textbf{Étape 2 :} Étiquetage des sous graphe en choisissant la structure $x \in \Omega$ qui représente chaque sous- graphe avec le moindre coût.
%	\STATE \textbf{Étape 3 :} Assemblage des sous graphes en utilisant des heuristiques pour sélectionner un sous-ensemble non-redondant à partir des structure candidate de l'étape 2.
%	\STATE \textbf{Sortie :} Modèle \textit{M} et son cout d'encodage.
%\end{algorithmic}
%\end{algorithm}






							%%%VoG Overlapp
			Comme nous l'avons déjà précisé, VOG formule le problème de compression de graphe en tant que problème d'optimisation basé sur la théorie de l'information, l'objectif étant de rechercher les structures  qui minimisent la longueur de description globale du graphe. Un élément clé de VoG est la méthode de décomposition utilisée qui peut donner en sortie des sous-graphes ayant des nœuds et/ou des arêtes en commun et dont VoG\citep{koutra2015summarizing} ne suppose que le premiers cas. 
			
			En partant de ce constat, les auteurs de \citep{liu2015empirical} propose VoG-overlapp, une extension de VoG prenant en compte les chevauchement des structures sous forme d'une étude de l'effet de diverses méthodes de décomposition sur la qualité de la compression, à savoir la méthodes SlushBurn, Louvain, Spectral clustering, Metis et KCBC qui est une nouvelle méthode proposée dans cette article et dont la description et fournie dans l'(Algorithme 1).
			
			
			%%%Algorithme 01
			
			L'idée de base de VoG-overlapp est d'inclure une pénalité pour les chevauchements importants dans la fonction objective ce qui oriente le processus de sélection des structures vers la sortie souhaitée. Elle devient alors:
			\begin{center}
				$min\ L(G,M) = min\ \big\{L(M) + L(E) +\textbf{L(O)}\big\}$
			\end{center}
			
			Le principe de calcul de $L(M)\ et\ L(E)$ demeurent le même avec O, une matrice cumulant le nombre de fois que chacune des arêtes a été couverte par le modèle. Le cout du codage de la matrice des chevauchement est donné par la formule \eqref{eq:Lo}.
			\begin{equation} \label{eq:Lo}
				L(O) = log(|O|)\ +\ ||O||\ l_{1}\ +\ ||O||'\ l_{0}\ +\  \displaystyle{\sum_{o\in\varepsilon(O)}L_{N}(|o|)}
			\end{equation}
			Où :
			\begin{itemize}[label=$\circ$]
				\item $|O|$  est le nombre d'arêtes (distinctes) qui se répète dans le modèles M. 
				\item $||O||$ et $||O||'$ représentent respectivement le nombre des arêtes présentes et manquantes dans O.
				\item $l_{1} = -log (\frac{||O||}{||O||+||O||'})$, de manière analogique $l_{0}$, sont les longueurs des codes de préfixe optimaux pour les arêtes actuelles et manquantes, respectivement.
				\item $\varepsilon(O)$  est l'ensemble des entrées non nulles dans la matrice O.\\
			\end{itemize}
							%%%%TimeCrunch
			Durant la même année, \citep{shah2015timecrunch} ont proposé une autre variation de VoG, TimeCrunch, pour le cas des graphes simples (sans boucles) non orientés dynamiques représentés par un ensemble de graphes associés chacun à timestamp. En d'autres termes, ils considèrent les graphes $\displaystyle{G=\bigcup_{t_{i}}G_{t_{i}}(V,E_{t_{i}})}\ \ 1 \preceq i \preceq t$ où $G_{t_{i}}$ = G a l'instant $t_{i}$.
			
			
			Un nouveau vocabulaire est proposé pour décrire proprement l'évolution des sous-structures dans le temps. En effet, ils partent du même vocabulaire de structures statiques 
			%citer les structures Ω = {st,fc,nc,bc,nb,ch}
			$\Omega =\{ st\ (etoile),\ fc (clique),\ nc (quasi-clique),\ bc (bipartie),\ nb (quasi-bipartie),\ ch (chaine)\}$ 
			dont ils affectent une signature temporelle $\delta \in \Delta$ où: $\Delta$ = \{ o(oneshot), r(ranged), p(periodique), f(flickering), c(constante) \}. 
			
			 Comme les éléments du modèle sont modifiés, son cout est alors aussi modifié pour inclure pour chaque structure $s$ non seulement sa connectivité $c(s)$ correspondant aux arêtes des zones induites par $s$ mais aussi sa présence temporelle $u(s)$ correspondant aux timestamps dans lesquels $s$ apparait dans le graphe G. 
			 
			 $L(M) = L_{N}(|M|+1)\ +\ $log${|M|+|\Phi|-1}\choose{|\Phi|-1}$ $+\ \displaystyle{\sum_{s\in M}(logP(v(s)|M) + L(c(s)) + \textbf{L(u(s))})}$
			 
			 Le cout de l'encodage de la présence temporelle diffèrent selon ses caractéristiques. Nous présenterons dans ce qui suit la formule correspondant à chaque signature.
			 
			 \begin{itemize}[label=$\circ$]
			 \item \textbf{Oneshot}: cette signature décrit les sous-structure qui apparaissent dans un seul timestamp ,i.e $|u(s)|=1$. Donc le cout de l'encodage se réduit aux nombre de bits nécessaires pour sauvegarder le timestamp : $L(u(s)) = log(t)$.
			 \item \textbf{Ranged}: dans ce cas la sous-structure apparait dans tous les graphes se trouvant entre deux timestamps $t_{1}$ et $t_{2}$. Le cout englobe le nombre de timestamps dans lesquels elle apparait ainsi que les identifiants des deux timestamp $t_{debut}$ et $t_{fin}$ : $L(u(s)) = L_{N}(|u(s)|) +log$ ${t}\choose{2}$.
			 \item \textbf{Periodic}: cette catégorie est une extension   de la précédente avec les timestamps qui sont éloigné de plus d'un pas d'où: $L(p) = L(r)$.
			 
			 En effet, la périodicité peut être déduite des marqueurs début et de fin ainsi que du nombre de pas de temps $|u(s)|$, permettant ainsi de reconstruire $u(s)$
			 \item \textbf{Flickering}: ce type décrit les structures qui apparaissent dans $n$ timesteps de manière aléatoire. Le cout doit englober donc le nombre de timesteps ainsi que leurs identifiants d'où: 
			 $L(u(s)) = L_{N}(|u(s)|) +log$ ${t}\choose{|u(s)|}$.
			 \item \textbf{Constant}: dans ce cas la sous-structure apparait dans tout les timesteps et donc elle ne dépend pas du temps d'où L(c)=0.
			 \end{itemize}
			 
			 
			 Nous notons que décrire $u(s)$ est encore un autre problème de sélection de modèle pour lequel nous pouvons tirer parti de MDL. En effet juste comme pour le codage de la connectivité, il peut ne pas être précis avec une signature temporelle donnée. Toutefois, toute approximation entraînera des coûts supplémentaires pour l'encodage de l'erreur qui englobent l'erreur de l'encodage de la connectivité ainsi que l'erreur de l'encodage de la signature temporelle.
							Une dernière variante a été présentée par Liu et al. \citep{liu2018reducing} où ils abordent efficacement trois contraintes principales des  méthodes précédentes: 
				(i) leurs dépendance à la méthode d'extraction de motifs 
				(ii) l'incapacité de certaines à gérer les motifs qui se chevauchent 
				(iii) leur dépendance vis-à-vis de l'ordre dans lequel les structures candidates sont considérées lors de la phase d'assemblage. En effet, pour résoudre le premier problème, ils combinent plusieurs méthodes d'extraction de motifs ce qui améliore la qualité des structures candidates en dépit  du temps d'exécution. Tant dis que pour répondre à la deuxième contrainte ils utilisent la fonction objective proposée dans \citep{liu2015empirical}. Arrivant à la dernière phase de l'algorithme, ils proposent quatre nouvelles heuristique : 
				(1) STEP: choisie les K  meilleurs structures, 
				(2) STEP-P: partitionne le graphe et affecte chaque motif  à la partition ayant un chevauchement maximal de nœuds avec lui. Ces partitions sont parcourues parallèlement pour ne prendre que la meilleure de toutes les structures dans chacune des partitions,
				(3) STEP-PA: amélioration de STEP-P en désignant chaque partition du graphe comme étant active, puis si une partition échoue x fois pour trouver une structure qui réduit le coût MDL, cette partition est déclarée inactive et n'est pas visitée dans les prochaines itérations,
				 (4) K-STEP: combinaison des trois premières heuristiques. Il transforme par la suite chaque motif trouvé en un super-nœud.		
							%				\begin{landscape}
								\begin{table}
									\begin{tabular}{|c|c|c|c|c|c|c|c|c|c|c|c|}
										\hline
										\multirow{2}{*}[-25pt]{Article}  & \multicolumn{4}{c|}{Graphe en entrée} & \multicolumn{2}{c|}{Compression} & \multicolumn{2}{c|}{Structure en sortie}  & \multirow{2}{*}[-25pt]{Graphe de test} & \multirow{2}{*}[-25pt]{Résultat}  \\ \cline{2-11}
				& \rotatebox[origin=c]{90}{ Orienté }  & \rotatebox[origin=c]{90}{ Non orienté } & \rotatebox[origin=c]{90}{ Statique } & \rotatebox[origin=c]{90}{ Dynamique } & \rotatebox[origin=c]{90}{ Avec perte } & \rotatebox[origin=c]{90}{ Sans perte } & \rotatebox[origin=c]{90}{ Succincte } & \rotatebox[origin=c]{90}{ Structurelle }  & & \\ \hline				%%%%%%Fin du header
										
										\hline VoG \citep{koutra2015summarizing} & \xmark & \cmark & \cmark & \xmark & \xmark & \cmark & \cmark & \xmark  & 
	
	\begin{minipage}[t]{0.3\textwidth}
	ASOregon
    \begin{itemize}
    \item 13 milles nœuds
    \item 37 milles liens\\
    
    \end{itemize}
  \end{minipage}									 
			  & 71\%	\\

										\hline VoG-Overlapp \citep{liu2015empirical}& \xmark & \cmark & \cmark & \xmark & \xmark &\cmark & \cmark & \xmark  &
										
	\begin{minipage}[t]{0.3\textwidth}
	ASOregon
    \begin{itemize}
    \item 13 milles nœuds
    \item 37 milles liens\\
    
    \end{itemize}
  \end{minipage}	
										 & 25\%	\\
										\hline TimeCrunch \citep{shah2015timecrunch}& \xmark & \cmark & \cmark & \xmark & \xmark &\cmark & \cmark & \xmark & 

	\begin{minipage}[t]{0.3\textwidth}
	Enron
    \begin{itemize}
    \item 80 milles nœuds
    \item 288 milles liens\\
    
    \end{itemize}
  \end{minipage}											
										&74\%	\\
										\hline CanDenSe \citep{liu2018reducing}&     \xmark & \cmark & \xmark & \cmark & \xmark  &\cmark & \cmark & \cmark &
\begin{minipage}[t]{0.3\textwidth}
	Enron
    \begin{itemize}
    \item 80 milles nœuds
    \item 288 milles liens\\
    
    \end{itemize}
  \end{minipage}											
										 &	 78\%\\
										\hline
									\end{tabular}
									\caption{Synthèse des méthode de compression par extraction de motifs basées sur des méthodes de clustering.}									
									
								\end{table}
								
							\end{landscape}
			\newpage
					 \textbf{Les méthodes basées sur les propriétés de la matrice d'adjacence}
							
							Les graphes peuvent avoir différentes représentations. Chacune des structures de données présente des avantages et des inconvénients en ce qui concerne la quantité de mémoire nécessaire pour stocker les données et la facilité d'accès aux données. Selon les besoins, il est parfois utile de stocker les données dans des structures de données plus grandes, qui nécessitent plus d'espace mais offrent un accès efficace aux données. En se basant sur ce constat plusieurs méthodes ont été  proposées dans la littérature pour compresser la matrice d'adjacence en exploitant les propriétés des graphes réels pour trouver les motifs les plus fréquents dans cette dernière.
							
								\citep{asano2008efficient} ont exploité les propriétés du graphe du web pour présenter une nouvelle méthode de compression, appelée 
	\newacronym{ecwg}{ECWG}{Efficient Compression of web graph }	
	\gls{ecwg}, sans perte permettant d'extraire les motifs à partir de la matrice d'adjacence. Ils proposent un vocabulaire composé de six types de blocs (Motifs): un bloc horizontal de 1, un bloc vertical de 1, un bloc diagonal de 1, un rectangle de 1, un bloc de 1 sous forme de L et le singleton 1. Avant de procéder à l'extraction des motifs, la liste d'adjacence du graphe est partitionnée selon les domaines (ex: esi.dz, usthb.dz, ...). Une nouvelle matrice d'adjacence est donc construite pour chaque hôte(domaine) contenant les liens existants entre ses pages, auxquels les liens inter-hôtes sont concaténés.
					Les blocs B sont détectés par la suite et chacun est représenté par un quadruplets (i, j, type(B), dim(B)) où i,j représentent les coordonnées du premier élément du bloc dans la matrice d'adjacence de l'hôte, type(B) représente le type du bloc et dim(B) représente les dimensions du bloc (omis dans le cas du singleton). 
					\begin{figure}[h]
					\includegraphics[scale=0.5]{ressources/image/inter_intra.png} 
					\caption{Exemple illustrant le principe de fonctionnement \citep{asano2008efficient}}
					\label{interIntra}
				\end{figure}
										Dans une méthode toute récente 
\newacronym{gcupmt}{GCUPMT}{Graph Compression Using Pattern Matching}			
			s'intitulant \gls{gcupmt}, Shah et Rushabh \citep{shah2018graph}  partitionnent les lignes de la matrice d'adjacence en plusieurs blocs ayant la même taille des motifs qui sont dans ce cas sous forme de vecteurs prédéfinies. Les blocs sont comparés avec l'ensemble des motifs ce qui entraîne , en cas de correspondance , le remplacement du bloc par un indicateur du motif précédé par un 1 indiquant que les bits suivants appartiennent à un indicateur de motif. Dans le cas contraire, les données brutes sont stockées directement précédées par un 0.
				
						%							\begin{landscape}
								\begin{table}
									\begin{tabular}{|c|c|c|c|c|c|c|c|c|c|c|c|c|}
										\hline
										\multirow{2}{*}[-25pt]{Article}  & \multicolumn{4}{c|}{Graphe en entrée} & \multicolumn{2}{c|}{Compression} & \multicolumn{2}{c|}{Structure en sortie} & \multicolumn{2}{c|}{Complexité} & \multirow{2}{*}[-25pt]{Graphe de test} & \multirow{2}{*}[-25pt]{Résultat}  \\ \cline{2-11}
				& \rotatebox[origin=c]{90}{ Orienté }  & \rotatebox[origin=c]{90}{ Non orienté } & \rotatebox[origin=c]{90}{ Statique } & \rotatebox[origin=c]{90}{ Dynamique } & \rotatebox[origin=c]{90}{ Avec perte } & \rotatebox[origin=c]{90}{ Sans perte } & \rotatebox[origin=c]{90}{ Succincte } & \rotatebox[origin=c]{90}{ Structurelle } & \rotatebox[origin=c]{90}{ Temporelle} & \rotatebox[origin=c]{90}{ Spaciale} & & \\ \hline				%%%%%%Fin du header
				
				\hline ECWG
 \citep{asano2008efficient}& \cmark & \cmark & \cmark & \xmark & \xmark &\cmark & \cmark & \xmark & & &
										
	\begin{minipage}[t]{0.3\textwidth}
	uk-2002
    \begin{itemize}
    \item 18 millions de nœuds
    \item 298 millions liens\\
    
    \end{itemize}
  \end{minipage}	
										 & 76.1\%	\\
										
										\hline GCUPMT \citep{shah2018graph} & \cmark & \cmark & \cmark & \xmark & \xmark &\cmark & \cmark & \xmark & & & 
										
	\begin{minipage}[t]{0.3\textwidth}
	graphes avec :
    \begin{itemize}
    \item 8192 nœuds
    \\
    
    \end{itemize}
  \end{minipage}								 
			  & 70\%	\\

										\hline
									\end{tabular}
									\caption{Synthèse des méthode de compression par extraction de motifs basées vocabulaire exploitant les propriétés de la matrice d'adjacence.}									
									
								\end{table}
								
							\end{landscape}
					
				
				%%%%%%%%%%%%%%%%%%%%%%%%%%%%%%%%
				\subsection{Compression basée sur l'agrégation des motifs}
				
					Les méthodes de compression par extraction de motifs basées sur l'agrégation sont des méthodes   qui agrègent plusieurs nœuds ou liens d'un motif en un seul nœud, appelé super-nœud. Le graphe en sortie, dit super-graphe, devient dès lors plus simple et moins complexe offrant ainsi une aisance et une facilité de traitement, d'exploration et de visualisation. Nous présenterons dans ce qui suit les deux sous-classes de cette classe qui se distinguent selon que l'agrégation concerne les nœuds ou les liens.
					
					\begin{enumerate}[label=\alph*.]
				
					
					\item \subsubsection{Les méthodes de compression basées l'agrégation de nœuds}
					
					Les techniques de compression basées sur l'agrégation des nœuds des motifs sont des méthodes qui ont existé depuis plusieurs décennies offrant plusieurs avantages. 
					Elles visent à résumer le graphe initial en agrégeant les nœuds des motifs découvert dans le but de diminuer le nombre de nœuds existants  et d'offrir une meilleure visibilité et analyse du graphe. 
						
						Une première méthode de cette classe s'intitule Subdue \citep{ketkar2005subdue}. Elle effectue une recherche \textit{Branch\&Bound} qui commence à partir des sous-structures composées de tous les sommets avec des étiquettes uniques. Les sous-structures sont prolongées de toutes les manières possibles par un sommet et une arête ou par une arête afin de générer des sous-structures candidates. Subdue conserve les instances des sous-structures et utilise l'isomorphisme de graphe pour déterminer les instances de la sous-structure candidate. Les sous-structures sont ensuite évaluées en fonction de leur compression de la longueur de description (DL) du jeu de données. Cette procédure se répète jusqu'à ce que toutes les sous-structures soient prises en compte ou que les contraintes imposées par l'utilisateur ne soient plus vérifiées. A la fin de la procédure, Subdue indique les meilleures sous-structures de compression.
				Le système Subdue fournit également la possibilité d'utiliser la meilleure sous-structure trouvée lors d'une étape de découverte pour compresser le graphe d'entrée en remplaçant les instances de la sous-structure par un seul sommet et en effectuant le processus de découverte sur le  compressé. Cette fonctionnalité génère une description hiérarchique du jeu de données du graphe à différents niveaux d'abstraction en termes de sous-structures découvertes.\\
							
	Dans \citep{rossi2018graphzip}, les auteurs partent de l'observation que les graphes réels sont formés souvent de nombreuses cliques de grande taille. En utilisant ceci comme base, GraphZip décompose le graphe en un ensemble de grandes cliques, qui est ensuite utilisé pour compresser et représenter le graphe de manière succincte. 
					%							\begin{landscape}
								\begin{table}
									\begin{tabular}{|c|c|c|c|c|c|c|c|c|c|c|c|c|}
										\hline
										\multirow{2}{*}[-25pt]{Article}  & \multicolumn{4}{c|}{Graphe en entrée} & \multicolumn{2}{c|}{Compression} & \multicolumn{2}{c|}{Structure en sortie}  & \multirow{2}{*}[-25pt]{Graphe de test} & \multirow{2}{*}[-25pt]{Résultat}  \\ \cline{2-11}
				& \rotatebox[origin=c]{90}{ Orienté }  & \rotatebox[origin=c]{90}{ Non orienté } & \rotatebox[origin=c]{90}{ Statique } & \rotatebox[origin=c]{90}{ Dynamique } & \rotatebox[origin=c]{90}{ Avec perte } & \rotatebox[origin=c]{90}{ Sans perte } & \rotatebox[origin=c]{90}{ Succincte } & \rotatebox[origin=c]{90}{ Structurelle }  & & \\ \hline				%%%%%%Fin du header
				
				\hline Subdue
 \citep{ketkar2005subdue}& \xmark & \cmark & \cmark & \xmark & \xmark & \cmark & \xmark & \cmark &		
	\begin{minipage}[t]{0.3\textwidth}
	Composante chimique :
    \begin{itemize}
    \item 21 étiquettes
    \item 422 transactions\\
    
    \end{itemize}
  \end{minipage}	
										 & 16\%	\\
										
										\hline GraphZip \citep{rossi2018graphzip} & \xmark & \cmark & \cmark & \xmark & \xmark & \cmark & \xmark & \cmark & 
				Web-Google						
								 
			  & 19\%	\\

										\hline
									\end{tabular}
									\caption{Synthèse des méthodes de compression par extraction de motifs basées agrégation de nœuds.}									
									
								\end{table}
								
							\end{landscape}
					
					\item \subsubsection{Les méthodes de compression basées sur l'agrégation de liens}
						Les méthodes de compression par extraction de motifs basées sur l'agrégation de liens sont parmi les méthode les plus populaires. Leur objectif est de produire un graphe compressé à partir du graphe initial en remplaçant les liens denses du graphe par un nouveau super-nœud. Elles se divisent selon le principe en deux grandes classes: celles utilisant les règles de grammaire et celles utilisant des heuristiques de clustering. Nous détaillerons dans ce qui suit ces deux classes.
						
						\begin{enumerate}[label*=\arabic*.]
							\item \textbf{Les méthodes de compression basées sur les règles de grammaire}
							La classe des méthodes de compression basées sur les règles de grammaire est une généralisation d'une méthode de compression des dictionnaires s'intitulant Re-pair. Son principe de base consiste en la  recherche, à chaque itération, de la paire de symboles la plus fréquente dans une séquence de caractères et de la remplacer par un nouveau symbole, jusqu'à ce qu'il ne soit plus commode de les remplacer. Nous notons que dans ce cas le motif est sous forme de deux arêtes ayant un sommet en commun, nommé \textit{digraph}.
								
										Une première méthode suivant ce principe a été proposée dans \citep{claude2010fast} baptisée \textit{Approximate Re-pair}. Dans cette méthode un graphe G=(V,E) est représenté sous forme d'une sequence de caractères T: \begin{center}
		T=T(G)= $\overline{v_{1}}\ v_{1,1}\ ...\ v_{1,a}\ \overline{v_{2}}\ v_{2,1}\ ...\ v_{2,a_{2}} ... \overline{v_{n}}\ v_{n,1}\ ...\ v_{n,a_{n}}\ $						
\end{center}	
où $\overline{v_{i}}$ représente l'identificateur du sommet $v_{i}$. Elle procède en trois étapes essentielles expliquer dans l'algorithme \textbf{\ref{alg:re_pair}} .
	\begin{algorithm}
		\caption{Approximate Re-pair}
		\label{alg:re_pair}
		\begin{algorithmic}[1]
			\STATE \textbf{Calcule des fréquences:} T est parcourue séquentiellement et chaque pair $t_{i}t_{i+1}$ est ajoutée à un tableau de hachage H avec leur nombre d'occurrences. 
			\STATE \textbf{Recherche des k meilleurs paires:}  H est parcourue et les k paires les plus fréquentes sont retenues, en utilisant k pointeurs vers les cellules de H.
			\STATE \textbf{Le remplacement simultané:} les k paires identifiées dans l'étape précédente sont simultanément remplacées par un nouveau identifiant et une règle de production est ajoutée.
		\end{algorithmic}
	\end{algorithm}
	Lorsqu'il n'y a plus de paires à remplacer, Approximate Re-pair s'arrête donnant comme résultat un représentation compacte C de la chaine T. Pour finaliser le processus, tous les indicateurs de nœuds $\overline{v_{i}}$ seront supprimés de C. De plus, l'algorithme crée une table qui contiendra des pointeurs vers le début de la liste d'adjacence de chaque nœud dans C. Grâce à cette table l'algorithme pourra répondre aux requêtes de recherche de successeurs en un temps optimal.
								 %%%%exemple ???
								
	Dans un travail ultérieure \citep{claude2010extended} , les même auteurs s'intéressent aux requêtes de recherche des nœuds prédécesseurs et successeurs à partir du graphe compressé de \textit{Approximate Re-pair} directement. Il proposent alors de combiner leurs méthode avec une représentation basé sur les relations binaires de \citep{barbay2006adaptive}. En effet, ce dernier consiste à représenter les listes d'adjacence à l'aide d'une représentation séquentielle permettant de rechercher les occurrences d'un symbole puis de rechercher les voisins inverses à l'aide de cette primitive. 
								
Claude et Ladra \citep{claude2011practical} partagaient les mêmes préoccupations des auteurs de la méthode précédente et ont proposé comme solution de combiner la méthode Re-pair avec la représentation k2-tree. Ils obtiennent alors une compression de 2,27 (pbe) sur le graphe UK2002, tout en conservant la possibilité d'interroger les voisins entrants et sortants \citep{maneth2015survey}.
							
								
	Une dernière méthode de cette classe s'instituant gRepair a été proposé dans \citep{maneth2018grammar}. Ce nouveau algorithme de compression détecte de manière récursive des sous-structures répétées et les représente via des règles de grammaire.  Des requêtes spécifiques telles que l'accessibilité entre deux nœuds ou des requêtes de chemin normal peuvent ainsi être évaluées en temps linéaire (ou en temps quadratiques, respectivement), sur la grammaire, permettant ainsi des accélérations proportionnelles au taux de compression. la figure xxx presentent le resultat de cette methode sur un exemple. 
								%%%%exemple ???
							%															\begin{landscape}
								\begin{table}
									\begin{tabular}{|C{3cm}|c|c|c|c|c|c|c|c|c|c|c|c|}
										\hline
										\multirow{2}{*}[-25pt]{Article}  & \multicolumn{4}{c|}{Graphe en entrée} & \multicolumn{2}{c|}{Compression} & \multicolumn{2}{c|}{Structure en sortie}  & \multirow{2}{*}[-25pt]{Graphe de test} & \multirow{2}{*}[-25pt]{Résultat (bits/lien)}  \\ \cline{2-11}
				& \rotatebox[origin=c]{90}{ Orienté }  & \rotatebox[origin=c]{90}{ Non orienté } & \rotatebox[origin=c]{90}{ Statique } & \rotatebox[origin=c]{90}{ Dynamique } & \rotatebox[origin=c]{90}{ Avec perte } & \rotatebox[origin=c]{90}{ Sans perte } & \rotatebox[origin=c]{90}{ Succincte } & \rotatebox[origin=c]{90}{ Structurelle } & & \\ \hline				%%%%%%Fin du header
				
				\hline Approximate Re-pair
   \citep{claude2010fast}& \cmark & \xmark & \cmark & \xmark & \xmark &  \cmark & \cmark & \xmark	 &		
	\begin{minipage}[t]{0.3\textwidth}
	uk-2002 :
    \begin{itemize}
    \item 18 millions de nœuds
    \item 298 millions de liens\\
    
    \end{itemize}
  \end{minipage}	
										 & 4.23	\\

										\hline Approximate Re-pair \citep{claude2010extended} & \cmark & \xmark & \cmark & \xmark & \xmark & \cmark & \cmark & \xmark  & 
										\begin{minipage}[t]{0.3\textwidth}
	uk-2002 :
    \begin{itemize}
    \item 18 millions de nœuds
    \item 298 millions de liens\\
    
    \end{itemize}
  \end{minipage}	 & 3.98 \\ 
  			\hline gRe-pair \citep{maneth2018grammar} & \cmark & \cmark & \cmark & \xmark & \xmark & \cmark & \cmark & \xmark &
  				\begin{minipage}[t]{0.3\textwidth}
  			NotreDame :
    \begin{itemize}
    \item 325 milles de nœuds
    \item 1M de liens\\
  		\end{itemize}
  \end{minipage}		
  			& 4.84 \\\hline
									\end{tabular}
									\caption{Synthèse des méthodes de compression par extraction de motifs basées agrégation de liens en utilisant les règles de grammaire.}									
									
								\end{table}
								
							\end{landscape}
							\item \textbf{Les méthodes de compression basées sur des heuristiques de clustering} 				
							
							Les méthodes de compression appartenant à cette classe sont des méthodes basées sur la recherche des sous-graphes denses (ayant des nœuds fortement connectés). Elles sont destinées principalement aux graphes du Web et les graphes des réseaux sociaux dans le but de faciliter leur exploration et analyse.
								
									%-----------------Aggregation des liens ----------%
				\citep{buehrer2008scalable} ont exploité l'existence de plusieurs ensembles de pages web qui ont les même liens sortants. S'intitulant  VNM pour Virtual Node Miner, leurs approche est basée sur la réduction du nombre de liens en créant des nouveaux sommets virtuels qui sont ajoutés au graphe. Soit G = (V,E) un graphe orienté, l'algorithme proposé se compose de deux phases essentielles :
				\begin{enumerate}
				
				
					\item \textbf{Phase de Clustering :}
					
					Le but de cette première étape est de contourner la tâche presque impossible d'extraction simultanée de centaines de millions de points de données en groupant d'abord les sommets similaires dans le graphes dans des clusters. 
				Pour cela k fonctions de hachage indépendantes sont utilisées pour obtenir une matrice de taille V * k. Par la suite, les lignes de la matrice  sont triées lexicographiquement
				%Ce tri est assez rapide car il ne nécessite que O (2V log V) en mémoire à la fois, ce qui tient dans la RAM (ou nous le faisons en minant le graphique en morceaux). 
				et elle est parcourue colonne par colonne en regroupant les lignes ayant la même valeur. Lorsque le nombre total de lignes chute au-dessous d'un seuil ou que le bord de la matrice de hachage est atteint, les identifiants des sommets associés aux lignes sont renvoyés au processus d'extraction (Phase 02). 
					
					\item \textbf{Phase d'Extraction de Motifs :}				Le but de cette étape est de localiser des sous-ensembles communs de liens sortants dans les sommets donnés. 
				Ainsi les ensembles plus grands et fréquents présentent un intérêt, car ils peuvent représenter des motifs plus pertinents et une meilleure compression. En effet, les performances de compression d'un motif sont calculés en fonction de sa fréquence dans la liste d'adjacence, et de sa taille qui est le nombre de liens qu'il contient \eqref{eqcompperf}.
				%%% Formule
				\begin{equation}
				Compression(P)=(P.fréquence-1)(P.taille-1)-1
				\label{eqcompperf}
				\end{equation}
				Afin d'extraire ces motifs, VNM utilise une heuristique gloutonne. Cette heuristique procède comme suit :
				\begin{enumerate}
				\item Extraire un histogramme des identifiants de liaison sortante à partir de la liste d'adjacence des sommets données.
				\item Les listes sont réorganisées dans l'ordre décroissant des fréquences des liens sortants en éliminant ceux qui apparaissent une seul fois uniquement.
				\item Chaque lien sortant est ajouté à un arbre de préfixes avec l'ensemble trié de ces extrémités initiales selon leurs identifiants. 
				\item L'arbre est par la suite parcouru afin d'identifier les motifs qui maximisent la formule de performance de la compression \ref{eqcompperf}. Ces motifs sont ensuite convertis en nœuds virtuels et les identificateurs de sommets de leurs listes sont remplacés par les ids des nœuds virtuels dans la liste d'adjacence.
				\end{enumerate}
				 
				\end{enumerate}
				
				L'algorithme est appliqué jusqu'à ce que la réduction n'apporte pas un gain significatif. La figure \ref{VNM_exemple} illustre le principe de fonctionnement de cette méthode sur un exemple.
				
				
			%%%%		Inclure un exemple 
		
		
			\begin{figure}[!h]
			\includegraphics[scale=0.3,center]{./ressources/image/VNM_exemple.png}
			\caption[Exemple d'exécution de VNM.]{Exemple d'exécution de VNM.}
			\label{VNM_exemple}
	\end{figure}
								\newpage	
	Une variante de \gls{vnm} , 
	\newacronym{dsm}{DSM}{Dense SubGraph Mining}	
	\gls{dsm}, a été proposée par Hernandez et Navarro \citep{hernandez2014compressed}. Comme première contribution, ils augmentent les types de structures découvertes dans la phase de clustering pour englober aussi: les cliques, les bi-cliques. L'extraction de motifs cette fois-ci n'est pas basée sur un parcours des feuilles vers la racine mais l'inverse où  l'ensemble des sommets finaux des liens du motifs est constitué des étiquettes des nœuds de l'arbre inclues dans le chemin de la racine vers la feuille et  les sommets initiaux sont la liste des sommets inclus dans le nœud feuille. 
				%%Noublie pas revient vers la source 
				Leur deuxième contribution consiste en une hybridation dans le but de représenter le graphe en sortie à l'aide de structures compactes. Une première approche proposée est d'utiliser les  k2-trees \citep{brisaboa2009k}  qui donnent la représentation la plus compacte.  
				La deuxième hybridation consiste en une nouvelle structure proposée par les auteurs.\\
				
				
				
				\begin{figure}[!h]
					\includegraphics[scale=0.23]{ressources/image/VNM2_exemple.png} 
					\caption{Exemple d'exécution de DSM}
					\label{SDM}
				\end{figure}
					
						
						\end{enumerate}	
					
					\end{enumerate}
						
						\subsection{Synthèse des méthodes de compression par extraction de motifs }

	Durant les sections précédentes nous avons expliqué de manière générale les fondements de base de chaque sous-classe de la classe des méthodes de compression par extraction de motifs ainsi que le principe de fonctionnement de leurs méthodes. 
	
	Nous regroupant dans le tableau \ref{enfin} les différentes caractéristiques des méthodes de ces sous-classes. Nous observons qu'elles sont toutes sans perte et destinées aux graphes statiques, à l'exception de la classe des méthodes de compression Basées vocabulaire en utilisant les méthodes de clustering qui contiennent des méthodes de compression supportant les graphes dynamiques. On remarque aussi que la majorité des méthodes de compression par extraction de motifs fournissent en sortie une représentation succincte. Le tableau \ref{enfin} contient aussi un résumé sur les résultats de l'application de ces méthodes sur des graphes réels.
			

\begin{landscape}
								\begin{table}
								%\renewcommand{\arraystretch}{2}
									\begin{tabular}{|c|C{5.5cm}|c|c|c|c|c|c|c|c|C{2.5cm}|c|}
										\hline
										\multirow{2}{*}[-25pt]{Classe} & \multirow{2}{4cm}[-25pt]{Méthode}  & \multicolumn{4}{c|}{Graphe en entrée} & \multicolumn{2}{c|}{Compression} & \multicolumn{2}{c|}{Structure en sortie}  & \multirow{2}{*}[-25pt]{Graphe de test} & \multirow{2}{*}[-25pt]{Résultat}  \\ \cline{3-10}
				& & \rotatebox[origin=c]{90}{ Orienté }  & \rotatebox[origin=c]{90}{ Non orienté } & \rotatebox[origin=c]{90}{ Statique } & \rotatebox[origin=c]{90}{ Dynamique } & \rotatebox[origin=c]{90}{ Avec perte } & \rotatebox[origin=c]{90}{ Sans perte } & \rotatebox[origin=c]{90}{ Succincte } & \rotatebox[origin=c]{90}{ Structurelle }  & & \\ \hline	 \hline			%%%%%%Fin du header
										
										 \multirow{4}{4cm}{Basée vocabulaire en utilisant  les méthodes de clustering}& VoG \citep{koutra2015summarizing} & \xmark & \cmark & \cmark & \xmark & \xmark & \cmark & \cmark & \xmark  & 
	
	
	Enron									 
			  & 75\%	\\ \cline{2-12}

										 & VoG-Overlapp \citep{liu2015empirical}& \xmark & \cmark & \cmark & \xmark & \xmark &\cmark & \cmark & \xmark  & Enron							 & 75\%	\\  \cline{2-12}
										 & TimeCrunch \citep{shah2015timecrunch}& \xmark & \xmark & \cmark & \cmark & \xmark &\cmark & \cmark & \xmark & 

	
	Enron			&74\%	\\  \cline{2-12}
										 & CanDenSe \citep{liu2018reducing}&     \xmark & \cmark & \xmark & \cmark & \xmark  &\cmark & \cmark & \cmark & Enron									
										 &	 78\%\\ \hline \hline
										 
										 \multirow{2}{4cm}{\footnotesize{Basée vocabulaire en utilisant  les propriétés de la matrice d'adjacence}}&  ECWG
 \citep{asano2008efficient}& \cmark & \cmark & \cmark & \xmark & \xmark &\cmark & \cmark & \xmark  &
	uk-2002	 & 76.1\%	\\  \cline{2-12}
	
	& GCUPMT \citep{shah2018graph} & \cmark & \cmark & \cmark & \xmark & \xmark &\cmark & \cmark & \xmark & 
	graphes 8192 nœuds
							 
			  & 70\%	\\ \hline \hline
			  
			\multirow{2}{4cm}{\small{Basée Agrégation de nœuds des motifs}} & Subdue
 \citep{ketkar2005subdue}& \xmark & \cmark & \cmark & \xmark & \xmark & \cmark & \xmark & \cmark &		
	Graphe des composantes chimiques
										 & 16\%	\\ \cline{2-12}
										 
										 & GraphZip \citep{rossi2018graphzip} & \xmark & \cmark & \cmark & \xmark & \xmark & \cmark & \xmark & \cmark & 
				Web-Google						
								 
			  & 19\%	\\ \hline \hline
			  
			  \multirow{3}{4cm}{Basée Agrégation de liens des motifs en utilisant les règles de grammaire} & Approximate Re-pair
   \citep{claude2010fast}& \cmark & \xmark & \cmark & \xmark & \xmark &  \cmark & \xmark & \cmark	 &		
	uk-2002 & 4.23	\\ \cline{2-12}
	 &  Approximate Re-pair \citep{claude2010extended} & \cmark & \xmark & \cmark & \xmark & \xmark & \cmark & \xmark & \cmark  & 
	uk-2002	 & 3.98 bpe \\	\cline{2-12}
	
	& gRe-pair \citep{maneth2018grammar} & \cmark & \cmark & \cmark & \xmark & \xmark & \cmark & \xmark & \cmark &
  			
  			NotreDame
  			& 4.84 bpe \\ \hline \hline
  			  \multirow{2}{4cm}{\small{Basée Agrégation de liens des motifs et les méthodes de clustering}}& VNM
 \citep{buehrer2008scalable}& \cmark & \xmark & \cmark & \xmark & \xmark & \cmark & \xmark & \cmark &		
	uk-2002 	 & 1.95 bpe \\ \cline{2-12}
	& DSM \citep{hernandez2014compressed} & \cmark & \xmark & \cmark & \xmark &  \xmark & \cmark & \cmark & \cmark  & 
				
	uk-2002 					 
			  & 1.53 bpe	\\
									\hline	 
									\end{tabular}
									\caption{Synthèse des méthodes de compression par extraction de motifs.}
									\label{enfin}									
									
								\end{table}
								
							\end{landscape}
							
							

							
							
								
		\section{Bilan général}
	
De nos jours, les graphes sont omniprésents. Cependant, leur taille présente un obstacle presque insurmontable à la compréhension du caractère essentiel des données. D'où la nécessité de la compression qui permet de réduire la taille des graphes tout en gardant le caractère utile de l'information incluse dans ces derniers.

Nous avons étudié dans ce chapitre différentes méthodes de compression de graphes dans le but d'établir une classification des méthodes basée sur deux approches : l'extraction de motifs et les arbres k2-trees. Nous nous sommes appuyer pour cela sur le principe de fonctionnement de chacune d'elles. Nous  proposons six classes de méthodes :

\begin{enumerate}[label=\roman*]
\item  Les méthodes de compression par les k2-trees
	\item  Les méthodes de compression par extraction de motifs basées vocabulaires en utilisant des méthodes de clustering.
	\item  Les méthodes de compression par extraction de motifs basées vocabulaires exploitant les propriétés de la matrice d'adjacence.
	\item  Les méthodes de compression par extraction de motifs basées agrégation de nœuds.
	\item  Les méthodes de compression par extraction de motifs basées agrégation de liens en utilisant des règles de grammaire.
	\item  Les méthodes de compression par extraction de motifs basées agrégation de liens en utilisant des heuristiques de clustering.
\end{enumerate}
	
Nous avons présenté le principe de fonctionnement des méthodes relatif à chaque classe tout en comparant leurs caractéristiques. 
 Ces classes diffèrent les unes des autres dans leurs fondements théoriques, le type de compression considéré, la complexité des algorithmes, les objectifs et les domaines d'application. 
 Dans le tableau \ref{comgen} , nous allons essayer de synthétiser les principales différences et similitudes entre les deux approches, compression par extraction de motifs et compression par les arbres k2-trees, que nous avons pu constater à travers notre recherche bibliographique. Ils ont un fort impact sur le choix de la méthode de compression du moment qu'ils ont une influence directe sur les performances.
 
\begin{table}[H]
\begin{tabular}{|c|p{6cm}|p{4cm}|}

\hline & \begin{center}
\textbf{$k^2$-trees}
\end{center}     &  \begin{center} \textbf{Extraction de motifs} \end{center}  \\
										
										
\hline Type de compression & toujours sans perte & toujours sans perte \\
\hline Structure en sortie & Toujours succincte & Peut être succincte où structurelle où les deux en même temps\\

\hline technique utilisée & exploitation de la matrice d'adjacence du graphe & exploitation des motifs fréquents dans le graphe\\

\hline Dépendance & Dépend du paramètre k & Dépend selon la méthode de l'algorithme de clustering ou du vocabulaire de motifs utilisé  \\

\hline Objectif & 
\begin{minipage}[t]{0.35\textwidth}
  			Compression,\\
  			Réduire l'espace de stockage et le temps de parcours\\
  \end{minipage}
  &
  \begin{minipage}[t]{0.25\textwidth}
  			- Compression,\\
  			- Réduire l'espace de  stockage et le temps de parcours,\\
  			- Extraire les informations pertinentes, \\
  			- Visualisation \\
  \end{minipage}
  \\
  \hline Domaine d'application & tous les domaines & tous les domaines \\
  \hline Type de graphes supporté & 
 \begin{minipage}[t]{0.25\textwidth}
  			- Statique orienté,\\
  			- Attribué, étiqueté\\
  			- Dynamique, \\
  \end{minipage}  
  &  \begin{minipage}[t]{0.25\textwidth}
  			- Statique (orienté et non orienté),\\
  			- étiqueté,\\
  			- Dynamique, \\
  \end{minipage}  
  \\ \hline
\end{tabular}
									\caption{Comparaison entre les méthodes basées sur$k^2$-trees et basées sur l'extraction de motifs.}									\label{comgen}
									
								\end{table}
								
				
	

	
	
	


	
	


\part{Contribution}

%

Comme on vient de voir dans les chapitres précédents, de nombreux phénomènes dans des contextes très variés peuvent être représentés par des graphes. Citons en particuliers les réseaux sociaux qui nous attirent vu leur importance et leur fort impact sur la vie réelle. En effet, trois quarts des personnes connectées à Internet utilisent les réseaux sociaux. Ils rassemblent aujourd'hui 3,169 milliards d'utilisateurs actifs, voir 40\% de la population sur terre, avec 11 personnes s'inscrivant chaque seconde sur Facebook, Twitter et autres
\footnote{https://bfmbusiness.bfmtv.com/hightech/la-moitie-de-la-population-mondiale-est-connectee-a-internet-1361933.html}. De ce fait, deux constats s'imposent: le premier est l'incapacité des graphes statiques à décrire certaines situations de la vie réelle et leur évolution dans le temps, le deuxième est la montée en flèche de la quantité et l'hétérogénéité des informations modélisées par ces réseaux et qui rendent presque impossible tout traitement. D'où le besoin croissant de méthodes de compression de graphes dynamiques. 

Nous nous sommes aussi intéressées dans ce travail à un autre enjeu important qui est la recherche de motifs et particulièrement les cliques et les noyaux bipartis qui sont fréquents dans les réseaux faisant abstraction à des situations réelles importantes dans le processus de prise de décision. Nous citons par exemple, les attaquants de réseaux de botnet formant un noyau biparti avec leurs victimes pendant la durée d'une attaque, les membres de la famille se liant comme une clique au cours d'une période difficile, ou les collaborations de recherche s'étant créées et qui disparaissent au cours des années.

%Quoiqu'il existe de nombreuses techniques de compression efficaces adaptées au contexte statique, rares sont ceux qui proposent une contrepartie dynamique, ce qui nous amène à s'intéresser particulièrement à ce type de graphes toujours en cours d'évolution. 

Nous allons dans cette partie proposer une nouvelle méthode DDSM basée sur deux approches existantes. Notre but principal est de développer une technique de compression compétitive pour les graphes des réseaux d'interactions dans un contexte dynamique en se basant sur l'extraction de motifs pour filtrer les informations les plus importantes et les arbres $k^2$-trees pour aboutir à une compression sans perte. Pour accomplir ce travail nous nous sommes appuyées sur les deux techniques suivantes:
\begin{itemize}
\item Exploiter la structure de données proposée dans \gls{dsm} \citep{hernandez2014compressed} qui permet de représenter les sous-graphes denses comme les cliques, les quasi-cliques, noyaux bipartis de manière concise. Nous avons opté pour cette structure car  elle donne de bons résultats en matière d'espace et de temps de requêtes.  
\item Adopter les signatures temporelles utilisées dans TimeCrunch \citep{shah2015timecrunch} pour décrire le comportement des sous-graphes dans le temps. Nous avons utilisé ces signatures car elles englobent tous les types de comportements temporels d'un motif dans un graphe dynamique.
\end{itemize}	
%Nous allons dans cette partie présenter notre méthode théoriquement : principe général, structure utilisée et un pseudo code.	
		
		\subsection{Formulation du problème}
		Dans cette section, nous allons définir le problème de base auquel notre méthode convient tout en définissant le cadre dans lequel elle peut être appliquée. 
		
		Nous considérons les graphes orientés dynamiques $\displaystyle{G=\bigcup_{t_{i}}G_{t_{i}}(V,E_{t_{i}})}\ \ 1 \preceq i \preceq t$ où $G_{t_{i}}$ = G à l'instant $t_{i}$ et un ensemble de signatures temporelles. En d'autre termes, nous considérons des captures du graphe à des instants différents $t_{i}$ et un ensemble de descripteurs de comportement temporel des sous-graphes des différents $G_{t_{i}}$. Le problème peut ainsi être formulé:
		
			\textit{\textbf{Problème:}
		Trouver, à partir d'un graphe dynamique G et un lexique de signatures temporelles, la plus petite description du graphe initial en termes de ses sous structures les plus denses et leur comportement temporel, en offrant la possibilité de manipuler le graphe, d'extraire les voisins directs d'un nœud à un instant donné et d'extraire les sous-graphes au besoin. }

		\subsection{Principe général}
			Notre méthode s'intitule DDSM pour \textit{Dynamic Dense Substructure Mining}. Elle représente une généralisation du travail de Hernandez et Navarro \citep{hernandez2014compressed} pour le cas des graphes dynamiques qui sont de nos jours omniprésents.
			
			%Pour mieux expliquer notre méthode nous allons tout d'abord commencer par décrire les structures que nous allons utiliser. Nous enchainons par la suite  avec le pseudo algorithme détaillant les étapes essentielles de notre méthode et nous conclurons en expliquant comment les algorithmes proposés dans \citep{hernandez2014compressed} de manipulation des graphes peuvent être généralisés dans notre cas.
			%\subsection{Description conceptuelle}
			
			La codification du graphe en sortie doit respecter les contraintes du problème tout en réduisant un maximum d'espace mémoire. Pour cela, nous proposons d'étendre la codification proposée dans \citep{hernandez2014compressed} pour le cas des graphes dynamiques en rajoutant une information temporelle. Une fois les sous-structures identifiées, Hernandez et al. \citep{hernandez2014compressed} proposent de représenter chacune d'elles avec trois composantes: la première contenant les sommets ayant uniquement des arêtes sortantes, la deuxième contenant les sommets ayant des arêtes entrantes et sortantes et la troisième contenant les sommets ayant uniquement des arêtes entrantes. Pour pouvoir identifier les différentes composantes, ils associent un vecteur binaire à cette représentation marquant par un 1 le début de chacune des trois composantes (voir figure \ref{SDM}). Notre première contribution consiste en l'extension de cette structure. En effet, nous suggérons de représenter chaque sous structure avec non pas trois composantes mais quarte composantes. Les trois premières étant les même que dans \citep{hernandez2014compressed}, la quatrième représente l'information temporelle. Pour plus de clarté, nous illustrons la représentation dans figure \ref{ddsmCode}.
	
		\begin{figure}[H]
\centering
	\includegraphics[scale=0.45]{./ressources/image/dsmRepresentation.jpg}
	\caption[Codage des structures dans DDSM]{Codage des structures dans DDSM}
	\label{ddsmCode}
\end{figure}			
			
			
L'information temporelle dans l'encodage peut avoir cinq formats (voir section \ref{par:TimeCrunch}) dont nous résumons la représentation proposée pour chacune dans le tableau \ref{tab:signtmp}.  
			
			\begin{table}[h]
			
			\begin{center}
			\begin{tabular}{|r|l|}
			\hline Signature temporelle & Représentation	
			\\\hline constante & 0	
			
			\\\hline OneShot & 1 $t_{1}$	
			
			\\\hline ranged & 2	$t_{1}\ t_{2}$	
			
			\\\hline periodic & 3  $T$	
			
			\\\hline flikering & 4 $t_{1}\ t_{2}\ ...\ t_{n}$	
			
			\\\hline			
			\end{tabular}
			\end{center}
			
			 \caption{\small {Les types de signatures temporelles et leurs représentations, $t_{i}$	 représentent les timestamps et T représente la période.}}
				\label{tab:signtmp}			
			\end{table}
			
			Nous représentons G donc comme un ensemble de sous-graphes temporels denses. Cependant, pour obtenir une compression sans perte, nous devons aussi garder l'erreur modélisant l'ensemble d'arêtes restantes dans chaque $G_{t_{i}}$. Nous proposons pour cela d'utiliser une des structures dynamiques des k2-trees qui nous permettra de garder trace de cette dernière dans un format compacte. Nous ferons appel à cette étape au moteur $k^2$-GraCE.
			
			%\subsection{Notre méthode : DDSM}
			 Nous visons à travers la méthode que nous proposons d'exprimer le graphe en entrée avec ses sous-structures les plus denses et leur comportement temporel réduisant ainsi sa taille et offrant la possibilité d'effectuer les traitements dans un temps meilleur. Nous avons structuré notre algorithme sous forme de trois (03) étapes essentielles, chacune servant d'entrée pour l'étape suivante.
			 
			 En premier lieu, nous appliquons la découverte des sous-graphes les plus denses. Pour effectuer cela de manière efficace, nous suggérons d'utiliser la technique du MinHashing du moteur P-GraCE parallèlement sur chaque capture $G_{t_{i}}$ de G. 
			 
			 
			 Dans une deuxième phase, nous effectuons une comparaison entre les sous-structures. Nous utilisons pour cela le principe du MinHashing encore une fois pour grouper les sous-structures de différents timestamps ayant un nombre élevé de nœuds en commun. Ayant associé à chaque sous-structure ses différents timestamps, ces derniers seront à leur tour compressés en utilisant les descripteurs de comportement temporel précédemment décrits. L'algorithme \ref{alg:getTemporalSignature} représente les étapes de compression de l'ensemble des timestamps d'une structure.\\
			 
			 
			 	\begin{algorithm}[H]
					
					\caption{getTemporalSignature}
					\label{alg:getTemporalSignature}
					\textbf{Entrée :}
						\begin{itemize}[label=$\bullet$]
							\item timeseries[] : un tableau de timestamps
							\item size : la taille du tableau timeseries
						\end{itemize}
					\textbf{Sortie :} la signature temporelle\\							\noindent\rule{\textwidth}{1pt}
						
						
				\begin{algorithmic} [1]
					
					\IF{|timeseries| = T} 
						\STATE return 'c' 	\COMMENT{Constante}
					\ELSE 
						\IF {|timeseries| = 1} 
							\STATE return 'o' \COMMENT{OneShot}
						\ELSE
							\IF{$t_{i+1} -t_i$ =1 $\forall t_i \in $ timeseries}
								\STATE return 'r' : timeseries[0] :timeseries[size-1] \COMMENT{Ranged}
							\ELSE
								\IF {{$t_{i+1} -t_i$ = Constante $\forall t_i \in $ timeseries}}
									\STATE retourner 'p' : timeseries[1] - timeseries [0] \COMMENT{periodic}
								\ELSE
									\STATE return 'f':timeseries[0]:...:timeseries[size-1] \COMMENT{flickering}
								\ENDIF							
							\ENDIF
						\ENDIF
					\ENDIF
					
				\end{algorithmic}
			\end{algorithm}
			 
			 
			 \vskip 0.1in
			 
			 Une dernière phase consiste en la codification du graphe en utilisant le codage proposé pour chaque sous-structure de la phase (02). Nous concaténons par la suite ces représentations pour obtenir une seule représentation concise du graphe initial en terme de ses structures les plus denses. Nous schématisons ces différentes étapes dans la figure \ref{ddsmScheme}.
			 
			 

			  
			L'algorithme \ref{alg:DDSM} représente l'algorithme de compression globale.
			\begin{algorithm}
					
					\caption{DDSM}
					\label{alg:DDSM}
					%\label{Pseudo Algorithme de la méthode proposée (DDSM)}
				\begin{algorithmic} [1]
					\STATE \textbf{Génération des sous structures candidates: }Génération de sous-graphes, principalement les bicliques et les cliques.
					
					\STATE  \textbf{Étiquetage de sous-structures: }Associer chaque sous-structure à une signature temporelle décrivant son comportement.
					
					\STATE \textbf{Codification du graphe compressé: }Codifier les sous-structures étiquetées de l'étape (02) en utilisant les structures de la section précédente.
				\end{algorithmic}
			\end{algorithm}
			
			
		Pour les algorithmes de parcours et d'extraction de voisins, tous les algorithmes proposés dans \citep{hernandez2014compressed} peuvent être généralisés sur notre structure. En effet, le seul changement consiste en la  prise en compte de l'information temporelle dans l'incrémentation des indices de parcours. De ce fait, nous pensons que notre méthode peut offrir un très bon compromis entre l'espace mémoire et le temps d'accès des traitements dans le cas des graphes dynamiques. 
		
		\begin{figure}[H]
\centering
	\includegraphics[scale=0.4]{./ressources/image/ddsm_scheme.png}
	\caption[Les différentes phases de DDSM]{Les différentes phases de DDSM}
	\label{ddsmScheme}
\end{figure}
			
		%\subsubsection{Conclusion}
	%Dans ce chapitre, nous avons abordé le problème de compression des graphes dynamiques par extraction de motifs. Nous avons formalisé le problème de compression d'un graphe dynamique en se basant sur les sous-graphes denses temporelles qui le composent. Pour faire face à ce problème nous avons présenté une nouvelle méthode hybride intitulée DDSM, elle combine deux méthodes de la littérature que nous avons jugé efficaces. Notre prochain but est de tester cette méthode en la comparant avec les méthodes existantes.   

	

\chapter{Conception}
	\section{Introduction}
	
	La réalisation de l'étude bibliographique dans le précédent chapitre, nous a initié au domaine de compression en générale et nous a permis d'approfondir nos connaissances dans le domaine de compression des graphes. Nous avons pu voir en détails les caractéristiques de plusieurs méthodes de compression s'inscrivant dans les deux classes de méthodes de compression: celles basées sur l'extraction de motifs et celles basées sur les arbres $k^2$-tree. La vocation de ce travail est d'enrichir un projet existant en l'étendant avec deux moteurs de compression qui regroupent différentes stratégies des deux classes étudiées. La réalisation de ces deux moteurs permettra de comparer entre ces stratégies et d'avoir une idée plus claire sur leurs performances dans différents scénarios.
	
	Dans ce chapitre, nous présenterons les détails de la conception des deux moteurs. Nous allons, dans un premier temps, présenter leur principe de fonctionnement tout en mettant l'accent sur les différents modules qui les constituent. Nous expliquerons juste après notre deuxième contribution qui consiste en une nouvelle  méthode de compression destinée aux graphes dynamiques.
	

	\section{$k^2$-GraCE : }
	\label{k2}

Dans cette section, nous présenterons les détails de notre premier Moteur baptisé $k2$-GraCE pour $k^2$-trees Graph Compression Engine. Il consiste en un moteur de compression de graphes par les arbres $k^2$-trees qui exploitent les propriétés de localité et de similarité dans les graphes du web. 

		\subsection{Principe de fonctionnement :}
		
	Le moteur $k^2$-GraCE a été conçu pour permettre la compression de graphes statiques ou dynamiques et orientés ou non orientés. Il se base sur les travaux de Brisaboa et al. \citep{brisaboa2009k}. Il offre la possibilité de construire le compressé à partir de la matrice d'adjacence, de la liste d'adjacence ou du graphe directement. Il permet aussi dans le cas des graphes dynamiques de réduire davantage la taille de l'arbre en le construisant n'ont pas à partir de la matrice d'adjacence initiale mais en calculant une matrice de différence entre les instants $t_i$. 
	
	
$k^2$-GraCE est un moteur de compression sans perte de données. En effet, il construit en sortie un arbre $k^2$-tree incluant toute information présente dans le graphe initial. Cette information est représentée sous forme de deux chaines binaires. Le processus de compression d'un graphe de données par le moteur $k^2$-GraCE passe par les étapes suivantes :
\begin{enumerate}
\item Lecture et structuration du graphe de données en entrée 
\item Pré-traitement : cette étape est optionnelle, elle ne figure pas dans l'algorithme de base. 
\item Construction récursive de l'arbre k2-tree à partir de la matrice d'adjacence (resp. la liste d'adjacence) et la concaténation des différents niveaux dans une première chaine T, à l'exception du dernier niveau qui sera stocké dans une deuxième chaine L.

\item Écriture du graphe compressé sur le fichier de sortie.
\end{enumerate}

Nous illustrons ces phases par la figure \ref{k2grace} qui donne une vue globale sur le principe de fonctionnement de ce moteur.


\begin{figure}[H]
\includegraphics[scale=0.48]{./ressources/image/ograce.jpg}
\caption[Principe de fonctionnement du moteur $k^2$-GraCE]{Principe de fonctionnement du moteur $k^2$-GraCE.}
			\label{k2grace}
\end{figure}
Après avoir expliquer de manière générale le principe de fonctionnement du moteur $k^2$-GraCE, nous présenterons dans ce qui suit les détails de la phase de compression (construction de l'arbre). Nous allons tout d'abords commencer par présenter les notations et opérations de bases qui seront utilisées par la suite. Nous enchainerons par les différents algorithmes de construction selon le types de graphes en entrée.
		
		\subsection{Paramètre et notations :}
				
		\begin{table}[H]
		\centering
		\begin{tabular}{|c|L{12cm}|}
		\hline Paramètre & Signification \\ \hline\hline 
		$G$ & Graphe de données \\ \hline
		$M$ & Matrice d'adjacence du graphe G\\ \hline
		$List$ & Liste d'adjacence du graphe G\\ \hline
		$A$ & L'arbre $k^2$-tree \\\hline
		$T$ & Le nombre de instants dans lesquels le graphe a été capturé \\\hline
		$h$ & L'hauteur de l'arbre $k^2$-tree \\ \hline
		$N$ & Nombre de nœuds dans le graphe \\ \hline
		$k$ & Paramètre déterminant le nombre de fils dans l'arbre $k^2$-tree\\ \hline
		rank($T,i$) &  Fonction calculant le nombre de 1 existant dans le tableau binaire $T$ dans l'intervalle des indices [$1,i$] \\ \hline
		\end{tabular}
		\caption{Tableau des notations et paramètres du moteur $k^2$-GraCE.}	
		\label{notk2grace}		
	\end{table}
	
		\subsection{Conception Modulaire:}
		
		Dans cette partie, nous détaillerons chaque phase du processus de compression du moteur $k^2$-GraCE. Nous commencerons par présenter les différentes techniques de pré-traitement 
		que nous voulons utiliser tout au long avec notre moteur. Nous expliquerons par la suite le processus de compression pour chaque type de graphe supporter par le moteur $k^2$-GraCE et nous allons aussi fournir les différents algorithmes d'extraction de voisins en Annexe \ref{k2_annexe}. 
		
			\subsubsection{Pré-traitement du graphe de données:}
Durant cette première phase, le moteur $k^2$-GraCE utilise des stratégies qui permettrons d'aboutir à une meilleur compression. Il offre une alternative pour chaque type de graphe qu'il supporte. 		
			
			
			Dans le cas des graphes statiques orienté, $k^2$-GraCE offre la possibilité de ré-ordonnoncer les nœuds du graphe de données G.
			% Nous expliquerons  dans ce qui suit les différents algorithmes de ré-ordonnancement qui peuvent être utiliser dans notre moteur. 
			Comme notre travail est une suite d'un travail d'étudiants de l'année précédente \citep{master2017}, ce module a été déjà conçu et implémenté. Nous rappellerons uniquement dans cette section le principe de base de chaque méthode. 
			
			\begin{itemize}[label=$\bullet$]
\item\textbf{Ordre Lexicographique :} Les nœuds sont ordonnés selon leurs listes de successeurs. Les listes de successeurs seront donc triées selon un ordre croissant des identifiants et par la suite ordonnées.

	
\item\textbf{Ordre Gray :} Les nœuds sont permuter de telle sorte que deux nœuds dont l'ordre est successif différent dans au plus un voisin. 
\item\textbf{Ordre \gls{dfs} :} Les nœuds sont ordonnés selon leurs positions dans le parcours en largeur (\gls{dfs}) du graphe.  
\item\textbf{Ordre \gls{bfs} :} Les nœuds sont ordonnés selon leurs positions dans le parcours en profondeur (\gls{bfs}) du graphe. 
\item\textbf{Ordre Aléatoire :} Des permutations aléatoires des nœuds sont établies.
			
			\end{itemize}
		
	Le deuxième type de graphe supporté par le moteur $k^2$-GraCE sont les graphes statiques non orientés. Dans ce cas, nous obtenons une matrice d'adjacence symétrique. De ce fait, nous proposons de ne considérer que la partie triangulaire haute pour enlever la redondance porté par la symétrie. L'arbre construit ainsi offre toujours la possibilité d'extraire le voisinage d'un nœuds sans avoir recours à une décompression. En effet, extraire les voisins d'un nœud dans le graphe initial revient à extraire les voisins directes et inverses dans la nouvelle matrice construite en utilisant les mêmes algorithmes de l'annexe \ref{k2_annexe}.
	 Si nous prenons l'exemple du nœuds 2 dans la figure \ref{k2_non}, ses voisins sont représentés par les 1 de la deuxième ligne ou de la deuxième colonne dans la matrice d'origine (matrice gauche) et par l'union des cellules ayant un 1 de la deuxième ligne et de la deuxième colonne dans la matrice droite ce qui nous donne : $v(2) = \{1, 2, 5,6\}$.
	
\begin{figure}[H]
\centering
	\includegraphics[scale=0.4]{./ressources/image/k2_non.jpg}
	\caption[Exemple d'arbre $k^2$-tree (k=2) pour un graphe non orienté.]{Exemple d'arbre $k^2$-tree (k=2) pour un graphe non orienté.}
	\label{k2_non}
\end{figure}	


	
	$k^2$-GraCE supporte aussi les graphes dynamiques. Dans le cas de ce type de graphe, il offrent la possibilité de calculer une matrice différence à partir de la matrice initiale. Le but de cette fonctionnalité est de maximiser les zones nulles dans la matrices afin de réduire la taille de l'arbre. À t=0, on garde la même matrice bidimensionnelle du graphe initiale. Pour les instants restants (t > 0), nous comparons $M_{pq}$ à l'instant t avec $M_{pq}$ à l'instant t-1, si égalité alors la nouvelle matrice contiendra un 0 dans la cellule pq à l'instant t sinon elle contiendra un 0. La nouvelle matrice d'adjacence contiendra ainsi uniquement les changement qui occurrent entre les instants. L'inconvénient de cette fonctionnalité est qu'une reconstruction de la matrice initiale est nécessaire dans le cas d'interrogation du graphe. La figure \ref{mat_diff} montre une matrice d'adjacence d'un graphe dynamique capturé dans trois instants différents avec sa matrice de différence et la représentation $k^2$-tree dans les deux cas.

\begin{figure}[H]
\centering
	\includegraphics[scale=0.48]{./ressources/image/dynk2diff.png}
	\caption[Exemple d'arbre $k^2$-tree (k=2) pour la matrice de différence]{Exemple d'arbre $k^2$-tree (k=2) pour la matrice de différence.}
	\label{mat_diff}
\end{figure}

Nous notons que l'utilisation de ce module dans le processus de construction de l'arbre $k^2$-tree n'est pas obligatoire. Cette phase ne figure pas dans l'algorithme de base. Nous voulons, à travers son intégration dans le moteur \gls{$k^2$-GraCE}, rassembler davantage les nœuds ayant des voisins communs et donc les zones homogènes dans la matrice d'adjacence (zones plaines de 0 ou zones plaines de 1) ou augmenter le nombre de cellules nulles dans la matrice d'adjacence.
		
			\subsubsection{Construction de l'arbre $k^2$-tree}
			
			$k^2$-tree est une représentation compacte de la matrice d'adjacence qui exploite ses propriétés de dispersion et de clustering. Elle était destinée au départ pour les graphes du web et généralisée par la suite pour différents cas. La représentation est conçue pour compresser les grandes zones nulles de la matrice d'adjacence en les représentant avec un nombre réduit de bits. Plusieurs alternatives existent pour construire l'arbre $k^2$-tree: utilisation de la matrice d'adjacence, l'utilisation de la liste d'adjacence ou l'utilisation du graphe directement. Dans tous les cas, on obtiendra une représentation sous forme d'un arbre ayant une hauteur h= $log_k(N)$ où chaque nœud possède $k^2$ fils. 
			  Pour représenter l'arbre de manière concise, deux structures seront utilisées :
			\begin{itemize}
			\item \textbf{T }: Un tableau qui stocke tous les bits du $k^2$-tree sauf ceux du dernier niveau. Les bits de l' arbre $k^2$-tree sont placés après une traversée horizontale de l'arbre. $k^2$-GraCE représentent d'abord les $k^2$ valeurs binaires des enfants du nœud racine, puis les valeurs du deuxième niveau, ...etc.
			
			\item  \textbf{L} : Un tableau stockant le dernier niveau de l'arbre. Ainsi il représente la valeur de (ou certaines) cellules de la matrice d'adjacence du graphe initiale.
			\end{itemize}
			
			L'algorithme \ref{alg:k2_tree} résume les différentes étapes de construction de l'arbre $k^2$-tree dans le cas de la construction à partir de la liste d'adjacence. Nous dotons pour cela la liste d'adjacence de n curseurs, un par ligne, de sorte que chaque fois que nous devons accéder à $M_{pq}$, nous comparons le curseur actuel de la ligne $p$ à la valeur q. S'ils sont égaux, alors on aura $M_{pq} = 1$ et nous devons avancer le curseur vers le nœud suivant de la liste de la ligne p. Sinon, nous saurons que $M_{pq} = 0$. Nous supposons que la liste d'adjacence $List$, le nombre de fils $k$ ainsi que l'arbre $k^2$-tree $A$ sont des variables globales. Pour construire l'arbre à partie de la racine, elle sera invoquer comme suit :  ConstructK2Tree(N,1,0,0). Après avoir construit l'arbre $A$ sous forme d'un tableau de niveau, nous procèderons à la construction des deux structures selon les deux formules suivantes :
			\begin{itemize}[label=$\bullet$]
			 	\item T = $A_1: A_2:. . . :A_{h-1}$
			 	\item L = $A_h$
			\end{itemize}	

Des versions itératives de l'algorithme peuvent être établies. Cependant, elles dégradent les performances car elles nécessitent soit l'utilisation de structures supplémentaires ou plus d'accès mémoire. Nous avons privilégié la version récursive car elle permet de construire l'arbre $k^2$-tree en un seul parcourt de la liste d'adjacence $List$.\\

					\begin{algorithm}[H]
					\label{alg:k2_tree}
					\caption{ConstructK2Tree}
					\textbf{Entrée :}
						\begin{itemize}[label=$\bullet$]
							\item n : la taille de la sous-matrice
							\item l : le niveau de l'arbre
							\item p : l'indice ligne de début de la sous-matrice
							\item q : l'indice colonne de début de la sous-matrice
						\end{itemize}
					\textbf{Sortie :} La valeur du nœud de la sous-matrice\\							\noindent\rule{\textwidth}{1pt}
						
						
				\begin{algorithmic} [1]
					\STATE C = $ \emptyset$
					\FOR{$i = 1 ... k $} 
						\FOR{$i = 1 ... k $} 
							\IF {$l = log_k(N)$} 
								\STATE // Condition selon le type de représentation en entrée 
								\STATE // Matrice d'adjacence : M[p+i,q+j] = 1 
								\STATE // Graphe : $e_{p+i,q+j} \in E$
								\IF { $List$[p+i].courrant() = q+j }
								
									\STATE C = C : 1
									\STATE $List$[p+i].avancer()
								\ELSE
									\STATE C = C : 0
								\ENDIF
							\ELSE
								\STATE  C = C : ConstructK2Tree ( $n/k,l+1,p+i*(n/k), q+j*(n/k)$)
							\ENDIF
						\ENDFOR
					\ENDFOR
					\IF {$C\ est\ un\ vecteur\ nul$} 
						\RETURN 0
					\ENDIF
					\STATE A[l] = A[l] : C
					\RETURN 1
				\end{algorithmic}
			\end{algorithm}
	
	\newpage		
		Une fois construites, les deux structures, T et L, permettent d'extraire les voisins directes et inverses d'un nœud directement sans décompression. Nous fournissons dans l'annexe \ref{k2_annexe} les détails de ces  algorithmes d'extraction de voisinage.		
			
	

	Un autre type de graphe supporté par notre moteur sont les graphes dynamiques. Ces graphes sont représentés par un ensemble de graphes statiques chacun capturé à un instant $t_i$. De ce fait, l'algorithme de construction peut être facilement adapté avec chaque nœud dans l'arbre contenant, cette fois-ci, un vecteur de bits chacun faisant référence à un instant $t_i$. Nous fournissons ci-après (algorithme \ref{alg:dynk2_tree}) l'algorithme de construction de l'arbre $k^2$-tree à partir de la matrice d'adjacence tridimensionnelle.\\
	


	
					\begin{algorithm}[H]
					\label{alg:dynk2_tree}
					\caption{DynK2Tree}
					\textbf{Entrée :}
						\begin{itemize}[label=$\bullet$]
							\item n : la taille de la sous-matrice
							\item l : le niveau de l'arbre
							\item p : l'indice ligne de début de la sous-matrice
							\item q : l'indice colonne de début de la sous-matrice
						\end{itemize}
					\textbf{Sortie :} La valeur du nœud de la sous-matrice\\							\noindent\rule{\textwidth}{1pt}
						
						
				\begin{algorithmic} [1]
					\STATE C = $ \emptyset$
					\STATE Creturn  = $ \emptyset$
					\FOR{$i = 1 ... k $} 
						\FOR{$j = 1 ... k $} 
							\IF {$l = log_k(N)$}
								\FOR{$m = 1 ... T $}
									\STATE C[m] = C[m] : M [p+i][q+j][m]
								\ENDFOR
							\ELSE
								\STATE Ctmp =$\emptyset$
								\STATE  Ctmp = DynK2Tree ( $n/k,l+1,p+i*(n/k), q+j*(n/k)$)
								\FOR{$m = 1 ... taille(Ctmp) $}
									\STATE C[m] = C[m] : Ctmp[m]
									\STATE Creturn [m] = Creturn [m] ou Ctmp [m]
								\ENDFOR
							\ENDIF
						\ENDFOR
					\ENDFOR
					\IF {$C\ est\ un\ vecteur\ nul$} 
						\RETURN $0^{|T|}$ // retourner un vecteur nul de taille T.
					\ENDIF
					
					
					
					\FOR{$i = 1 ... k*k $} 
						\FOR{$j = 1 ... T $}
							\IF{ $C[i]$ n'est pas un vecteur nul}
								\STATE A[l] = A[l] : C [i][j]
							\ENDIF
						\ENDFOR
					\ENDFOR
	
					 \RETURN Creturn
					
				\end{algorithmic}
			\end{algorithm}
			
			
	
	
	
	\section{P-GraCE :}
	
	Notre deuxième moteur \gls{P-GraCE} est un moteur de compression par extraction de motifs. Nous présenterons dans cette partie, en détails, son principe de fonctionnement et les différents modules qui le constituent.
	
		\subsection{Principe de fonctionnement :}
		
		Les méthodes de compression par extraction de motifs sont des méthodes qui essayent de représenter le graphe de données à travers ses motifs, autrement dit ses sous-graphes portant les informations les plus importantes. 
		
		P-GraCE est un moteur de compression comportant plusieurs méthodes qui peuvent être avec ou sans perte de données. 
		En effet, le modèle produit en sortie (résultat de la compression) est parfois accompagné avec une matrice d'erreur. Le processus de compression d'un graphe de données par le moteur P-GraCE passe par les étapes suivantes :
		
\begin{enumerate}

\item Lecture et structuration du graphe de données en entrée 

\item Extraction et évaluation des motifs : Durant cette phase, une détection des motifs les plus denses ou les plus fréquents est réalisée. Leur évaluation dans cette phase permet de ne garder que les structures (motifs) susceptible de donner une meilleur compression.

\item Traitement des motifs: Ce module permet d'encoder ou d'agréger, dans certains cas, les motifs déjà découverts dans le but de minimiser d'avantage la taille du graphe en sortie. Dans d'autres cas, le module retourne une liste de structures sélectionnés parmi les motifs précédemment identifiés en utilisant des heuristiques dans le but de garder que les motifs importants du graphe. Nous notons que l'utilisation de ce module n'est pas présente dans toutes les méthodes de ce moteur mais dans la plupart 

\item Écriture du graphe compressé sur le fichier de sortie.
\end{enumerate}

La figure \ref{P_grace} permet d'illustrer le principe de fonctionnement globale du moteur \gls{P-GraCE}


\begin{figure}[H]
	\includegraphics[scale=0.48]{./ressources/image/pgrace.jpg}
	\caption[Vue globale sur le fonctionnement du moteur \gls{P-GraCE}.]{Vue globale sur le fonctionnement du moteur \gls{P-GraCE}.}
	\label{P_grace}
\end{figure}

Nous présenterons dans ce qui suit les trois modules composants le moteur P-GraCE. Nous précéderons  cela par expliquer les différents paramètres et notations que nous avons adopté durant la conception de notre deuxième moteur
		
		\subsection{Paramètre et notations :}
		
			\begin{table}[H]
		\centering
		\begin{tabular}{|c|L{12cm}|}
		\hline Paramètre & Signification \\ \hline\hline 
		$G$ & Graphe de données \\ \hline
		$V$ & L'ensemble des nœuds de G\\ \hline
		$E$ & L'ensemble des arêtes de G \\ \hline
		$L$ & L'ensemble des étiquettes de G \\ \hline
		$A$ & Matrice d'adjacence du graphe G\\ \hline
		$M$ & Le modèle produit par la compression\\ \hline
		$h$ & Heuristique d'évaluation \\ \hline
		$N$ & Nombre de nœuds dans le graphe \\ \hline
		$k$ & Paramètre déterminant le nombre de fils à considérer dans le beam-search\\ \hline
		
		\end{tabular}
		\caption{Tableau des notations et paramètres du moteur $k^2$-GraCE.}	
		\label{notk2grace}		
	\end{table}
		
		\subsection{Conception modulaire :}
		
		Durant cette section, nous allons présenter les différentes approches qui peuvent être utilisées dans chacune des phases. Nous commencerons par expliquer les méthodes d'extraction de motifs supportées par le moteur P-GraCE. Nous enchainerons avec les techniques d'évaluation que notre moteur offre, pour finir par expliquer comment ces motifs seront traités et utilisés pour compresser le graphe. 	
		
		
		\subsubsection{Extraction des motifs :}

La phase d'extraction de motifs est une phase très importante	 dans le processus de compression. Elle permet de trouver les composantes les plus denses qui représentent en générale l'information utile dans un graphe de données. Plusieurs techniques existent pour réaliser cette tache. Elles diffèrent dans la qualité de sous-structures découvertes selon le domaine d'application et le type de graphe en entrée. Nous présenterons ci-dessous les trois méthodes offertes par le moteur P-GraCE.
		
\begin{enumerate}

\item \textbf{Beam search:}\\

Le beam search est une méthode de recherche locale gloutonne. C'est une version améliorer de l'algorithme de recherche en largeur \gls{bfs}. Elle permet de n'explorer que les $k$ meilleurs fils dans chaque niveau à travers des heuristiques d'évaluation.

Dans le cas d'extraction de motifs dans un graphe, le beam search commence par considérer que chaque nœud du graphe est un motif. Il utilise pour cela un arbre de recherche où ses nœuds sont des sous-structures. À chaque itération les motifs sont étendus par une arête et un nœud donnant ainsi un ensemble de sous-structures. Par la suite, les $k$ meilleurs sous-structures sont choisies. Les fils restants sont donc élagués. %Le processus continues jusqu'à ce que
Cette alternative sera employée pour le cas des graphes étiquetés. Nous fournissons ci-après l'algorithme du beam search que nous allons utiliser. 


\begin{algorithm}[H]
					\label{alg:beamSearch}
					\caption{Beam-Search}
					\textbf{Entrée :}
						\begin{itemize}[label=$\bullet$]
							\item G : le graphe de donnée
							\item k : le nombre de fils à considérer
							\item limit : limite de la profondeur de l'arbre
						\end{itemize}
					\textbf{Sortie :} retourne la meilleur sous-structure\\							\noindent\rule{\textwidth}{1pt}
						
						
				\begin{algorithmic} [1]
					\STATE C = $\{v\ |\ v$ est un nœud ayant une unique étiquette dans  $G\}$
					\STATE $meilleurStruct$ = la première sous-structure de C
					\REPEAT
					\STATE nouvelC= $\emptyset$
					\FOR{ chaque S dans C} 
						\STATE nouvelStruct = Etendre(S)
						\STATE Evaluer(nouvelStruct)
						\STATE nouvelC = nouvelC $\cup$ \{ les k meilleur sous-structures de nouvelStruct \}
					\ENDFOR
					\STATE limit = limit -1
					\IF {$h$( meilleur sous-structure de C) <= $h$($meilleurStruct$)  }
					
					\STATE $meilleurStruct$  = meilleur sous-structure de C
					\ENDIF
					\STATE C = nouvelC
					\UNTIL{$C = \emptyset $ ou $limit \leq 0$}
					
				\end{algorithmic}
			\end{algorithm}



\item \textbf{Méthodes de clustering :}

Plusieurs méthodes de clustering existent dans la littérature. P-GraCE utilise une  méthode de clustering qui a été destinée pour compresser un graphe de données en entrée s'intitulant \textit{SlashBurn}. Cette technique parte de l'observation  que les graphes du monde réel sont facilement décomposables en supprimant leurs nœuds concentrateurs qui sont définies comme les neouds ayant un degré maximale dans le graphe G.  

Ce module de P-GraCE permet donc de réordonner les nœuds du graphe de tel sorte à avoir un plus petit nombre de blocs plus denses dans la matrice d'adjacence qui représentent les motifs en sortie.  

À chaque itération le nœud concentrateur est supprimé et le graphe est décomposé en un nœud concentrateur (hub), un
\newacronym{gcc}{CCG}{Composant Connecté Géant}
\gls{gcc} et des rayons restants que nous définissons comme étant le composant connecté non géant connecté aux anciens \gls{gcc}. Le nœud concentrateur obtient ainsi le plus petit identifiant , les nœuds des rayons (spokes) reçoivent les identifiants les plus élevés  dans l’ordre décroissant de la taille du composant connecté auquel ils appartiennent, et le nouveau \gls{gcc} prend les identifiants restants. Le même processus s'applique aux nœuds dans \gls{gcc}, de manière récursive.\\

Nous illustrons le principe de fonctionnement de cette méthode de clustering sur le graphe de la figure \ref{Img:slashb}. Le nœud concentrateur (8) obtient le plus petit identifiant (1), les nœuds des rayons reçoivent les identifiants les plus élevés (9-16), et le \gls{gcc} prend les identifiants restants (2-8). La prochaine itération considère le nouveau \gls{gcc}.

\begin{figure}[H]
	\centering
	\label{Img:slashb}
	\includegraphics[scale=0.25]{ressources/image/slashburn.png}
	\caption{Exemple d'application de l'algorithme SlashBurn}

 \end{figure}





\item \textbf{Fonctions de hachage :}

Dans cette alternative, nous allons partitionner le graphe en des motifs denses dont les nœuds sont fortement connectés. Nous utiliserons pour cela une approximation de \textit{la similarité de Jaccard} qui est une métrique permettant de mesurer la similarité entre deux ensembles $S_1$ et $S_2$ (voir formule \ref{eq:Jacc}). 

\begin{equation} \label{eq:Jacc}
SIM(S_1,S_2)=\frac{|S_1 \cap S_2|}{|S_1 \cup S_2|}
\end{equation}


L'objectif de cette méthode est de remplacer les listes d'adjacence des nœuds par des représentations beaucoup plus petites appelées « signatures ». La propriété importante de ces signatures est que leur comparaison permet d'estimer la similitude de Jaccard des listes d'adjacence sans avoir recours à les comparer deux à deux. 

Avant d'expliquer comment il est possible de construire de petites signatures à partir des listes d'adjacence, il est utile de les visualiser en tant que matrice caractéristique. Les colonnes de cette matrice correspondent aux listes d'adjacence et les lignes correspondent aux nœuds. Nous rappelons que la matrice caractéristique est peu susceptible d'être la façon dont les données sont stockées, mais elle est utile pour visualiser les données (voir figure \ref{jacc}). 


Dans un premier temps, k-permutations aléatoires des nœuds devront être choisies. L'utilisation des k-permutations se base sur le fait que :
%% Ajouter demonstration en Annexe ou donner la source 
\begin{equation} \label{eq:prob}
P(\pi_i(S_1) = \pi_j(S_2)) = \frac{|S_1 \cap S_2|}{|S_1 \cup S_2|} = SIM(S_1,S_2)
\end{equation}
Cependant, le choix de ces permutations peut être prohibitive dans le cas des grands graphes.
À la place des k-permutations, nous utiliserons donc une famille de fonctions de hachage indépendantes qui donnerons le même résultats. La famille de fonction que nous visons utiliser est une famille polynômiale : $h_{a_0, ..., a_{k-1}}(x)= (\sum a_i x^i)\ mod\ N$. Ainsi, au lieu de choisir n permutations aléatoires de lignes, nous sélectionnons n fonctions de hachage choisies de manière aléatoire $h_1, h_2,. . . , h_k$ sur les lignes. Nous construisons la matrice de signature en considérant chaque ligne dans leur ordre donné. Soit $SIG (i, c)$ l'élément de la matrice de signature pour la $i^{ème}$ fonction de hachage et la colonne c. SIG (i, c) est initialisée à $\infty$ pour tout i et c. Nous traitons une ligne r en procédant comme suit:
\begin{enumerate}
\item nous calculons $h_1 (r), h_2 (r), ..., h_k (r)$.
\item Pour chaque colonne, nous procédons comme suit: 
	\begin{itemize}
		\item  Si c a 0 dans la rangée r, rien à faire.
		\item  Cependant, si c a 1 dans la rangée r, alors pour chaque i = 1,2, ..., k  \\$SIG (i, c)$ = min( $SIG (i, c)$ , $h_i(r)$ ).
	\end{itemize}
\end{enumerate}


\end{enumerate}

Une fois les signatures sont construites, les nœuds ayant les même valeurs seront regroupés. Nous obtiendrons ainsi en sortie la liste des motifs les plus denses dans le graphe de données. Nous illustrons le principe de fonctionnement de cette stratégie sur un graphe G dans la figure \ref{jacc}. Le tableau gauche de la figure \ref{jacc} montre une matrice caractéristique des listes d'adjacence des différents nœuds du graphe G ainsi que deux permutations aléatoires $h_1$, $h_2$. Tant dis que le tableau à droite de la même figure montre les étapes de calcul des signatures en utilisant ces deux fonctions. La matrice des signatures finale montre bien que les nœuds 1 et 4 sont les plus similaires ce qui est justifié par leurs listes d'adjacence qui ne diffèrent que dans un seul élément.  

\begin{figure}[H]
\centering
	\includegraphics[scale=0.48]{./ressources/image/jacc.png}
	\caption[Exemple de calcul de la matrice des signatures.]{Exemple de calcul de la matrice des signatures.}
	\label{jacc}
\end{figure}
	
		
		\subsubsection{Évaluation des motifs :}
		
		 Dans cette deuxième phase, P-GraCE cherche à obtenir la meilleur compression en filtrant les sous-structures et ne gardant que celle qui offrent un bon \gls{mdl}. 
		 
Comme nous l'avons déjà préciser P-GraCE utilise le beam-search pour le cas des graphe  étiqueté. Il est fondamentalement guidé par le principe \gls{mdl}. L'évaluation heuristique effectuée par le \gls{mdl} suppose que la meilleure sous-structure est celle qui minimise le \gls{mdl} du graphe en entrée lorsqu'il est compressé par cette sous-structure. Notons la longueur de description de S est représentée par DL (S), la longueur de description du graphe d'entrée G par représentée par DL (G) et la longueur de description du graphe après compression par DL (G | S). L'heuristique est alors rien d'autre que le taux de compression et peut être définie ainsi :
		 
		\begin{center}
		compression = $\frac{DL(S)+DL(G|S)} {DL(G)}$
		\end{center}
		 
   
		 
		  Dans le cas de l'utilisation des méthodes de clustering, nous obtenons une liste des structures les plus fréquentes. Pour chaque structure identifiée nous cherchons le motif qui la décrit le mieux en se basant sur un ensemble prédéfini de motifs (étoile, clique, noyau bipartie, semi clique, semi noyau bipartie). Afin de trouver le motif approximatif de la structure, nous utiliserons les formules proposées dans \citep{koutra2015summarizing} (voir section \ref{vog_desc}).
		  
		   
		\subsubsection{Traitement des motifs :}
		
		Nous proposons pour traiter les motifs deux alternatives dans le moteur P-GraCE. La première consiste à faire une agrégation des nœuds du meilleur motif. Les trois étapes seront donc répétées jusqu'à ce que la taille du graphe en sortie ne peut plus être optimisée. La deuxième alternative consiste à considérer que la liste des sous-structures de l'étape 2 représentent le compressé du graphe de données. Nous offrons dans ce cas la possibilité d'utiliser le codage proposé dans \citep{liu2018reducing}. Il représente cette liste de motifs de manière concise tout en permettant les requêtes d'extraction de voisinage. Dans certains méthodes une matrice d'erreur est conservée avec la liste des structures pour une compression sans perte.
		
		
		
	\section{Notre méthode: Dynamic Dense Subgraph Mining (DDSM)}
		

Comme on vient de voir dans les chapitres précédents, de nombreux phénomènes dans des contextes très variés peuvent être représentés par des graphes. Citons en particuliers les réseaux sociaux qui nous attirent vu leur importance et leur fort impact sur la vie réelle. En effet, trois quarts des personnes connectées à Internet utilisent les réseaux sociaux. Ils rassemblent aujourd'hui 3,169 milliards d'utilisateurs actifs, voir 40\% de la population sur terre, avec 11 personnes s'inscrivant chaque seconde sur Facebook, Twitter et autres
\footnote{https://bfmbusiness.bfmtv.com/hightech/la-moitie-de-la-population-mondiale-est-connectee-a-internet-1361933.html}. De ce fait, deux constats s'imposent: le premier est l'incapacité des graphes statiques à décrire certaines situations de la vie réelle et leur évolution dans le temps, le deuxième est la montée en flèche de la quantité et l'hétérogénéité des informations modélisées par ces réseaux et qui rendent presque impossible tout traitement. D'où le besoin croissant de méthodes de compression de graphes dynamiques. 

Nous nous sommes aussi intéressées dans ce travail à un autre enjeu important qui est la recherche de motifs et particulièrement les cliques et les noyaux bipartis qui sont fréquents dans les réseaux faisant abstraction à des situations réelles importantes dans le processus de prise de décision. Nous citons par exemple, les attaquants de réseaux de botnet formant un noyau biparti avec leurs victimes pendant la durée d'une attaque, les membres de la famille se liant comme une clique au cours d'une période difficile, ou les collaborations de recherche s'étant créées et qui disparaissent au cours des années.

%Quoiqu'il existe de nombreuses techniques de compression efficaces adaptées au contexte statique, rares sont ceux qui proposent une contrepartie dynamique, ce qui nous amène à s'intéresser particulièrement à ce type de graphes toujours en cours d'évolution. 

Nous allons dans cette partie proposer une nouvelle méthode DDSM basée sur deux approches existantes. Notre but principal est de développer une technique de compression compétitive pour les graphes des réseaux d'interactions dans un contexte dynamique en se basant sur l'extraction de motifs pour filtrer les informations les plus importantes et les arbres $k^2$-trees pour aboutir à une compression sans perte. Pour accomplir ce travail nous nous sommes appuyées sur les deux techniques suivantes:
\begin{itemize}
\item Exploiter la structure de données proposée dans \gls{dsm} \citep{hernandez2014compressed} qui permet de représenter les sous-graphes denses comme les cliques, les quasi-cliques, noyaux bipartis de manière concise. Nous avons opté pour cette structure car  elle donne de bons résultats en matière d'espace et de temps de requêtes.  
\item Adopter les signatures temporelles utilisées dans TimeCrunch \citep{shah2015timecrunch} pour décrire le comportement des sous-graphes dans le temps. Nous avons utilisé ces signatures car elles englobent tous les types de comportements temporels d'un motif dans un graphe dynamique.
\end{itemize}	
%Nous allons dans cette partie présenter notre méthode théoriquement : principe général, structure utilisée et un pseudo code.	
		
		\subsection{Formulation du problème}
		Dans cette section, nous allons définir le problème de base auquel notre méthode convient tout en définissant le cadre dans lequel elle peut être appliquée. 
		
		Nous considérons les graphes orientés dynamiques $\displaystyle{G=\bigcup_{t_{i}}G_{t_{i}}(V,E_{t_{i}})}\ \ 1 \preceq i \preceq t$ où $G_{t_{i}}$ = G à l'instant $t_{i}$ et un ensemble de signatures temporelles. En d'autre termes, nous considérons des captures du graphe à des instants différents $t_{i}$ et un ensemble de descripteurs de comportement temporel des sous-graphes des différents $G_{t_{i}}$. Le problème peut ainsi être formulé:
		
			\textit{\textbf{Problème:}
		Trouver, à partir d'un graphe dynamique G et un lexique de signatures temporelles, la plus petite description du graphe initial en termes de ses sous structures les plus denses et leur comportement temporel, en offrant la possibilité de manipuler le graphe, d'extraire les voisins directs d'un nœud à un instant donné et d'extraire les sous-graphes au besoin. }

		\subsection{Principe général}
			Notre méthode s'intitule DDSM pour \textit{Dynamic Dense Substructure Mining}. Elle représente une généralisation du travail de Hernandez et Navarro \citep{hernandez2014compressed} pour le cas des graphes dynamiques qui sont de nos jours omniprésents.
			
			%Pour mieux expliquer notre méthode nous allons tout d'abord commencer par décrire les structures que nous allons utiliser. Nous enchainons par la suite  avec le pseudo algorithme détaillant les étapes essentielles de notre méthode et nous conclurons en expliquant comment les algorithmes proposés dans \citep{hernandez2014compressed} de manipulation des graphes peuvent être généralisés dans notre cas.
			%\subsection{Description conceptuelle}
			
			La codification du graphe en sortie doit respecter les contraintes du problème tout en réduisant un maximum d'espace mémoire. Pour cela, nous proposons d'étendre la codification proposée dans \citep{hernandez2014compressed} pour le cas des graphes dynamiques en rajoutant une information temporelle. Une fois les sous-structures identifiées, Hernandez et al. \citep{hernandez2014compressed} proposent de représenter chacune d'elles avec trois composantes: la première contenant les sommets ayant uniquement des arêtes sortantes, la deuxième contenant les sommets ayant des arêtes entrantes et sortantes et la troisième contenant les sommets ayant uniquement des arêtes entrantes. Pour pouvoir identifier les différentes composantes, ils associent un vecteur binaire à cette représentation marquant par un 1 le début de chacune des trois composantes (voir figure \ref{SDM}). Notre première contribution consiste en l'extension de cette structure. En effet, nous suggérons de représenter chaque sous structure avec non pas trois composantes mais quarte composantes. Les trois premières étant les même que dans \citep{hernandez2014compressed}, la quatrième représente l'information temporelle. Pour plus de clarté, nous illustrons la représentation dans figure \ref{ddsmCode}.
	
		\begin{figure}[H]
\centering
	\includegraphics[scale=0.45]{./ressources/image/dsmRepresentation.jpg}
	\caption[Codage des structures dans DDSM]{Codage des structures dans DDSM}
	\label{ddsmCode}
\end{figure}			
			
			
L'information temporelle dans l'encodage peut avoir cinq formats (voir section \ref{par:TimeCrunch}) dont nous résumons la représentation proposée pour chacune dans le tableau \ref{tab:signtmp}.  
			
			\begin{table}[h]
			
			\begin{center}
			\begin{tabular}{|r|l|}
			\hline Signature temporelle & Représentation	
			\\\hline constante & 0	
			
			\\\hline OneShot & 1 $t_{1}$	
			
			\\\hline ranged & 2	$t_{1}\ t_{2}$	
			
			\\\hline periodic & 3  $T$	
			
			\\\hline flikering & 4 $t_{1}\ t_{2}\ ...\ t_{n}$	
			
			\\\hline			
			\end{tabular}
			\end{center}
			
			 \caption{\small {Les types de signatures temporelles et leurs représentations, $t_{i}$	 représentent les timestamps et T représente la période.}}
				\label{tab:signtmp}			
			\end{table}
			
			Nous représentons G donc comme un ensemble de sous-graphes temporels denses. Cependant, pour obtenir une compression sans perte, nous devons aussi garder l'erreur modélisant l'ensemble d'arêtes restantes dans chaque $G_{t_{i}}$. Nous proposons pour cela d'utiliser une des structures dynamiques des k2-trees qui nous permettra de garder trace de cette dernière dans un format compacte. Nous ferons appel à cette étape au moteur $k^2$-GraCE.
			
			%\subsection{Notre méthode : DDSM}
			 Nous visons à travers la méthode que nous proposons d'exprimer le graphe en entrée avec ses sous-structures les plus denses et leur comportement temporel réduisant ainsi sa taille et offrant la possibilité d'effectuer les traitements dans un temps meilleur. Nous avons structuré notre algorithme sous forme de trois (03) étapes essentielles, chacune servant d'entrée pour l'étape suivante.
			 
			 En premier lieu, nous appliquons la découverte des sous-graphes les plus denses. Pour effectuer cela de manière efficace, nous suggérons d'utiliser la technique du MinHashing du moteur P-GraCE parallèlement sur chaque capture $G_{t_{i}}$ de G. 
			 
			 
			 Dans une deuxième phase, nous effectuons une comparaison entre les sous-structures. Nous utilisons pour cela le principe du MinHashing encore une fois pour grouper les sous-structures de différents timestamps ayant un nombre élevé de nœuds en commun. Ayant associé à chaque sous-structure ses différents timestamps, ces derniers seront à leur tour compressés en utilisant les descripteurs de comportement temporel précédemment décrits. L'algorithme \ref{alg:getTemporalSignature} représente les étapes de compression de l'ensemble des timestamps d'une structure.\\
			 
			 
			 	\begin{algorithm}[H]
					
					\caption{getTemporalSignature}
					\label{alg:getTemporalSignature}
					\textbf{Entrée :}
						\begin{itemize}[label=$\bullet$]
							\item timeseries[] : un tableau de timestamps
							\item size : la taille du tableau timeseries
						\end{itemize}
					\textbf{Sortie :} la signature temporelle\\							\noindent\rule{\textwidth}{1pt}
						
						
				\begin{algorithmic} [1]
					
					\IF{|timeseries| = T} 
						\STATE return 'c' 	\COMMENT{Constante}
					\ELSE 
						\IF {|timeseries| = 1} 
							\STATE return 'o' \COMMENT{OneShot}
						\ELSE
							\IF{$t_{i+1} -t_i$ =1 $\forall t_i \in $ timeseries}
								\STATE return 'r' : timeseries[0] :timeseries[size-1] \COMMENT{Ranged}
							\ELSE
								\IF {{$t_{i+1} -t_i$ = Constante $\forall t_i \in $ timeseries}}
									\STATE retourner 'p' : timeseries[1] - timeseries [0] \COMMENT{periodic}
								\ELSE
									\STATE return 'f':timeseries[0]:...:timeseries[size-1] \COMMENT{flickering}
								\ENDIF							
							\ENDIF
						\ENDIF
					\ENDIF
					
				\end{algorithmic}
			\end{algorithm}
			 
			 
			 \vskip 0.1in
			 
			 Une dernière phase consiste en la codification du graphe en utilisant le codage proposé pour chaque sous-structure de la phase (02). Nous concaténons par la suite ces représentations pour obtenir une seule représentation concise du graphe initial en terme de ses structures les plus denses. Nous schématisons ces différentes étapes dans la figure \ref{ddsmScheme}.
			 
			 

			  
			L'algorithme \ref{alg:DDSM} représente l'algorithme de compression globale.
			\begin{algorithm}
					
					\caption{DDSM}
					\label{alg:DDSM}
					%\label{Pseudo Algorithme de la méthode proposée (DDSM)}
				\begin{algorithmic} [1]
					\STATE \textbf{Génération des sous structures candidates: }Génération de sous-graphes, principalement les bicliques et les cliques.
					
					\STATE  \textbf{Étiquetage de sous-structures: }Associer chaque sous-structure à une signature temporelle décrivant son comportement.
					
					\STATE \textbf{Codification du graphe compressé: }Codifier les sous-structures étiquetées de l'étape (02) en utilisant les structures de la section précédente.
				\end{algorithmic}
			\end{algorithm}
			
			
		Pour les algorithmes de parcours et d'extraction de voisins, tous les algorithmes proposés dans \citep{hernandez2014compressed} peuvent être généralisés sur notre structure. En effet, le seul changement consiste en la  prise en compte de l'information temporelle dans l'incrémentation des indices de parcours. De ce fait, nous pensons que notre méthode peut offrir un très bon compromis entre l'espace mémoire et le temps d'accès des traitements dans le cas des graphes dynamiques. 
		
		\begin{figure}[H]
\centering
	\includegraphics[scale=0.4]{./ressources/image/ddsm_scheme.png}
	\caption[Les différentes phases de DDSM]{Les différentes phases de DDSM}
	\label{ddsmScheme}
\end{figure}
			
		%\subsubsection{Conclusion}
	%Dans ce chapitre, nous avons abordé le problème de compression des graphes dynamiques par extraction de motifs. Nous avons formalisé le problème de compression d'un graphe dynamique en se basant sur les sous-graphes denses temporelles qui le composent. Pour faire face à ce problème nous avons présenté une nouvelle méthode hybride intitulée DDSM, elle combine deux méthodes de la littérature que nous avons jugé efficaces. Notre prochain but est de tester cette méthode en la comparant avec les méthodes existantes.   

	
	
	\section{Conclusion}

Dans ce chapitre, nous avons expliqué la partie conceptuelle de nos deux moteurs: $k^2$-GraCE et P-GraCE, qui se représentent les deux classes qui intéressent notre recherche. Nous avons présenter leurs différents modules et fonctionnalités tout en clarifiant leurs paramètres ainsi que leurs principe de fonctionnement.

	Dans le chapitres suivant, nous présenteront les choix d'implémentation que nous avons effectué. Nous décrirons l'environnement de développement ainsi que l'architecture globale de notre projet. 
	
	
	
	
	
\chapter{Implémentation}
	\section{Introduction}
	
	Notre projet s'intègrent dans un projet de recherche scientifique. De ce fait, une implémentation efficace et optimisée de notre conception est nécessaire. Nous commencerons ce chapitre par présenter l'architecture existante et comment nous l'avons étendue avec nos deux moteurs tout en détaillant les différentes couches de la nouvelle architecture. Nous conclurons par présenter l'environnement de développement.
	
	
	\section{Architecture globale}
	
	L'architecture existante est une architecture en pipeline de 3-tiers. Elle se compose de trois couches logicielles que nous avons enrichie avec nos deux moteurs et la couche présentation. Nous les présenterons ci-dessous:
	
	\begin{enumerate}
	\item \textbf{La couche données ou persistance :} cette couche est responsable de la gestion des données en entrée et les fonctions d'accès et de stockage. Les données en entrées ainsi que les résultats de compression sont sous forme de fichiers.
	
	\item \textbf{La couche traitement :} c'est le noyau des moteur de compression de notre projet. Elle inclue tout les algorithmes de compression et de manipulation des graphes. 
	\item \textbf{La couche présentation :} cette couche permet de faciliter l'utilisation de notre solution à travers une interface simple et intuitive.
	\end{enumerate}
	
	
	
\begin{figure}[H]
	\centering
	\label{Img:archglob}
	\includegraphics[scale=0.35]{ressources/image/ArchGlob.jpg}
	\caption{Architecture globale de la solution}

 \end{figure}
	
	
	
	\section{Données}
	Les données manipulées par notre solution sont tous sous format de fichiers textes. La structure de ses fichiers diffèrent selon le type de graphe en entrée. Comme nous l'avons déjà implicitement mentionnées, notre solution prend en charge les types de graphes suivant : graphe dynamique orienté, graphe statique orienté, graphe statique non orienté et finalement graphe statique étiqueté. Nous présenterons dans ce qui suit ses structures de fichiers:
	
	\begin{itemize}[label=$\bullet$]
	\item 1
	\item 2
	\item 3
	\item 4
	
	\end{itemize}
	
	
	
	\section{Traitement}
	\section{Présentation}
	
	La couche présentation est responsable de donner accès à toutes les fonctionnalités offertes par notre solution, entre autres le choix des paramètres pour différentes méthodes, ainsi que de visualiser les résultats. En plus du fichier de sortie, elle fournit les différentes mesures de performances. Elle offre ainsi aux chercheurs la possibilité de comparer les performances de différents méthodes sur un même graphe. Cette couche permet d'utiliser notre solution sans avoir recours à chaque fois au guide d'utilisation pour trouver les commandes d'exécution.
	
	\section{Environnement de développement}
		
		Le choix des outils de développement dans tout projet informatique est très important vue leur fort impact sur les performances du produit finale. Comme notre solution fait partie d'un projet de recherche, il faut aussi prendre en compte la flexibilité et la capacité de la solution à s'interfacer avec d'autre outils dans des systèmes qui peuvent être hétérogènes. 
		
		\subsection{Langage de Programmation}
		Nous avons adopté le langage de programmation avec lequel la    première version de notre solution a été développée: le C++. En effet, le langage C++ est un langage très performants pour les calculs lourds. Il permet d'avoir des exécutions très rapides, ce qui en fait un langage de choix pour les applications critiques qui ont besoin de performances. Il permet aussi d'avoir un code portable : un même code source peut être facilement transformé en exécutable sous Windows, Mac OS ou Linux. Un autre aspect du langage c++ est sa richesse de bibliothèques optimisées pour le traitement et le stockage des grands graphes en mémoire. 
		
		Nous avons choisi Visual Studio 2015 comme environnement de développement (IDE). Notre choix a été influencé par le fait que Visual Studio s'ouvre à toutes les tendances du moment et permet facilement de travailler en équipe sur le même projet.
		
		\newacronym{ide}{IDE}{Integrated Developement Environement} 
		\subsection{Bibliothèque Snap}
		
		\newacronym{snap}{SNAP}{Stanford Network Analysis Package}		
		Afin de faciliter la manipulation des graphes et d'améliorer les performances de notre solution, nous avons offert la possibilité d'exécuter les différents algorithmes de compression en utilisant une des plus performantes bibliothèque de manipulation de graphes en c++ : SNAP.
		\gls{snap} est une bibliothèque d'analyse et d'exploration de graphes à usage général qui s'adapte facilement à des graphes massives. Elle présente aussi l'avantage d'être  efficace et facilement extensible. Elle prend naturellement en charge les graphes riches avec des types de données complexes associés aux nœuds et aux arêtes. 
		
		\subsection{Bibliothèque Boost}
		

Boost est un ensemble de bibliothèques pour le langage de programmation C ++ qui prend en charge des tâches et des structures telles que l'algèbre linéaire, la génération de nombres pseudo-aléatoires, le multithreading, les expressions régulières et les tests unitaires. Elle contient plus de quatre vingt bibliothèques individuelles.

Nous l'avons utilisées dans notre projet pour l'implémentation des arbres $k^2$-trees. En effet, elle contient une implémentation des tableaux de bits en c++ qui donnent un temps d'exécution optimale parmi toutes les autres implémentations existantes \citep{pieterse2010performance}.  



		\subsection{Bibliothèque Qt }
	\section{Conclusion}
	
Durant ce chapitre, nous avons présenté notre implémentation où nous avons essayer d'utiliser différentes bibliothèques permettant d'avoir de meilleurs performances. L'indépendance des différentes couches de l'architecture existantes nous ont faciliter la tache d'implémentation. Nous avons ainsi essayer de respecter cette indépendance afin de faciliter tout autre contribution future dans le projet. 

Nous évaluerons dans le chapitre suivant les différentes méthodes de nos deux moteurs en vue de l'obtention d'une étude comparative plus objective et plus clair.



\chapter{Test}
	\section{Introduction}
	\section{Environnement de Test}
	\section{Présentation des graphes de test}
	\section{Évaluation du moteur $k^2$-GraCE}
			\subsection{Analyse de l'impact du paramètre k}
			\subsection{Étude de l'effet de l'ordre}
			
	\section{Évaluation du moteur P-GraCE}
	
	\section{Comparaison entre les différents moteurs}
	
	\section{Conclusion}

\chapter{Conclusion Générale}








\pagestyle{plain}
\newpage

\begin{appendix}


\chapter{Extraction de voisins dans un arbre $k^2$-tree}
    \label{k2_annexe}
La compression par les arbres $k^2$-trees a connue un large succès dans la communauté scientifique. L'une des causes les plus importantes qui ont favorisé cela est qu'elle permet d'extraire le voisinage des nœuds sans reconstitution du graphe initiale. Cette partie se veut une explication des deux algorithmes d'extraction de voisins directes et inverses. 

Comme nous l'avons déjà expliqué dans la section \ref{k2} de la partie conception   du moteur $k^2$-GraCE, les arbres $k^2$-tree sont représentés en mémoire à l'aide de deux chaines binaires, T et L, qui regroupe le contenue de l'arbre de haut vers le bas et de la gauche vers la droite. Nous fournissons ci-après les algorithmes d'extraction de voisinage \citep{brisaboa2009k}.



\begin{multicols}{2}

\begin{algorithm}[H]
					\label{alg:Direct}
					\caption{Direct}
					\textbf{Entrée :}
						\begin{itemize}[label=$\bullet$]
							\item n : la taille de la sous-matrice
							\item p : l'indice de la ligne 
							\item q : l'indice de la colonne
							\item x : l'indice dans l'arbre
						\end{itemize}
					\textbf{Sortie :} affichage des voisins Directes\\							\noindent\rule{\textwidth}{1pt}
						
						
				\begin{algorithmic} [1]
					\IF {x $\geqslant$ |T|}
						\IF { L[x-|T|] =1 }
							\STATE afficher q
						\ENDIF
					\ELSE
						\IF { x = -1 ou T[x]=1  }
							\STATE y = rank(T,x)*$k^2$ + p/(n/k)
							\FOR{ j =0 ... k-1}
								\STATE Direct(n/k,p mod(n/k),q+ j*(n/k),y+j)
							\ENDFOR
						\ENDIF
					\ENDIF 
					
					
				\end{algorithmic}
			\end{algorithm}


 
\columnbreak
 
\begin{algorithm}[H]
					\label{alg:Inverse}
					\caption{Inverse}
					\textbf{Entrée :}
						\begin{itemize}[label=$\bullet$]
							\item n : la taille de la sous-matrice
							\item q : l'indice de la colonne
							\item p : l'indice de la ligne 
							\item x : l'indice dans l'arbre
						\end{itemize}
					\textbf{Sortie :} affichage des voisins Inverses\\							\noindent\rule{\textwidth}{1pt}
						
						
				\begin{algorithmic} [1]
					\IF {x $\geqslant$ |T|}
						\IF { L[x-|T|] =1 }
							\STATE afficher p
						\ENDIF
					\ELSE
						\IF { x = -1 ou T[x]=1  }
							\STATE y = rank(T,x)*$k^2$ + q/(n/k)
							\FOR{ j =0 ... k-1}
								\STATE Inverse(n/k,q mod(n/k),p+ j*(n/k),y+j*k)
							\ENDFOR
						\ENDIF
					\ENDIF 
					
					
				\end{algorithmic}
			\end{algorithm}
			
\end{multicols}

\chapter{Description des graphes de test}
 \label{tableau des graphes de test}
 \begin{table}[H]
 \begin{tabular}{|C{5cm}|C{4.5cm}|C{4.5cm}|}
		\hline
		\multirow{2}{*}{Graphe de test} & \multicolumn{2}{c|}{Caractéristiques}  \\ \cline{2-3}
				&  Nombre de nœuds & Nombre de liens  \\ \hline	 \hline	
 
eu-2005 & 862 milles de nœuds & 19M de liens \\ \hline
Cond-mat & 36 milles de nœuds & 171 milles de liens \\ \hline
CommNet & \multicolumn{2}{c|}{taille : 225.9MB} \\ \hline
SalesDay & 1 milles de nœuds & \\ \hline
Movielens10M & 10 milles de nœuds & 10M de liens \\ \hline
ASOregon &  13 milles nœuds & 37 milles liens \\ \hline
 Enron &  80 milles nœuds& 288 milles liens \\ \hline
 uk-2002 & 18 milles nœuds & 298 milles liens \\ \hline
 graphe test de GCUPMT &  8 192 milles nœuds & \\ \hline
 Composante chimique &  21 étiquettes & 422 transactions \\ \hline
 NotreDame &  325 milles de nœuds & 1M de liens \\ \hline



\end{tabular}
									\caption{Graphes de test}									\label{comgen}

 								\end{table}
\end{appendix}






\newpage




\bibliographystyle{apalike}
\bibliography{Bibliographie}


\end{document}

%\renewcommand{\thefigure}{\arabic{figure}}
%\setcounter{figure}{0}
%\begin{figure}[H]
%	\centering
%	\includegraphics[scale=1]{ressources/image/LAAS-2016.jpg}
%	\label{fig:figure1}
%	\caption{This is a teste of figure}
	
%\end{figure}








