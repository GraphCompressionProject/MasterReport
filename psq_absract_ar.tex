\documentclass[12pt,a4paper]{report}
\usepackage{arabtex}
\usepackage[utf8]{inputenc}
\usepackage[LFE,LAE]{fontenc}
\usepackage[arabic]{babel}

\usepackage[top=2.5cm, bottom=2.5cm, left=2.5cm, right=2.5cm]{geometry}


\usepackage{fancyhdr}
\fancyhf{} % clear all header and footers
\renewcommand{\headrulewidth}{0pt} % remove the header rule
\fancyfoot[LE,LO]{\thepage} % Left side on Even pages; Right side on Odd pages
\pagestyle{fancy}
\fancypagestyle{plain}{%
  \fancyhf{}%
  \renewcommand{\headrulewidth}{0pt}%
  \fancyhf[ref,lof]{\thepage}%
}

%\pagenumbering{Roman}
\begin{document}


\begin{center}
\thispagestyle{plain}

	\par
	\textbf{
		\vskip 0.5in
		\LARGE ملخص
			 \\[0.35in]}
\end{center}


	\par
  % \newfontfamily\arabicfont[Script=Arabic]{Scheherazade}
\begin{otherlanguage}{arabic}

إننا نعيش في وقتنا الحالي واقعا يعرف تزايد في الكم المعرفي و المعلوماتي مما اقتضى و أوجب استعمال المنحنيات التي اصبحت واسعة الانتشار و اكتسحت عدة ميادين مختلفة انطلاقا من مواقع التواصل الاجتماعي و الاتصالات وصولا لميادين الكمياء و البيولوجيا .  إن  هذه الكمية الهائلة من المعلومات تلزم الرجوع إلى  تقنية قديمة قدم مجال  معالجة البيانات و التي تواجه تحديات جديدة : الضغط ... ضغط المنحنيات هو مجال يخضع فيه الرسم البياني الأولي لتحولات من اجل الحصول على نسخة أصغر  تسمح في  غالبية الحالات بتحسين الوقت اللازم لمعالجة و تحليل البيانات.\\  

سنركز في دراستنا على فئتان من أساليب الضغط : طرق ضغط عن طريق استخراج الأنماط والأساليب المعتمدة على الأشجار. نتيجة لذلك ، نقترح محركي ضغط يشمل كل منهما طريقة أو أكثر من كل فئة. المحرك الأول يجعل من الممكن ضغط الرسم البياني من خلال هياكله الأكثر كثافة والأكثر أهمية . في حين أن المحرك الثاني يستغل خصائص مصفوفة الجوار للحصول على تمثيل مضغوط . نقترح أيضًا طريقة جديدة لضغط الرسوم البيانية الديناميكية تقع عند تقاطع الفئتين المدروستين. في الواقع، فإنه يجعل من الممكن ضغط الرسم البياني من خلال هذه الهياكل الأكثر كثافة مع تتبع الخطأ ، الذي عادة ما يتم عرضه كمصفوفات مجوفة ، في بنية الأشجار .\\

سوف نختتم بدراسة مقارنة لأداء خوارزميات الضغط المختلفة والخوارزمية التي نقترحها حيث نستخدم تقييم مقاييس المتابعة: نسبة الضغط والحجم اللازم لتخزين كل معلومة بالاضافة الى وقت المعالجة. من أجل أن نكون قادرين على إجراء هذه الدراسة بموضوعية ، سيتم اختبار طرق الضغط على معايير معروفة من الرسوم البيانية مستخرجة من وضعيات حقيقية و من مجالات مختلفة. \\\\



 \textbf{ 
 كلمات مفتاحية:} ضغط الرسم البياني ، البيانات الكبيرة ، استخراج الأنماط ، أشجار ، الرسم البياني على الويب.
 
\end{otherlanguage}
 
\newpage

       \end{document}