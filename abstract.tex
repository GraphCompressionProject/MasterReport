
\thispagestyle{plain}
\begin{center}
	\par
	\textbf{
		\vskip 0.5in
		\LARGE 
			Résumé \\[0.15in]
			\addcontentsline{toc}{chapter}{\numberline{}Résumé}
	}
\end{center}
	\par
    
    Nous vivons dans un monde où la quantité d'informations ne cesse d'augmenter et dont la bonne gestion implique l'utilisation des graphes qui se sont répandus dans différents domaines allant des réseaux sociaux et de communication jusqu'aux domaines de la chimie et de la biologie. 	Cette abondance de données générées fait appel à une technique aussi vieille que la discipline de traitement de données mais qui connait de nouveaux défis aujourd'hui : la compression. La compression de graphes est un domaine dans lequel le graphe initial subit des transformations pour en obtenir une version plus réduite et compacte permettant, dans la majorité des cas, d'effectuer les traitements dans un temps nettement meilleur.

 Dans ce travail, nous étudierons les différentes méthodes de compression existantes dans la littérature dans le but  de compléter et d'affiner davantage une classification entamée dans un travail de Master précédent. Nous mettrons l'accent sur les méthodes de compression basées sur l'extraction de motifs ainsi que les méthodes basées sur les $k^2$-trees. Par conséquent, nous proposons d'établir une synthèse bibliographique sur ces deux classes et nous établirons une étude comparative entre les méthodes de ces deux classes selon divers critères. 
 \\\\ \textbf{Mots Clés :} \textit{Compression de graphes, Big Data, Extraction de motifs, K2-trees, Graphe du Web.}
 
\newpage
\thispagestyle{plain}
\begin{center}
	\par
	\textbf{
		\vskip 0.5in
		\LARGE 
			Abstract \\[0.15in]
	}
\end{center}
	\par
    
    We live in a world where the amount of data is constantly increasing and whose good management involves the use of graphs that have spread in different fields from social and communication networks to the fields of chemistry and biology. This abundance of generated data calls for a technique that is as old as the discipline of data processing but which is facing new challenges today: compression. Graph compression is a field in which the initial graph undergoes transformations in order to obtain a smaller and more compact version allowing, in the majority of cases, to perform the processings in a much better time.

 In this work, we will study the different methods of compression existing in the literature in order to complete and further refine a classification started in a previous Master's work. We will focus on compression methods based on pattern extraction as well as $ k^2$ -trees methods. Therefore, we propose to establish a bibliographic synthesis on these two classes and we will establish a comparative study between the methods of these two classes according to various criteria.
 \\\\ \textbf{Key words :} \textit{Graph compression, Big Data, Pattern extraction, K2-trees, Web graph.}

\newpage
 
 %%% Manque resume en arabe 

