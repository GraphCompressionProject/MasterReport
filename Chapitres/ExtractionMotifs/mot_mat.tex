
			Dans une méthode toute récente(GCUPMT), Shah et Rushabh \citep{shah2018graph}  partitionnent les lignes de la matrice d'adjacence en plusieurs blocs ayant la même taille des motifs qui sont dans ce cas sous forme de vecteur prédéfinies. Les blocs sont comparés avec l'ensemble des motifs ce qui entraîne , en cas de correspondance , le remplacement du bloc par un indicateur du motif précédé par un 1 indiquant ainsi que les bits suivants appartenant à un indicateur de motif. Dans le cas contraire, les données brutes sont stockées directement précédés par un 0.
				