Une première méthode de cette classe s'intitule Subdue \citep{ketkar2005subdue}. Elle effectue une recherche \textit{Branch\&Bound} qui commence à partir des sous-structures composées de tous les sommets avec des étiquettes uniques. Les sous-structures sont prolongées de toutes les manières possibles par un sommet et une arête ou par une arête afin de générer des sous-structures candidates. Subdue conserve les instances des sous-structures et utilise l'isomorphisme de graphe pour déterminer les instances de la sous-structure candidate. Les sous-structures sont ensuite évaluées en fonction de leur compression de la longueur de description (DL) du jeu de données. Cette procédure se répète jusqu'à ce que toutes les sous-structures soient prises en compte ou que les contraintes imposées par l'utilisateur ne soient plus vérifiées. A la fin de la procédure, Subdue indique les meilleures sous-structures de compression.
				Le système Subdue fournit également la possibilité d'utiliser la meilleure sous-structure trouvée lors d'une étape de découverte pour compresser le graphe d'entrée en remplaçant les instances de la sous-structure par un seul sommet et en effectuant le processus de découverte sur le  compressé. Cette fonctionnalité génère une description hiérarchique du jeu de données du graphe à différents niveaux d'abstraction en termes de sous-structures découvertes.