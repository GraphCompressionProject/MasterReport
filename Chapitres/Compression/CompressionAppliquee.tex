la compression de graphe a été proposé comme solutions pour le traitement et le stockage des graphe volumineux, elle permet de transformer un grand graphe en un autre plus petit tout en préservant ses propriétés générales et ses composants les plus importants.
Les principaux avantages de la compression sont  \citep{liu2018graph} :
\begin{itemize}

\item Réduction de la taille des données et de l'espace de stockage : De nos jours, les graphes représentant les bases de donnés, les réseaux sociaux et tout types de données numérique accroissent d'une manière exponentiel, ce qui rend leur stockage difficile et couteux en terme d'espace mémoire, les techniques de compression produisent des graphes plus petits qui nécessitent moins d'espace que leurs graphe d'origines. Cela permet aussi de décroitre le nombre d'opérations d'E/S, de réduire les communications entre nœuds dans un environnement distribué et de charger le graphe en mémoire central.   

\item Exécution rapide des algorithmes de traitement et des requêtes sur les graphe : l'exécution des différents algorithmes de traitement sur des graphes volumineux peut avérer couteux en terme de temps et peut ne pas donner les résultats attendues. La compression permet d'obtenir de petits graphes qui peuvent être traiter, analyser et interroger plus efficacement et dans un temps raisonnable. 
  
\item Facilité d'analyse et visualisation du graphe : Les techniques de compression permettent de représenter les donnés et les structures des graphes massives d'une manière plus significative permettant ainsi leurs analyse et leurs visualisation contrairement au graphes d'origines qui ne peuvent même pas être charger en mémoire.  

\item Élimination du bruit : les grands graphes du web sont considérablement bruité, il contiennent des liens et des nœuds erronés, ce bruit peut perturber l'analyse en faussant les résultats et en augmentant la charge de travail liée au traitement des donnés. la compression permet donc de filtrer le bruits et de ne mettre en évidence que les donnés importantes.

\end{itemize}


