 %% classifier selon le type du graphe en entree

	Avec les avancées technologique au fil du temps, plusieurs types de graphes ont connus le jours. En effet, La complexité et la variété des problèmes scientifiques existants modélisés par ces derniers ont poussé les chercheurs à adapter leurs structure selon le problème auquel  ils font face. Durant cette section nous allons définir les principaux types existants.
	
		\begin{itemize}[label=$\circ$]
		
			\item \textbf{Graphe Complet:} Un graphe G = (V , E) est un graphe complet si tous les sommets $v_{i}$ $\in$ V sont adjacents \citep{Pres}. Il est souvent noté $K_{n}$ où n = card(V) \citep{DUT}.
				
			
			\item \textbf{Graphe étiqueté et graphe pondéré:}
			 Un graphe étiqueté G = (V , E , W) est un graphe, qui peut être orienté ou non orienté, dont chacune des arêtes $e_{i}$ $\in$ E est doté d'une étiquette $w_{i}$. Si de plus, $w_{i}$ est un nombre alors G est dit graphe pondéré (valué) \citep{DUT}.
		
			\item \textbf{Graphe simple et graphe multiple:}
			Un graphe G = (V , E) est dit simple si il ne contient pas de boucles et tout pair de sommet est reliée par au plus une arête. Dans le cas contraire, G est dit multiple \citep{IUTLyonInformatique}.
			
			\item \textbf{Graphe connexe:}
			Un graphe non orienté (resp. orienté) est dit connexe (resp. fortement connexe) si pour tout pair de sommets ($v_{i}$, $v_{j}$) il existe un chemin S les reliant \citep{muller}.
		\end{itemize}
		