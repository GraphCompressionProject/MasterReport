  La diversité des domaines faisant appel à la modélisation par des graphes ne cesse d'augmenter, allant des réseaux sociaux aux réseaux électriques et réseaux biologiques et arrivant jusqu'aux World Wide Web. Dans cette partie, nous allons décrire trois domaines d'application les plus répandus des graphes.
	
		\subsection{Graphes des réseaux sociaux:}
		Les réseaux sociaux représentent un lieu d'échange et de rencontre entre individus (entités) et dont l'utilisation est devenue de nos jours une nécessité.  
		Pour représenter les interactions entre ces individus, nous avons généralement besoin d'avoir recours aux graphes où les sommets sont des individus ou des entités et les interactions entre eux sont représentées par des liens. 
		Vue la diversité des interactions sociales, la modélisation de ces réseaux nécessite différents types de graphes: graphes non orientés pour les réseaux sociaux avec des relations  symétriques, graphes orientés pour représenter des relations non symétriques
comme c'est la cas dans les réseaux de confiance, graphes pondérés pour les réseaux sociaux qui contiennent différents niveaux d'intensité dans les relations, ..., etc. \citep{lemmouchi2012etude}
		
		\subsection{Graphes en Bioinformatique:}
		
		La bio-informatique est un domaine qui se trouve à l'intersection de deux grands domaines celui de l'informatique et celui de la biologie. Elle a pour but d'exploiter la puissance de calcul des équipements informatiques pour effectuer des traitements sur des données moléculaires massives \citep{pellegrini2004protein}.
		
		Elle est largement utilisée dans l’analyse des séquences d’ADN et des protéines à travers leur modélisation sous forme de graphe. A titre d'exemple, les graphes non orientés multiples sont un outil de modélisation des réseaux d’interaction protéine-protéine \citep{pellegrini2004protein}, 
		le but dans ce cas est l'étude du comportement d'un protéine par rapport à un autre.
		
		\subsection{Le Graphe du web:}
		 Le graphe du Web est un graphe orienté dont les sommets sont les pages du web et les arcs modélisent l'existence d'un lien hypertexte dans une page vers une autre \citep{brisaboa2009k}. Il représente l'un des graphes les plus volumineux: en juillet 2000 déja, on estimait qu’il contenait environ 2,1 milliards de sommets et 15 milliards d’arêtes avec 7,3 millions de pages ajoutées chaque jour \citep{guillaume2002web}. De ce fait, ce graphe a toujours attiré l'attention des chercheurs. En effet, l'étude de ses caractéristiques a donné naissance à plusieurs algorithmes intéressants, notamment l'algorithme PageRank de classement des pages web qui se trouve derrière le moteur de recherche le plus connu de nos jours : Google.