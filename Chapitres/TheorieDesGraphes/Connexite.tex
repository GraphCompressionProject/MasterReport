Les structures de graphes sont généralement exploitables à travers leurs interrogation qui permet de fournir des réponses aux problèmes modélisés. L'un des informations les plus importantes dans un graphe est la notion des relations (indirectes ou indirecte) entre deux nœuds ou plus formellement la connexité dans un graphe. Dans cette partie nous allons définir les concepts relatives à cette notion.
	
	\begin{itemize} [label = $\bullet$]
		
			 
			 \item \textbf{Chemin (resp. Chaine):}
			est une liste de sommets S= $(v_{0},v_{1},v_{2},...,v_{k})$ telle qu’il existe un arc (resp. une arête) entre chaque couple de sommets successifs.
			 
			 
			  \item \textbf{Cycle (resp. Circuit):} 
			 est un chemin (resp. chaine) dont le premier et le dernier sommet sont identiques \citep{DUT}.
			 
			 		\item \textbf{Graphe connexe:}
			Un graphe non orienté (resp. orienté) est dit connexe (resp. fortement connexe) si pour tout pair de sommets ($v_{i}$, $v_{j}$) il existe un chemin S les reliant \citep{muller}.
			 
		\end{itemize}