 
 
Les graphes sont généralement exploitables à travers leur interrogation qui permet de fournir des réponses aux problèmes modélisés. L'une des informations les plus importantes dans un graphe est la notion de relations (directes ou indirectes) entre deux nœuds ou plus formellement la connexité dans un graphe. Dans cette partie, nous allons définir les concepts relatifs à cette notion dans le cas d'un graphe non orienté (resp. orienté).
 \begin{itemize} [label = $\bullet$]
			 \item \textbf{Chaine (resp. Chemin):}
			est une liste de sommets S= $(v_{0},v_{1},v_{2},...,v_{k})$ tel qu'il existe une arête (resp. un arc) entre chaque couple de sommets successifs \citep{muller}.
			 
			 
			  \item \textbf{Cycle (resp. Circuit):} 
			 est une chaine (resp. chemin) dont le premier et le dernier sommet sont identiques \citep{DUT}.
			 
			 \item \textbf{Graphe connexe:}
			Un graphe non orienté (resp. orienté) est dit connexe (resp. fortement connexe) si pour toute paire de sommets ($v_{i}$, $v_{j}$), il existe une chaine (resp. chemin) S les reliant \citep{muller}.
		
		\end{itemize}