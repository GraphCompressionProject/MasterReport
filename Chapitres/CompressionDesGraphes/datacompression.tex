 La compression de donnée est principalement une branche de la théorie de l'information  qui traite des techniques et méthodes liées à la minimisation de la quantité de données à transmettre et à stocker.
Sa caractéristique de base est de convertir une chaîne de caractères vers un autre jeu de caractères occupant un espace mémoire le plus réduit possible tout en conservant le sens et la pertinence de l'information \citep{lelewer1987data}.

	Les techniques de compression de données sont principalement motivées par la nécessité d'améliorer l'efficacité du traitement de l'information. En effet, la compression des données en tant que moyen peut rendre l'utilisation des ressources existantes beaucoup plus efficace. 
	
	De ce fait, une large gamme d'application usant de ce domaine tel que le domaine des télécommunications et le domaine du multimédia est apparue offrant une panoplie d'algorithmes de compression \citep{sethi2014data}. Sans les techniques de compression, Internet, la télévision numérique, les communications mobiles et les communications vidéo, qui ne cessaient de croître, n'auraient été que des développements théoriques.
	
	%Afin de pouvoir comparer entre ces différentes méthodes de compression, plusieurs critères ont été proposés dans la littérature. Parmi les mesures utilisées on trouve:
	%\begin{itemize}
	%	\item \textbf{Le taux de compression:} représente le rapport entre la taille des données après et avant la compression.
		
		%%% donner la formule
		
		%\item \textbf{Le facteur de compression:} représente l'inverse du taux de compression.
		
		%\item \textbf{Le temps de compression:}
	%\end{itemize}