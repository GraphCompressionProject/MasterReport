
Nous avons étudié dans ce chapitre deux classes de compression de graphes: (i)  Les méthodes de compression par les k2-trees, (ii) Les méthodes de compression par extraction de motifs.
%\begin{itemize}
%	\item (i)  Les méthodes de compression par les k2-trees
%	\item (ii) Les méthodes de compression par extraction de motifs
%\end{itemize} 
Nous avons ainsi présenté le principe de fonctionnement des méthodes de compression relatif à chaque classe tout en comparant entre leurs caractéristiques. 

Cependant une étude comparative entre les deux classes reste primordiale. Dans le tableau \ref{comgen} , nous allons essayer de synthétiser les principales différences et similitudes entre ces deux classes que nous avons put constater à travers notre recherche bibliographique et qui ont un fort impact sur le choix de la méthode de compression.

 
\begin{table}[H]
\begin{tabular}{|c|p{6cm}|p{4cm}|}

\hline & \begin{center}
\textbf{$k^2$-trees}
\end{center}     &  \begin{center} \textbf{Extraction de motifs} \end{center}  \\
										
										
\hline Type de compression & toujours sans perte & toujours sans perte \\
\hline Structure en sortie & Toujours succincte & Peut être succincte où structurelle où les deux en même temps\\

\hline technique utilisée & exploitation de la matrice d'adjacence du graphe & exploitation des motifs fréquents dans le graphe\\

\hline Dépendance & Dépend du paramètre k & Dépend selon la méthode de l'algorithme de clustering ou du vocabulaire de motifs utilisé  \\

\hline Objectif & 
\begin{minipage}[t]{0.35\textwidth}
  			Compression,\\
  			Réduire l'espace de stockage et le temps de parcours\\
  \end{minipage}
  &
  \begin{minipage}[t]{0.25\textwidth}
  			- Compression,\\
  			- Réduire l'espace de  stockage et le temps de parcours,\\
  			- Extraire les informations pertinentes, \\
  			- Visualisation \\
  \end{minipage}
  \\
  \hline Domaine d'application & tous les domaines & tous les domaines \\
  \hline
\end{tabular}
									\caption{Comparaison entre les méthodes basées sur$k^2$-trees et basées sur l'extraction de motifs.}									\label{comgen}
									
								\end{table}
