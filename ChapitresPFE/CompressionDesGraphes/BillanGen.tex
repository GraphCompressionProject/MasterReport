De nos jours, les graphes sont omniprésents. Cependant, leur taille présente un obstacle presque insurmontable à la compréhension du caractère essentiel des données. D'où la nécessité de la compression qui permet de réduire la taille des graphes tout en gardant le caractère utile de l'information incluse dans ces derniers.

Nous avons étudié dans ce chapitre différentes méthodes de compression de graphes dans le but d'établir une classification des méthodes basée sur deux approches : l'extraction de motifs et les arbres k2-trees. Nous nous sommes appuyer pour cela sur le principe de fonctionnement de chacune d'elles. Nous  proposons six classes de méthodes :

\begin{enumerate}[label=\roman*]
\item  Les méthodes de compression par les k2-trees
	\item  Les méthodes de compression par extraction de motifs basées vocabulaires en utilisant des méthodes de clustering.
	\item  Les méthodes de compression par extraction de motifs basées vocabulaires exploitant les propriétés de la matrice d'adjacence.
	\item  Les méthodes de compression par extraction de motifs basées agrégation de nœuds.
	\item  Les méthodes de compression par extraction de motifs basées agrégation de liens en utilisant des règles de grammaire.
	\item  Les méthodes de compression par extraction de motifs basées agrégation de liens en utilisant des heuristiques de clustering.
\end{enumerate}
	
Nous avons présenté le principe de fonctionnement des méthodes relatif à chaque classe tout en comparant leurs caractéristiques. 
 Ces classes diffèrent les unes des autres dans leurs fondements théoriques, le type de compression considéré, la complexité des algorithmes, les objectifs et les domaines d'application. 
 Dans le tableau \ref{comgen} , nous allons essayer de synthétiser les principales différences et similitudes entre les deux approches, compression par extraction de motifs et compression par les arbres k2-trees, que nous avons pu constater à travers notre recherche bibliographique. Ils ont un fort impact sur le choix de la méthode de compression du moment qu'ils ont une influence directe sur les performances.
 
\begin{table}[H]
\begin{tabular}{|c|p{6cm}|p{4cm}|}

\hline & \begin{center}
\textbf{$k^2$-trees}
\end{center}     &  \begin{center} \textbf{Extraction de motifs} \end{center}  \\
										
										
\hline Type de compression & toujours sans perte & toujours sans perte \\
\hline Structure en sortie & Toujours succincte & Peut être succincte où structurelle où les deux en même temps\\

\hline technique utilisée & exploitation de la matrice d'adjacence du graphe & exploitation des motifs fréquents dans le graphe\\

\hline Dépendance & Dépend du paramètre k & Dépend selon la méthode de l'algorithme de clustering ou du vocabulaire de motifs utilisé  \\

\hline Objectif & 
\begin{minipage}[t]{0.35\textwidth}
  			Compression,\\
  			Réduire l'espace de stockage et le temps de parcours\\
  \end{minipage}
  &
  \begin{minipage}[t]{0.25\textwidth}
  			- Compression,\\
  			- Réduire l'espace de  stockage et le temps de parcours,\\
  			- Extraire les informations pertinentes, \\
  			- Visualisation \\
  \end{minipage}
  \\
  \hline Domaine d'application & tous les domaines & tous les domaines \\
  \hline Type de graphes supporté & 
 \begin{minipage}[t]{0.25\textwidth}
  			- Statique orienté,\\
  			- Attribué, étiqueté\\
  			- Dynamique, \\
  \end{minipage}  
  &  \begin{minipage}[t]{0.25\textwidth}
  			- Statique (orienté et non orienté),\\
  			- étiqueté,\\
  			- Dynamique, \\
  \end{minipage}  
  \\ \hline
\end{tabular}
									\caption{Comparaison entre les méthodes basées sur$k^2$-trees et basées sur l'extraction de motifs.}									\label{comgen}
									
								\end{table}
								
				
	