			Dans une méthode toute récente 
\newacronym{gcupmt}{GCUPMT}{Graph Compression Using Pattern Matching}			
			s'intitulant \gls{gcupmt}, Shah et Rushabh \citep{shah2018graph}  partitionnent les lignes de la matrice d'adjacence en plusieurs blocs ayant la même taille des motifs qui sont dans ce cas sous forme de vecteurs prédéfinies. Les blocs sont comparés avec l'ensemble des motifs ce qui entraîne , en cas de correspondance , le remplacement du bloc par un indicateur du motif précédé par un 1 indiquant que les bits suivants appartiennent à un indicateur de motif. Dans le cas contraire, les données brutes sont stockées directement précédées par un 0.
				