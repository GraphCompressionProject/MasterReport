 %% classifier selon le type du graphe en entree
	Avec les avancées technologiques au fil du temps, plusieurs types de graphes ont vu le jour. En effet, la complexité et la variété des problèmes scientifiques existants modélisés par ces derniers ont poussé les chercheurs à adapter leur structure selon le problème auquel ils font face. Durant cette section, nous allons définir les principaux types existants.
	
		\begin{itemize}[label=$\circ$]
		
			\item \textbf{Graphe Complet:} Un graphe G = (V , E) est un graphe complet si tous les sommets $v_{i}$ $\in$ V sont adjacents \citep{Pres}. Il est souvent noté $K_{n}$ où n = card(V) \citep{DUT}.
				
			
			\item \textbf{Graphe étiqueté et graphe pondéré:}
			 Un graphe étiqueté G = (V , E , W) est un graphe non orienté (resp. orienté) dont chacune des arêtes (resp. arcs) $e_{i}$ $\in$ E est doté d'une étiquette $w_{i}$. Si de plus, $w_{i}$ est un nombre alors G est dit graphe pondéré (valué) \citep{DUT}.
		
			\item \textbf{Graphe simple et graphe multiple:}
			Un graphe G = (V , E) non orienté (resp. orienté) est dit simple s'il ne contient pas de boucles et toute paire de sommets sont reliés par au plus une arête (resp. un arcs). Dans le cas contraire, G est dit multiple \citep{IUTLyonInformatique}.
			
		
		\end{itemize}
		