
\subsubsection{Synthèse des méthodes de compression basée sur les $K^2$-trees }

Nous avons présenté dans les parties précédentes les différentes méthodes de compression existantes basées sur la représentation $k^2$-trees, et expliqué le principe de fonctionnement de chaqu'une d'elles. Nous allons par la suite comparer entre ces méthodes et résumer nos recherches.

Une présentation synthétique de notre étude est fournie dans le tableau \ref{K2-trees-table}. Chaque ligne du tableau représente une méthode, tandis que chaque colonne
représente un aspect susceptible d’être utilisé dans la méthode (type de graphe, type de compression, structure en sortie). Nous constatons que toutes les méthodes de cette classe sont des méthodes de compression sans perte qui supportent les graphes orientés et non orientés, cependant nous remarquons une variation dans l'aspect temporelle, certains méthodes sont destinées aux graphes statiques comme $k^2$-trees de base et $k^2$-trees1, tandis que d'autres sont appliquées aux graphes dynamiques comme $k^n$-trees et d$k^2$-trees. Certains méthodes se distinguent aussi par le type de graphe comme $k^2$-treaps qui s'applique aux graphes pondérés et Att$k^2$-trees qui accepte les graphes étiquetés, attribués et multiples. Nous observons aussi que toutes les méthodes donnent en sortie une représentation succincte. Le tableau résume aussi les résultats de l'application des méthodes sur des graphes de test. \\


	


    
 												
														\begin{landscape}
								\begin{table}
									\begin{tabular}{|C{3cm}|c|c|c|c|c|c|c|c|c|c|c|c|c|}
										\hline
										\multirow{2}{*}[-25pt]{     Article   }  & \multicolumn{5}{c|}{Graphe en entrée} & \multicolumn{2}{c|}{Compression} & \multicolumn{2}{c|}{Structure en sortie} & \multirow{2}{*}[-25pt]{Graphe de test} & \multirow{2}{*}[-25pt]{Résultat }  \\ \cline{2-10}
				     & \rotatebox[origin=c]{90}{ Orienté }  & \rotatebox[origin=c]{90}{ Non orienté } & \rotatebox[origin=c]{90}{ Statique } & \rotatebox[origin=c]{90}{ Dynamique } & \rotatebox[origin=c]{90}{ Autre Propriétés } &  \rotatebox[origin=c]{90}{ Avec perte } & \rotatebox[origin=c]{90}{ Sans perte } & \rotatebox[origin=c]{90}{ Succincte } & \rotatebox[origin=c]{90}{ Structurelle } & & \\ \hline				%%%%%%Fin du header
				
\hline  $k^2$-trees : Algorithme de base
   \citep{brisaboa2009k} 
   & \cmark & \cmark & \cmark & \xmark &  & \xmark &  \cmark & \cmark & \xmark	 & 		
	\begin{minipage}[t]{0.15\textwidth}
	eu-2005\\
	
	cond-mat
  \end{minipage}	
										 &
	\begin{minipage}[t]{0.4\textwidth}
	 5.21 bits/lien \\
	 taux de compression : 16.88\% \\
	 taux de compression : 15.58\% 
  \end{minipage}	\\

\hline $k^2$-trees : Hybridation \citep{brisaboa2009k} & \cmark & \cmark & \cmark & \xmark &  & \xmark &  \cmark & \cmark & \xmark  & 
										\begin{minipage}[t]{0.15\textwidth}
	eu-2005
  \end{minipage}	
										 &
	\begin{minipage}[t]{0.4\textwidth}
	 5.21 bits/lien 
  \end{minipage}	\\  
\hline $k^2$-trees : Optimisation \citep{shi2012optimizing} & \cmark & \cmark & \cmark & \xmark &  & \xmark &  \cmark & \cmark & \xmark  & 
\begin{minipage}[t]{0.15\textwidth}
	cond-mat
  \end{minipage}	
										 &
	\begin{minipage}[t]{0.4\textwidth}
	
	 taux de compression : 37.96\% 
  \end{minipage}	\\
  			\hline
  			
\hline $k^2$-trees : Amélioration \citep{brisaboa2014compact} & \cmark & \cmark & \cmark & \xmark &  & \xmark &  \cmark & \cmark & \xmark  & 
  				\begin{minipage}[t]{0.15\textwidth}
	eu-2005
  \end{minipage}	
										 &
	\begin{minipage}[t]{0.4\textwidth}
	
	 3.22 bits/lien
  \end{minipage}	\\
  				
  				\hline  		
  			
\hline d$k^2$-trees \citep{brisaboa2012compressed} & \cmark & \cmark & \xmark & \cmark &  & \xmark &  \cmark & \cmark & \xmark  &
  				\begin{minipage}[t]{0.15\textwidth}
	eu-2005
  \end{minipage}	
										 &
	\begin{minipage}[t]{0.4\textwidth}
	
	 6.2 bits/lien
  \end{minipage}	\\
  \hline 
  			
\hline $k^n$-trees \citep{de2013compact} & \cmark & \cmark & \xmark & \cmark & & \xmark & \cmark & \cmark & \xmark  &
  				\begin{minipage}[t]{0.15\textwidth}
	CommNet
  \end{minipage}	
										 &
	\begin{minipage}[t]{0.4\textwidth}
	
	 taux de compression : 65.16\% 
  \end{minipage}	\\
  \hline  	

  			

  					
 	
  			
									\end{tabular}
									\caption{Synthèse des méthodes de compression par $k^2$-trees.}									
									
								\end{table}
								
							\end{landscape}
							
							
							
%% deuxiéme tableau

					\begin{landscape}
								\begin{table}
									\begin{tabular}{|C{3cm}|c|c|c|c|c|c|c|c|c|c|c|c|c|}
										\hline
										\multirow{2}{*}[-25pt]{Article}  & \multicolumn{5}{c|}{Graphe en entrée} & \multicolumn{2}{c|}{Compression} & \multicolumn{2}{c|}{Structure en sortie} & \multirow{2}{*}[-25pt]{Graphe de test} & \multirow{2}{*}[-25pt]{Résultat }  \\ \cline{2-10}
				& \rotatebox[origin=c]{90}{ Orienté }  & \rotatebox[origin=c]{90}{ Non orienté } & \rotatebox[origin=c]{90}{ Statique } & \rotatebox[origin=c]{90}{ Dynamique } & \rotatebox[origin=c]{90}{ Autre Propriétés } &  \rotatebox[origin=c]{90}{ Avec perte } & \rotatebox[origin=c]{90}{ Sans perte } & \rotatebox[origin=c]{90}{ Succincte } & \rotatebox[origin=c]{90}{ Structurelle } & & \\ \hline				%%%%%%Fin du header
				

  			
\hline $k^2$-trees1\citep{de2014new} & \cmark & \cmark & \cmark & \xmark & & \xmark & \cmark & \cmark & \xmark  & 
  							\begin{minipage}[t]{0.1\textwidth}
	eu-2005
  \end{minipage}	
										 &
	\begin{minipage}[t]{0.35\textwidth}
	 taux de compression : 16.23\% 
  \end{minipage}	\\
  \hline  
  			
\hline Delta-$k^2$-trees  \citep{zhang2014delta} & \cmark & \cmark & \cmark & \xmark & & \xmark & \cmark & \cmark & \xmark  & 
  							\begin{minipage}[t]{0.1\textwidth}
	eu-2005
  \end{minipage}	
										 &
	\begin{minipage}[t]{0.35\textwidth}
	 3.24 bits/lien 
  \end{minipage}	\\
  \hline  
  
\hline $k^2$-treaps  \citep{brisaboa2014k} & \cmark & \cmark & \cmark & \xmark & 
\begin{minipage}[t]{0.15\textwidth}
  			Pondéré 
  \end{minipage}		
   & \xmark & \cmark & \cmark & \xmark  & 
  							\begin{minipage}[t]{0.1\textwidth}
	SalesDay
  \end{minipage}	
										 &
	\begin{minipage}[t]{0.35\textwidth}
	 2.48 bits/lien
  \end{minipage}	\\
  \hline  
 
\hline I$k^2$-trees  \citep{garcia2014interleaved} & \cmark & \cmark & \xmark & \cmark & & \xmark & \cmark & \cmark & \xmark  & 
  							\begin{minipage}[t]{0.1\textwidth}
	CommNet
  \end{minipage}	
										 &
	\begin{minipage}[t]{0.35\textwidth}
	
	 taux de compression : 60.43\% 
  \end{minipage}	\\
  \hline  	
  			
 
\hline Diff I$k^2$-trees  \citep{alvarez2017succinct} & \cmark & \cmark & \xmark & \cmark & & \xmark & \cmark & \cmark & \xmark  & 
  							\begin{minipage}[t]{0.1\textwidth}
	CommNet
  \end{minipage}	
										 &
	\begin{minipage}[t]{0.35\textwidth}
	
	 taux de compression : 60.43\% 
  \end{minipage}	\\
  \hline 
				
				 
\hline Att $k^2$-trees  \citep{alvarez2018compact} & \cmark & \cmark & \cmark & \xmark & 
\begin{minipage}[t]{0.1\textwidth}
  			Étiqueté\\
  			Attribué\\
  			Multiple
  \end{minipage}	
& \xmark & \cmark & \cmark & \xmark  & 
  							\begin{minipage}[t]{0.15\textwidth}
	Movielen-10M
  \end{minipage}	
										 &
	\begin{minipage}[t]{0.35\textwidth}
	
	 taux de compression : 89.97\% 
  \end{minipage}	\\
  \hline 
  			
\hline dynAtt $k^2$-trees  \citep{alvarez2018compact} & \cmark & \cmark & \xmark & \cmark & 
\begin{minipage}[t]{0.1\textwidth}
  			Étiqueté\\
  			Attribué\\
  			Multiple
  \end{minipage}	
& \xmark & \cmark & \cmark & \xmark  & 
			\begin{minipage}[t]{0.1\textwidth}
	Movielen-10M
  \end{minipage}	
										 &
	\begin{minipage}[t]{0.35\textwidth}
	
	 taux de compression : 93.75\% 
  \end{minipage}	\\  				
  				\hline 
				
									\end{tabular}
									\caption{Synthèse des méthodes de compression par $k^2$-trees.}									
								\label{K2-trees-table}
								\end{table}
								
							\end{landscape}			
							
							
	